\documentclass{amsart}

\usepackage[colorlinks=true]{hyperref}
\usepackage{enumerate}
\usepackage{color}
\usepackage{mathrsfs}
\usepackage{tikz-cd}
\usepackage{amssymb}

\theoremstyle{plain}
\newtheorem{theorem}{Theorem}
\newtheorem{lemma}[theorem]{Lemma}
\newtheorem{proposition}[theorem]{Proposition}
\newtheorem{corollary}[theorem]{Corollary}

\theoremstyle{definition}
\newtheorem{definition}[theorem]{Definition}
\newtheorem{example}[theorem]{Example}
\newtheorem{exercise}[theorem]{Exercise}

\theoremstyle{remark}
\newtheorem{remark}[theorem]{Remark}

% Fonts
\newcommand{\A}{\mathbb{A}}
\newcommand{\C}{\mathbb{C}}
\newcommand{\R}{\mathbb{R}}
\newcommand{\Q}{\mathbb{Q}}
\newcommand{\Z}{\mathbb{Z}}
\newcommand{\N}{\mathbb{N}}
\newcommand{\G}{\mathbb{G}}
\newcommand{\fr}{\mathfrak}

% Topology/geometry

\DeclareMathOperator{\Gr}{Gr}
\DeclareMathOperator{\Fl}{Fl}
\DeclareMathOperator{\PP}{\mathbb{P}}
\DeclareMathOperator{\Der}{Der}
\DeclareMathOperator{\Lie}{Lie}
\DeclareMathOperator{\SL}{SL}
\DeclareMathOperator{\GL}{GL}
\DeclareMathOperator{\SO}{SO}
\DeclareMathOperator{\Sp}{Sp}
\DeclareMathOperator{\HH}{H}
\DeclareMathOperator{\Symp}{Symp}

% Representation theory

\DeclareMathOperator{\Ad}{Ad}

% Algebra

\DeclareMathOperator{\End}{End}
\DeclareMathOperator{\Aut}{Aut}
\DeclareMathOperator{\Hom}{Hom}
\DeclareMathOperator{\sHom}{\mathscr{H}\!om}
\DeclareMathOperator{\sEnd}{\mathscr{E}\!nd}
\DeclareMathOperator{\id}{id}
\DeclareMathOperator{\irr}{irr}
\DeclareMathOperator{\Diff}{Diff}
\DeclareMathOperator{\gr}{gr}
\DeclareMathOperator{\im}{im}
\DeclareMathOperator{\ad}{ad}
\DeclareMathOperator{\rk}{rk}
\DeclareMathOperator{\Spec}{Spec}
\DeclareMathOperator{\Specm}{Specm}
\DeclareMathOperator{\Stab}{Stab}
\DeclareMathOperator{\Sym}{Sym}

% Category theory

\DeclareMathOperator*{\colim}{colim}

\makeatletter
\renewcommand\d[1]{\mspace{6mu}\mathrm{d}#1\@ifnextchar\d{\mspace{-3mu}}{}}
\makeatother



\DeclareMathOperator{\ord}{ord}
\DeclareMathOperator{\Div}{Div}

\title{Notes on divisors and Riemann-Roch}
\author{Nilay Kumar}
\date{Summer 2016}

\begin{document}
\maketitle

Recall that due to the maximum principle, there are no nonconstant holomorphic functions on a
compact complex manifold. The next best objects to study, as we learn in complex analysis of a single
variable, are meromorphic functions. On $\C$ we say a function $f$ is meromorphic if it is holomorphic
outside an isolated set of points, at which it has poles. We generalize this definition as follows.

\begin{definition}
    Let $X$ be a complex manifold. A {\color{blue}meromorphic function} $f$ on $X$ is the data of a
    covering $\{U_i\}_{i\in I}$ of $X$ with $f|_{U_i}=g_i/h_i$ for $g_i,h_i\in\mathcal{O}(U_i)$
    relatively prime and $g_ih_j=g_jh_i$ in $\mathcal{O}(U_i\cap U_j)$. We will denote the sheaf of
    meromorphic functions by $\mathcal{M}$.
\end{definition}

\begin{remark}
    Strictly speaking, meromorphic functions are not functions (qua maps to $\C$ or even $\C P^1$). Consider,
    for instance, the function $f(z_1,z_2) = z_1/z_2$ on $\C^2$. Clearly $f$ has no value at the origin.
\end{remark}

\begin{exercise}
    Show that in the case of Riemann surfaces, however,  meromorphic functions can indeed be seen as
    holomorphic maps to $\C P^1$.
\end{exercise}

Consider now the following analytic question -- the {\color{blue} second Cousin problem}:
Let $\fr U=\{U_\alpha\}$ be an open cover of a complex manifold $X$ and suppose we have
meromorphic functions $f_\alpha$ defined on the corresponding opens, such that $f_{\alpha\beta}=f_\alpha f_\beta^{-1}$
is holomorphic and nonvanishing on $U_\alpha\cap U_\beta$. Does there exist a global meromorphic function on
$X$ such that $f f_\alpha^{-1}$ is holomorphic and nowhere vanishing on $U_\alpha$?

Let us reformulate the problem in the language of \v Cech cohomology, as we will find a connection
to divisors and line bundles. Recall that the \v Cech cochain complex of a cover $\fr U$ for a sheaf
$\mathcal{F}$ is defined as
\begin{equation*}
    \check C^p(\fr U,\mathcal{F}) =\prod_{|I|=p}\mathcal{F}(U_{i_0,\ldots,i_p})
\end{equation*}
with the differential given by
\begin{equation*}
    (\partial^p\mu)(U_{i_0,\ldots,i_{p+1}}) = \sum_{j=0}^{p+1} (-1)^j\mu(U_{i_0,\ldots,i_p})|_{U_{i_0,\ldots,\hat i_j,\ldots, i_{p+1}}}
\end{equation*}
The \v Cech cohomology groups are then defined in the usual way as the kernels modulo the images of the
differentials.

In our case, there are two sheaves of interest: $\mathcal{M}^*$, the sheaf of abelian groups of nontrivial meromorphic functions
and $\mathcal{O}^*$, the sheaf of abelian groups of nowhere vanishing holomorphic functions. The data given in the
statement of the second Cousin problem is that of a zero-cocycle $\mu\in\check H^0(\fr U, \mathcal{M}^*/\mathcal{O}^*)$
of the quotient sheaf:\footnote{I am sweeping some details/technicalities about quotient sheaves under the rug to simplify
the presentation. Everything is probably fine if we assume that the cover is ``fine'' enough.}
\begin{equation*}
    (\partial^0 \mu)(U_{\alpha\beta}) = f_\alpha f_\beta^{-1} \equiv 0\mod \mathcal{O}^*.
\end{equation*}
The goal is to find a global section $f\in\check H^0(\fr U, \mathcal{M}^*)$ whose image under the quotient map
$q: \mathcal{M}^* \to \mathcal{M}^*/\mathcal{O}^*$ is the given $\mu$. Notice, now, that we have a short
exact sequence of sheaves,
\begin{equation*}
    \begin{tikzcd}
        0\rar & \mathcal{O}^* \rar & \mathcal{M}^*\rar{q} & \mathcal{M}^*/\mathcal{O}^*\rar & 0,
    \end{tikzcd}
\end{equation*}
which induces a long exact sequence on cohomology (recall that a surjective map of sheaves need not be
surjective on sections -- this is the reason the Cousin problem is nontrivial!):
\begin{equation*}
    \begin{tikzcd}
        0\rar & \check H^0(\fr U, \mathcal{O}^*) \rar & \check H^0(\fr U, \mathcal{M}^*)\rar & \check H^0(\fr U, \mathcal{M}^*/\mathcal{O}^*)\rar{\delta} & \check H^1(\fr U,\mathcal{O}^*)\rar & \cdots
    \end{tikzcd}
\end{equation*}
Exactness at $\check H^0(\fr U,\mathcal{M}^*/\mathcal{O}^*)$ implies that there exists a solution to the Cousin
problem for $\mu$ if and only if $\delta \mu=0\in \check H^1(\fr U, \mathcal{O}^*)$. It is sufficient, for
instance for this first cohomology group $\check H^1(\fr U, \mathcal{O}^*)$ to vanish.

Hence we have translated the analytic problem of globalizing certain meromorphic functions to a cohomological problem,
which in certain cases may be more straightforward. Consider again the long exact sequence above and recall that
the set of holomorphic line bundles is in bijection with the sheaf cohomology group $\HH^1(X,\mathcal{O}^*)$.
Meanwhile, the group $\HH^1(X, \mathcal{M}^*/\mathcal{O}^*)$ will turn out to be the group of divisors,
which encapsulates codimension-one geometry of $X$.
So we see that the second Cousin problem ties together meromorphic functions, divisors, and line bundles --
we will focus today on the topic of divisors, returning to line bundles for the next talk.

\begin{definition}
    An {\color{blue} analytic hypersurface} $Y$ of a complex manifold $X$ is a closed subset locally given
    as the vanishing locus of a holomorphic function. A {\color{blue} divisor} on $X$ is a formal linear combination
    \begin{equation*}
        D = \sum_i a_i[Y_i]
    \end{equation*}
    with $Y_i$ irreducible hypersurfaces and $a_i\in\Z$, such that the sum is locally finite: for any $x\in X$
    there exists an open neighborhood $U$ on which all but finitely many $a_i$ satisfy $Y_i\cap U=\varnothing$.
    We denote the group of divisors on $X$ by $\Div X$.
\end{definition}

\begin{remark}
    Every hypersurface $Y$ with irreducible components naturally determines a divisor $\sum_i[Y_i]$. Conversely,
    one can assign to a divisor $\sum_i a_i[Y_i]$ the hypersurface $\cup_i Y_i$, though this assignment is not
    particularly natural. Notice, moreover, that in the case of Riemann surfaces, analytic hypersurfaces are
    sets of isolated points, and so divisors take on a particularly simple form.
\end{remark}

A special class of divisors are those that come from meromorphic functions.
\begin{definition}
    Let $f\in\mathcal{M}^*(X)$ be a meromorphic function and $Y\subset X$ be an irreducible hypersurface.
    With a little bit of work, we can define the {\color{blue}order of vanishing of $f$ along $Y$}, $\ord_Y f\in\Z$.
    See, for instance, Huybrechts, though the definition for Riemann surfaces is the usual notion of the
    order of vanishing at a point from complex analysis.

    The {\color{blue}principal divisor associated to $f$} is defined as
    \begin{equation*}
        (f) = \sum_Y (\ord_Y f)[Y],
    \end{equation*}
    where the sum is taken over all irreducible analytic hypersurfaces $Y\subset X$. Notice that the set
    of principal divisors forms a subgroup of $\Div X$ as $(f)+(g)=(fg)$.
\end{definition}

Let us prove the earlier claim relating the Cousin problem to divisors:
\begin{proposition}
    Let $X$ be a complex manifold.
    There exists a natural isomorphism
    \begin{equation*}
        \HH^0(X, \mathcal{M}^*/\mathcal{O}^*) \cong \Div X.
    \end{equation*}
\end{proposition}
\begin{proof}
    Fix an element $\mu\in\HH^0(X, \mathcal{M}^*/\mathcal{O}^*)$, i.e. meromorphic functions $f_\alpha$ on
    open subsets $U_\alpha$ (of some open cover) such that $f_\alpha f_\beta^{-1}$ is nonvanishing holomorphic
    on $U_{\alpha\beta}$. Then for any irreducible hypersurface $Y\subset X$ intersecting $U_{\alpha\beta}$
    nontrivially, $\ord_Y f_\alpha = \ord_Y f_\beta$. Hence we can define an integer $\ord_Y \mu$ to be the order
    of vanishing of $\mu$ along $Y$. Associate to $\mu$ the divisor $\sum_Y (\ord_Y \mu)[Y]$. 
    The product $\mu \cdot \nu$ is then taken to the sum $\sum_Y(\ord_Y \mu + \nu)[Y]$ whence we obtain
    a group homomorphism.

    Conversely, fix a divisor $\sum_i a_i[Y_i]$. There exists an open cover $\{U_\alpha\}$ of $X$ such that
    $Y_i\cap U_\alpha$ is defined by some $g_{i\alpha}\in\mathcal{O}(U_\alpha)$. Notice that $g_{i\alpha}$
    is unique up to elements of $\mathcal{O}^*(U_\alpha)$. Define
    \begin{equation*}
        f_\alpha = \prod_i g_{i\alpha}^{a_i},
    \end{equation*}
    which is a locally finite product in $\mathcal{M}^*(U_\alpha)$. Now $g_{i\alpha}$ and $g_{i\beta}$ define
    the same hypersurface on $U_{\alpha\beta}$, so they can differ only by an element of $\mathcal{O}^*(U_{\alpha\beta})$.
    The data of the $\{f_\alpha\}$ is thus exactly the data of an element $\mu\in\HH^0(X, \mathcal{M}^*/\mathcal{O}^*)$.
\end{proof}

\begin{exercise}
    Check that the maps defined in the proof above are indeed inverse to each other.\footnote{This may require
    worrying about choices of open covers when computing \v Cech cohomology for the sheaf cohomology.}
\end{exercise}

It is worth noting that in algebraic geometry, cohomology classes on the left are known as {\color{blue}Cartier divisors}
while elements of $\Div X$ are known as {\color{blue}Weil divisors}. The distinction is necessary when working with
schemes in general.

Now let us focus on the case of divisors on a compact Riemann surface $X$. Our goal is to sketch a proof of 
Riemann-Roch.

\begin{definition}
    In this case any divisor $D$ is of the form $D=\sum_{i=1}^m a_i[p_i]$ for $p_i\in X$ and we can define
    the {\color{blue} degree} $\deg: \Div X \to\Z$ to be $\sum_i a_i$. Next, define the sheaf $\mathcal{O}_D$
    associated to a divisor $D$ by
    \begin{equation*}
        \mathcal{O}_D(U) = \{ f\in\mathcal{M}(U) \mid \ord_x f\geqslant -D(x), \forall x\in U\}.
    \end{equation*}
\end{definition}

Roughly speaking, $\mathcal{O}(D)$ is the sheaf of meromorphic functions with poles and zeroes no worse than dictated
by $D$. One obvious question -- and the one that Riemann-Roch addresses -- is how many global sections does this
sheaf have?

\begin{theorem}[Riemann-Roch]
    Let $X$ be a compact Riemann surface of genus $g$ and $D\in\Div X$. Then $\HH^0(X,\mathcal{O}_D)$ and
    $\HH^1(X,\mathcal{O}_D)$ are finite dimensional vector spaces and
    \begin{equation*}
        h^0(X,\mathcal{O}_D) = h^1(X, \mathcal{O}_D) + \deg D + 1 - g.
    \end{equation*}
\end{theorem}

\begin{proof}[Proof sketch]
    Consider first the case $D=0$. Here the sheaf $\mathcal{O}_D$ is the sheaf $\mathcal{O}$ of holomorphic functions.
    Since the only holomorphic functions on all of $X$ are constants, the left hand side is 1 and it
    remains to show that $h^1(X,\mathcal{O}) = g$.  From basic topology we know that $\HH^1(X, \C)=\C^{2g}$.
    The Hodge theorem then allows us to decompose $\HH^1(X, \C)\cong \HH_{\bar\partial}^{1,0}(X,\C)\oplus\HH_{\bar\partial}^{0,1}(X,\C)$,
    and using Serre duality we see that each of the summands is isomorphic to $\C^g$. Finally, the Dolbeault
    theorem yields an isomorphism $\HH_{\bar\partial}^{0,1}(X,\C)\cong \HH^1(X,\mathcal{O})$, as desired.

    Now suppose $D$ is any divisor and $p\in X$ is a point. Write $E= D + [p]$ and suppose the result holds
    for one of the divisors $D$ or $E$. We show that then the result holds for the other.
    Given $D, E$, we obtain a short exact sequence
    \begin{equation*}
        \begin{tikzcd}
            0\rar & \mathcal{O}_D\rar & \mathcal{O}_E\rar{\psi} & \C_p\rar & 0,
        \end{tikzcd}
    \end{equation*}
    where the map $\psi$ to the skyscraper sheaf $\C_p$ is defined as follows. For $U$ an open set, $\psi$ is
    the zero map if $p\notin U$, otherwise for $f\in\mathcal{O}_E(U)$ we can write, near $p$,
    \begin{equation*}
        f(z) = \sum_{n=-D(p)-1}^\infty c_n z^n,
    \end{equation*}
    and send $\psi(U)(f)=c_{-D(p)-1}\in\C=\C_p(U)$.
    Exactness at $\mathcal{O}_D$ is clear since $D\leqslant E=D+[p]$, and exactness at $\mathcal{O}_E$ and
    $\C_p$ are straightforward.
    We now obtain a long exact sequence on cohomology:
    \begin{equation*}
        \begin{tikzcd}
            0\rar & \HH^0(X, \mathcal{O}_D)\rar & \HH^0(X, \mathcal{O}_E)\rar & \C \\
            \;\rar & \HH^1(X,\mathcal{O}_D)\rar & \HH^1(X,\mathcal{O}_E)\rar & 0
        \end{tikzcd}
    \end{equation*}
    where we have computed the cohomology of the skyscraper sheaf. Writing
    \begin{align*}
        V &= \im(\HH^0(X,\mathcal{O}_E)\to \C)\\
        W &= \C/V,
    \end{align*}
    the above exact sequence splits into two short exact sequences
    \begin{equation*}
        \begin{tikzcd}
            0\rar & \HH^0(X, \mathcal{O}_D)\rar & \HH^0(X, \mathcal{O}_E)\rar & V \\
            0\rar & W\rar & \HH^1(X, \mathcal{O}_D)\rar & \HH^1(X,\mathcal{O}_E)
        \end{tikzcd}
    \end{equation*}
    from which it follows that
    \begin{align*}
        h^0(X,\mathcal{O}_E) &= h^0(X,\mathcal{O}_D)+\dim V\\
        h^1(X,\mathcal{O}_D) &= h^1(X,\mathcal{O}_E)+\dim W.
    \end{align*}
    Notice now that we obtain finite-dimensionality inductively, as we know that the cohomologies
    of $\mathcal{O}$ are finite dimensional.
    Summing and noting that $\dim V+\dim W=1=\deg E-\deg D$, we obtain
    \begin{equation*}
        h^0(X,\mathcal{O}_E)-h^0(X,\mathcal{O}_E)-\deg E = h^0(X,\mathcal{O}_D)-h^1(X,\mathcal{O}_D)-\deg D,
    \end{equation*}
    which yields the result.

    An arbitrary divisor may be written as a sum of the zero divisor and finitely many points.
    Hence we find by induction that Riemann-Roch holds for all divisors.
\end{proof}

\begin{exercise}
    Verify that $\HH^0(X, \C_p)\cong\C$ and $\HH^1(X, \C_p)=0$. Hint: refine the cover so that only
    one open contains $p$.
\end{exercise}

In the next talk Pyongwon will speak in more depth about these sheaves $\mathcal{O}_D$ and will
relate them to holomorphic line bundles.

\end{document}
