\documentclass{amsart}

\usepackage[colorlinks=true]{hyperref}
\usepackage{enumerate}
\usepackage{color}
\usepackage{mathrsfs}
\usepackage{tikz-cd}
\usepackage{amssymb}

\theoremstyle{plain}
\newtheorem{theorem}{Theorem}
\newtheorem{lemma}[theorem]{Lemma}
\newtheorem{proposition}[theorem]{Proposition}
\newtheorem{corollary}[theorem]{Corollary}

\theoremstyle{definition}
\newtheorem{definition}[theorem]{Definition}
\newtheorem{example}[theorem]{Example}
\newtheorem{exercise}[theorem]{Exercise}

\theoremstyle{remark}
\newtheorem{remark}[theorem]{Remark}

% Fonts
\newcommand{\A}{\mathbb{A}}
\newcommand{\C}{\mathbb{C}}
\newcommand{\R}{\mathbb{R}}
\newcommand{\Q}{\mathbb{Q}}
\newcommand{\Z}{\mathbb{Z}}
\newcommand{\N}{\mathbb{N}}
\newcommand{\G}{\mathbb{G}}
\newcommand{\fr}{\mathfrak}

% Topology/geometry

\DeclareMathOperator{\Gr}{Gr}
\DeclareMathOperator{\Fl}{Fl}
\DeclareMathOperator{\PP}{\mathbb{P}}
\DeclareMathOperator{\Der}{Der}
\DeclareMathOperator{\Lie}{Lie}
\DeclareMathOperator{\SL}{SL}
\DeclareMathOperator{\GL}{GL}
\DeclareMathOperator{\HH}{H}

% Representation theory

\DeclareMathOperator{\Ad}{Ad}

% Algebra

\DeclareMathOperator{\End}{End}
\DeclareMathOperator{\Aut}{Aut}
\DeclareMathOperator{\Hom}{Hom}
\DeclareMathOperator{\sHom}{\mathscr{H}\!om}
\DeclareMathOperator{\sEnd}{\mathscr{E}\!nd}
\DeclareMathOperator{\id}{id}
\DeclareMathOperator{\irr}{irr}
\DeclareMathOperator{\Diff}{Diff}
\DeclareMathOperator{\gr}{gr}
\DeclareMathOperator{\im}{im}
\DeclareMathOperator{\ad}{ad}
\DeclareMathOperator{\rk}{rk}
\DeclareMathOperator{\Spec}{Spec}
\DeclareMathOperator{\Specm}{Specm}
\DeclareMathOperator{\Stab}{Stab}
\DeclareMathOperator{\Sym}{Sym}

% Category theory

\DeclareMathOperator*{\colim}{colim}



\DeclareMathOperator{\CS}{CS}

\title{Notes on ``Geometric quantization of Cherns-Simons gauge theory''}
\author{Nilay Kumar}
\date{Winter 2016}

\begin{document}

\maketitle
\tableofcontents

Outline:
\begin{enumerate}
    \item Describe Witten's claim that RT invariants come from Wilson loop observables
    \item Briefly mention how GQ applies in this situation, outline GQ (wishlist), review symplectic geometry
    \item Prequantization and why it is too big (take the example of $\R^{2n}$)
    \item Polarizations in general, focus on Kahler polarization
\end{enumerate}

\section{Introduction}
Let $\Sigma$ be a compact oriented 2-manifold and write $M=\Sigma\times\R$. We will think of $\Sigma$ as the
spacelike directions of spacetime and $\R$ the timelike direction. Chern-Simons theory with gauge
group $G$ (that we will take to be compact, connected, and simply-connected) on $M$ is the data of a principal
$G$-bundle $\pi:P\to\Sigma$ together with a
\textit{Lagrangian density} $\mathscr{L}:\mathscr{A}\to C^\infty(\Sigma)$ on the space of connections $\mathscr{A}$ on $P$
given by
\begin{equation*}
    \mathscr{L}_\text{CS}(A) = \langle A\wedge F\rangle + \frac{2}{3}\langle A\wedge [A\wedge A]\rangle.
\end{equation*}
Let us detail the notation used here. Recall first that a connection $A\in\mathscr{A}$ is
a $G$-invariant $\fr g$-valued one-form, i.e. $A\in C^\infty(\underline{\fr g}\otimes T^*P)$ such that
$R_g^*A = \Ad_{g^{-1}}A$, satisfying the additional condition that if $\xi\in\fr g$ then
$A(\xi_P)=\xi$ if $\xi_P$ is the vector field associated to $\xi$. Notice that $\mathscr{A}$,
though not a vector space, is an affine space modelled on $C^\infty(P\times_G\fr g)$. {\color{red}understand this!}
The curvature $F$ of a connection $A$ is is the $\fr g$-valued two-form given by $F(v,w) = dA(v_h,w_h)$,
where $\bullet_h$ denotes projection onto the horizontal distribution $\ker\pi_*$.\footnote{Here
$[-\wedge-]$ and $d$ are given by a combination of the wedge product and commutator.}
Finally, by $\langle-,-\rangle$ we denote an $\ad$-invariant inner product on $\fr g$.

The Chern-Simons action is now given
\begin{equation*}
    S_\text{SC}(A) = \int_M \mathscr{L}_\text{SC}(A)
\end{equation*}
and the quantities of interest are expectation values of observables $\mathscr{O}:\mathscr{A}\to\R$
\begin{equation*}
    \langle\mathscr{O}\rangle = \int_{\mathscr{A}/\mathscr{G}} \mathscr{O}(A) e^{iS_\text{SC}(A)/\hbar}.
\end{equation*}
Here $\mathscr{G}$ is the group of automorphisms of $P\to\Sigma$, which acts by pullback on
$\mathscr{A}$ -- the physical states are unaffected by these gauge transformations, so we integrate
over the quotient $\mathscr{A}/\mathscr{G}$ to eliminate the redundancy.
In {\color{red}??} Witten consider the following observable: let $C$ be a closed oriented curve
in $M$ and $V$ be an irreducible representation of $G$. Then we define the \textit{Wilson line}
\begin{equation*}
    W_{C,V}(A) = \tr_V \exp\int_C A.
\end{equation*}
Witten computed the expectation values
\begin{equation*}
    \langle \prod_i W_{C_i,V_i}\rangle
\end{equation*}
for $(C_i,V_i)$ pairs of curves and $G$-irreps, and recovered in the case of $M=S^3$ the Jones
polynomial and its generalizations. Moreover, taking $M$ to be an arbitrary 3-manifold and taking
no curves, we obtain invariants of 3-manifolds that are effectively computable.

The goal of these notes is to outline a rigorous procedure for obtaining quantum Chern-Simons theory
without resorting to the path integral formalism. To do this, we will use \textit{geometric quantization},
a method of quantizing a symplectic manifold to obtain a Hilbert space. We will follow the constructions
of {\color{red}??}.

\section{Classical versus quantum}
Recall that the data of a classical mechanical system can be encoded as symplectic geometry.
A symplectic form on a manifold $M$ is a nondegenerate closed two-form $\omega\in\Omega^2(M)$. 
By nondegenerate we simply mean that $\omega_p$ is a nondegenerate skew-symmetric bilinear form
for each $p\in M$, or more globally, that $\omega$ induces an isomorphism $TM\to T^*M$.
A symplectic manifold is then a pair $(M,\omega)$, and it is not hard to see that $\dim M$ must be even.
Let us fix some notation: by $X_f$ we mean the unique vector field corresponding to the one-form $df$:
\begin{equation*}
    \iota_{X_f}\omega = df.
\end{equation*}
Then we have a Poisson bracket $\{-,-\}$ on $C^\infty(M)$ given by
\begin{equation*}
    \{f, g\} = \omega(X_f, X_g),
\end{equation*}
under which $C^\infty(M)$ forms a Lie algebra. Notice that the bracket is also a biderivation. We say that
$C^\infty(M)$ forms a Poisson algebra (over $\R$).

Consider, for concreteness, a free particle in $\R^n$. The associated symplectic manifold
$(\R^{2n}, \sum dq^i\wedge dp^i)$ represents the phase space of the system -- all posible states $(q,p)$
of the particle. The observables in this formulation are simply smooth functions on $M$. The energy, for
instance, is given $H(q,p)=|p|^2/2m$. {\color{red}time-evolution and the Poisson bracket?}

In quantum mechanics, on the other hand, the phase space is given as a (complex) Hilbert space $\mathcal{H}$
(or more precisely the projectivized space $\PP \mathcal{H}$) and observables correspond to selfadjoint
operators. In particular, one computes the expectation value of a given observable as
\begin{equation*}
    \langle\mathcal{O}\rangle = \int_\mathcal{H} \mathcal{O}(\psi) e^{iS(\psi)/\hbar},
\end{equation*}
where $S:\mathcal{H}\to\R$ is the action of the system. {\color{red}time-evolution?}

Notice that there is a canonical procedure for obtaining a classical system from a quantum one:
take $\hbar\to0$. As $\hbar$ becomes small the exponential in the integral above oscillates wildly
and the integral is dominated by contributions from the classical locus $\delta S=0$.

The problem of quantization, then, is the converse question: does a classical system determine
a quantum system? This is an interesting question to ask because often in physics one starts
with a classical theory such as electromagnetism (a classical field theory) and wishes to obtain
a quantum theory such as quantum electrodynamics (a quantum field theory). Unfortunately, as the saying
goes, ``quantization is an art, not a functor.'' But let us be more precise and describe exactly
what we mean by quantization (at least for our purposes).
\begin{definition}[Dirac]
    Let $(M,\omega)$ be a symplectic manifold. A \textit{quantization} of $M$ is a complex Hilbert space
    $(\mathcal{H},\mathcal{O})$ with selfadjoint operators $\mathcal{O}$, together with an $\R$-linear map
    $\hat\bullet:C^\infty(M)\to\mathcal{O}$ such that $\hat1$ is the identity operator on $\mathcal{H}$ and 
    \begin{equation*}
        [\hat f,\hat g] = -i\hbar\widehat{\{f,g\}}.
    \end{equation*}
\end{definition}

Unfortunately, it is unclear how to do this in general without making $\mathcal{H}$ unphysically
large.\footnote{In fact, there are various no-go theorems in the literature, c.f. Gronewald-van Hove.}

Before we begin discussing the procedure of geometric quantization, which will approximate the
notion of quantization above, let us see how it applies to the case of Chern-Simons theory.
Recall that the phase space of Chern-Simons theory is the space $\mathscr{A}^\flat/\mathscr{G}$
of flat connections on $\Sigma$ up to gauge transformation. There is a natural symplectic structure on
$\mathscr{A}^\flat$ inherited from the symplectic structure of $\mathscr{A}$ via Marsden-Weinstein reduction.
Since $\mathscr{A}$ is an affine space modelled on the vector space $\Gamma(\Sigma,P\times_G\fr g)$, the
tangent space to $\mathscr{A}$ at any connection is said vector space. {\color{red} Why?}
There is thus a natural symplectic form on $\mathscr{A}$ given
\begin{equation*}
    \omega_\mathscr{A}(\alpha,\beta) = \int_\Sigma \langle \alpha\wedge\beta\rangle.
\end{equation*}

In order to describe how this symplectic form descends to $\mathscr{A}^\flat$, we recall some details
of symplectic reduction. Let $G$ be a Hamiltonian group action on $(M,\omega)$. That is, the $G$
action satisfies:
\begin{enumerate}
    \item $G$ acts through symplectomorphisms;
    \item if $\xi\in\fr g$, the one form associated to the vector field $\xi_M$ is exact:
        \begin{equation*}
            \iota_{\xi_M}\omega = d\kappa(\xi),
        \end{equation*}
        for $\kappa(\xi)\in C^\infty(M)$;
    \item the associated comoment map $\kappa:\fr g\to C^\infty(M)$ is a Lie algebra homomorphism.
\end{enumerate}
Then there exists a $G$-equivariant moment map $\mu:M\to\fr g^*$ determined by
\begin{equation*}
    \kappa(\xi)(p) = \mu(p)(\xi).
\end{equation*}
The fundamental result of symplectic reduction is that if a Hamiltonian action of $G$ is free and proper
then $\mu^{-1}(0)/G_0$ is a symplectic manifold with symplectic form $\omega_0$ uniquely
characterized by $\pi_0^*\omega_0 = \iota_0^*\omega$.

Let us return to Chern-Simons theory. We claim for now that the action of $\mathscr{G}$ is Hamiltonian,
free, proper, and moreover that the curvature $F:\mathscr{A}\to\fr g^*$ provides a moment map for this action.
{\color{red} Prove this!} Then the theory of reduction yields a symplectic structure on
the moduli of flat connections $F^{-1}(0)/\mathscr{G}=\mathscr{A}^\flat/\mathscr{G}$. Moreover,
one can check that this symplectic form is in fact integral. Thus the story of geometric quantization is
indeed applicable.

{\color{red} What are some other ways of obtaining this symplectic structure? C.f. Goldman, Karshon, Weinstein,
Guillou, Huebschmann.}

\section{Geometric quantization}

The first step in the geometric quantization of a symplectic manifold $(M,\omega)$ is \textit{prequantization},
which assigns to $M$ a line bundle with connection whose curvature is $\omega$. The prequantum Hilbert
space is then taken to be the square-integrable sections of this line bundle.

There is, of course, an obvious line bundle: the trivial one. Can we get away with this?
Consider the space $L^2(M)$ of
square-integrable smooth complex functions on $M$. This space has a natural inner product given by
\begin{equation*}
    \langle\psi,\psi'\rangle = \int_M \bar\psi \psi' \varepsilon,
\end{equation*}
where $\varepsilon=(\omega/2\pi\hbar)^n$ is a volume form. There is an obvious quantization of
$f\in C^\infty(M)$ to
\begin{equation*}
    \psi \mapsto -i\hbar X_f\psi.
\end{equation*}
Unfortunately this will send constants to the zero operator. There is an immediate correction:
\begin{equation*}
    \psi \mapsto (-i\hbar X_f + f)\psi.
\end{equation*}
This quantization is no longer a Lie algebra homomorphism, so we add yet another term
\begin{equation*}
    \psi \mapsto (-i\hbar X_f + f -i\iota_{X_f}\lambda/\hbar)\psi,
\end{equation*}
where $\lambda$ is a one-form such that $d\lambda=\omega$. This prescription works, but now
depends on $\lambda$, which need not exist in general. The way out is to replace the
trivial bundle with a Hermitian bundle together with a connection whose curvature is $\omega$.

\begin{theorem}
    There exists a Hermitian line bundle $L\to M$ and a connection $\nabla$ on $L$ with
    curvature $\hbar^{-1}\omega$ if and only if $(2\pi\hbar)^{-1}\omega\in \HH^2(M,\Z)$.
    In this case, the choice of $(L,\nabla)$ is parameterized by $\HH^1(M,U(1))$.
\end{theorem}


\end{document}
