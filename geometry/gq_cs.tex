\documentclass{amsart}

\usepackage[colorlinks=true]{hyperref}
\usepackage{enumerate}
\usepackage{color}
\usepackage{mathrsfs}
\usepackage{tikz-cd}
\usepackage{amssymb}

\usepackage[backend=bibtex, style=alphabetic]{biblatex}
\addbibresource{gq_cs_ref.bib}

\theoremstyle{plain}
\newtheorem{theorem}{Theorem}
\newtheorem{lemma}[theorem]{Lemma}
\newtheorem{proposition}[theorem]{Proposition}
\newtheorem{corollary}[theorem]{Corollary}

\theoremstyle{definition}
\newtheorem{definition}[theorem]{Definition}
\newtheorem{example}[theorem]{Example}
\newtheorem{exercise}[theorem]{Exercise}

\theoremstyle{remark}
\newtheorem{remark}[theorem]{Remark}

% Fonts
\newcommand{\A}{\mathbb{A}}
\newcommand{\C}{\mathbb{C}}
\newcommand{\R}{\mathbb{R}}
\newcommand{\Q}{\mathbb{Q}}
\newcommand{\Z}{\mathbb{Z}}
\newcommand{\N}{\mathbb{N}}
\newcommand{\G}{\mathbb{G}}
\newcommand{\fr}{\mathfrak}

% Topology/geometry

\DeclareMathOperator{\Gr}{Gr}
\DeclareMathOperator{\Fl}{Fl}
\DeclareMathOperator{\PP}{\mathbb{P}}
\DeclareMathOperator{\Der}{Der}
\DeclareMathOperator{\Lie}{Lie}
\DeclareMathOperator{\SL}{SL}
\DeclareMathOperator{\GL}{GL}
\DeclareMathOperator{\SO}{SO}
\DeclareMathOperator{\Sp}{Sp}
\DeclareMathOperator{\HH}{H}
\DeclareMathOperator{\Symp}{Symp}

% Representation theory

\DeclareMathOperator{\Ad}{Ad}

% Algebra

\DeclareMathOperator{\End}{End}
\DeclareMathOperator{\Aut}{Aut}
\DeclareMathOperator{\Hom}{Hom}
\DeclareMathOperator{\sHom}{\mathscr{H}\!om}
\DeclareMathOperator{\sEnd}{\mathscr{E}\!nd}
\DeclareMathOperator{\id}{id}
\DeclareMathOperator{\irr}{irr}
\DeclareMathOperator{\Diff}{Diff}
\DeclareMathOperator{\gr}{gr}
\DeclareMathOperator{\im}{im}
\DeclareMathOperator{\ad}{ad}
\DeclareMathOperator{\rk}{rk}
\DeclareMathOperator{\Spec}{Spec}
\DeclareMathOperator{\Specm}{Specm}
\DeclareMathOperator{\Stab}{Stab}
\DeclareMathOperator{\Sym}{Sym}

% Category theory

\DeclareMathOperator*{\colim}{colim}

\makeatletter
\renewcommand\d[1]{\mspace{6mu}\mathrm{d}#1\@ifnextchar\d{\mspace{-3mu}}{}}
\makeatother



\DeclareMathOperator{\CS}{CS}

\title{Some background on geometric quantization}
\author{Nilay Kumar}
\date{Winter 2016}

\begin{document}

\maketitle
\tableofcontents

\section{Introduction}
Let $M$ be a compact oriented 3-manifold. Chern-Simons theory with gauge
group $G$ (that we will take to be compact, connected, and simply-connected) on $M$ is the data of a principal
$G$-bundle $\pi:P\to M$ together with a
\textit{Lagrangian density} $\mathscr{L}:\mathscr{A}\to \Omega^3(P)$ on the space of connections $\mathscr{A}$ on $P$
given by
\begin{equation*}
    \mathscr{L}_\text{CS}(A) = \langle A\wedge F\rangle + \frac{2}{3}\langle A\wedge [A\wedge A]\rangle.
\end{equation*}
Let us detail the notation used here. Recall first that a connection $A\in\mathscr{A}$ is
a $G$-invariant $\fr g$-valued one-form, i.e. $A\in \Omega^1(P;\fr g)$ such that
$R_g^*A = \Ad_{g^{-1}}A$, satisfying the additional condition that if $\xi\in\fr g$ then
$A(\xi_P)=\xi$ if $\xi_P$ is the vector field associated to $\xi$. Notice that $\mathscr{A}$,
though not a vector space, is an affine space modelled on $\Omega^1(M,P\times_G\fr g)$.
The curvature $F$ of a connection $A$ is is the $\fr g$-valued two-form given by $F(v,w) = dA(v_h,w_h)$,
where $\bullet_h$ denotes projection onto the horizontal distribution $\ker\pi_*$.\footnote{Here
$[-\wedge-]$ and $d$ are given by a combination of the wedge product and commutator.}
Finally, by $\langle-,-\rangle$ we denote an $\ad$-invariant inner product on $\fr g$.

The Chern-Simons action is now given
\begin{equation*}
    S_\text{SC}(A) = \int_M \mathscr{L}_\text{SC}(A)
\end{equation*}
and the quantities of interest are expectation values of observables $\mathscr{O}:\mathscr{A}\to\R$
\begin{equation*}
    \langle\mathscr{O}\rangle = \int_{\mathscr{A}/\mathscr{G}} \mathscr{O}(A) e^{iS_\text{SC}(A)/\hbar}.
\end{equation*}
Here $\mathscr{G}$ is the group of automorphisms of $P\to\Sigma$, which acts by pullback on
$\mathscr{A}$ -- the physical states are unaffected by these gauge transformations, so we integrate
over the quotient $\mathscr{A}/\mathscr{G}$ to eliminate the redundancy.
In \cite{witten} Witten consider the following observable: let $C$ be a closed oriented curve
in $M$ and $V$ be an irreducible representation of $G$. Then we define the \textit{Wilson line}
\begin{equation*}
    W_{C,V}(A) = \tr_V \exp\int_C A.
\end{equation*}
Witten computed the expectation values
\begin{equation*}
    \langle \prod_i W_{C_i,V_i}\rangle
\end{equation*}
for $(C_i,V_i)$ pairs of curves and $G$-irreps, and recovered in the case of $M=S^3$ the Jones
polynomial and its generalizations. Moreover, taking $M$ to be an arbitrary 3-manifold and taking
no curves, we obtain invariants of 3-manifolds that are effectively computable.

The goal of these notes is to provide some background on geometric quantization which Pyongwon
will use to tell us about a rigorous procedure for obtaining quantum Chern-Simons theory
on a compact oriented 2-manifold $\Sigma$
without resorting to the path integral formalism. To do this, we will use \textit{geometric quantization},
a method of quantizing a symplectic manifold to obtain a Hilbert space. We will follow the constructions
of \cite{gq}.

\section{Classical versus quantum}
Recall that the data of a classical mechanical system can be encoded as symplectic geometry.
A symplectic form on a manifold $M$ is a nondegenerate closed two-form $\omega\in\Omega^2(M)$. 
By nondegenerate we simply mean that $\omega_p$ is a nondegenerate skew-symmetric bilinear form
for each $p\in M$, or more globally, that $\omega$ induces an isomorphism $TM\to T^*M$.
A symplectic manifold is then a pair $(M,\omega)$, and it is not hard to see that $\dim M$ must be even.
Let us fix some notation: by $X_f$ we mean the unique vector field corresponding to the one-form $df$:
\begin{equation*}
    \iota_{X_f}\omega = df.
\end{equation*}
Then we have a Poisson bracket $\{-,-\}$ on $C^\infty(M)$ given by
\begin{equation*}
    \{f, g\} = \omega(X_f, X_g),
\end{equation*}
under which $C^\infty(M)$ forms a Lie algebra. Notice that the bracket is also a biderivation. We say that
$C^\infty(M)$ forms a Poisson algebra (over $\R$).

Consider, for concreteness, a free particle in $\R^n$. The associated symplectic manifold
$(\R^{2n}, \sum dq^i\wedge dp^i)$ represents the phase space of the system -- all posible states $(q,p)$
of the particle. The observables in this formulation are simply smooth functions on $M$. The energy, for
instance, is given $H(q,p)=|p|^2/2m$.

In quantum mechanics, on the other hand, the phase space is given as a (complex) Hilbert space $\mathcal{H}$
(or more precisely the projectivized space $\PP \mathcal{H}$) and observables correspond to selfadjoint
operators. In particular, one computes the expectation value of a given observable as
\begin{equation*}
    \langle\mathcal{O}\rangle = \int_\mathcal{H} \mathcal{O}(\psi) e^{iS(\psi)/\hbar},
\end{equation*}
where $S:\mathcal{H}\to\R$ is the action of the system.

Notice that there is a canonical procedure for obtaining a classical system from a quantum one:
take $\hbar\to0$. Very roughly speaking, as $\hbar$ becomes small the exponential in the integral
above oscillates wildly and the integral is dominated by contributions from the classical locus $\delta S=0$.

The problem of quantization, then, is the converse question: does a classical system determine
a quantum system? This is an interesting question to ask because often in physics one starts
with a classical theory such as electromagnetism (a classical field theory) and wishes to obtain
a quantum theory such as quantum electrodynamics (a quantum field theory). Unfortunately, as the saying
goes, ``quantization is an art, not a functor.'' But let us be more precise and describe exactly
what we mean by quantization (at least for our purposes).
\begin{definition}[Dirac]
    Let $(M,\omega)$ be a symplectic manifold. A \textit{quantization} of $M$ is a complex Hilbert space
    $(\mathcal{H},\mathcal{O})$ with selfadjoint operators $\mathcal{O}$, together with an $\R$-linear map
    $\hat\bullet:C^\infty(M)\to\mathcal{O}$ such that $\hat1$ is the identity operator on $\mathcal{H}$ and 
    \begin{equation*}
        [\hat f,\hat g] = -i\hbar\widehat{\{f,g\}}.
    \end{equation*}
\end{definition}

Unfortunately, it is unclear how to do this in general without making $\mathcal{H}$ unphysically
large.\footnote{In fact, there are various no-go theorems in the literature, c.f. Gronewald-van Hove.}

Before we begin discussing the procedure of geometric quantization, which will approximate the
notion of quantization above, let us see how it applies to the case of Chern-Simons theory.
Recall that the phase space of Chern-Simons theory is the space $\mathscr{A}^\flat/\mathscr{G}$
of flat connections on $\Sigma$ up to gauge transformation. There is a natural symplectic structure on
$\mathscr{A}^\flat$ inherited from the symplectic structure of $\mathscr{A}$ via Marsden-Weinstein reduction.
Since $\mathscr{A}$ is an affine space modelled on the vector space $\Gamma(\Sigma,P\times_G\fr g)$, the
tangent space to $\mathscr{A}$ at any connection is said vector space.
There is thus a natural symplectic form on $\mathscr{A}$ given
\begin{equation*}
    \omega_\mathscr{A}(\alpha,\beta) = \int_\Sigma \langle \alpha\wedge\beta\rangle.
\end{equation*}

In order to describe how this symplectic form descends to $\mathscr{A}^\flat$, we recall some details
of symplectic reduction. Let $G$ be a Hamiltonian group action on $(M,\omega)$. That is, the $G$
action satisfies:
\begin{enumerate}
    \item $G$ acts through symplectomorphisms;
    \item if $\xi\in\fr g$, the one form associated to the vector field $\xi_M$ is exact:
        \begin{equation*}
            \iota_{\xi_M}\omega = d\kappa(\xi),
        \end{equation*}
        for $\kappa(\xi)\in C^\infty(M)$;
    \item the associated comoment map $\kappa:\fr g\to C^\infty(M)$ is a Lie algebra homomorphism.
\end{enumerate}
Then there exists a $G$-equivariant moment map $\mu:M\to\fr g^*$ determined by
\begin{equation*}
    \kappa(\xi)(p) = \mu(p)(\xi).
\end{equation*}
The fundamental result of symplectic reduction is that if a Hamiltonian action of $G$ is free and proper
then $\mu^{-1}(0)/G_0$ is a symplectic manifold with symplectic form $\omega_0$ uniquely
characterized by $\pi_0^*\omega_0 = \iota_0^*\omega$.

Let us return to Chern-Simons theory.
\begin{exercise}
    Make sense of the statement that $F:\mathscr{A}\to\Omega^2(\Sigma, P\times_G\fr g)$ provides
    a moment map for the $\mathscr{G}$-action on $\mathscr{A}$.
\end{exercise}
In view of the exercise above the theory of reduction yields a symplectic structure on
the moduli of flat connections $F^{-1}(0)/\mathscr{G}=\mathscr{A}^\flat/\mathscr{G}$.
Thus the story of geometric quantization is indeed applicable.\footnote{Exercise: why is the form integral?}

\section{Geometric quantization}

In this section we follow \cite{wood} and \cite{hall}.

The first step in the geometric quantization of a symplectic manifold $(M,\omega)$ is \textit{prequantization},
which assigns to $M$ a line bundle with connection whose curvature is $\omega$. The prequantum Hilbert
space is then taken to be the square-integrable sections of this line bundle.

There is, of course, an obvious line bundle: the trivial one. Can we get away with this?
Consider the space $L^2(M)$ of
square-integrable smooth complex functions on $M$. This space has a natural inner product given by
\begin{equation*}
    \langle\psi,\psi'\rangle = \int_M \bar\psi \psi' \varepsilon,
\end{equation*}
where $\varepsilon=(\omega/2\pi\hbar)^n$ is a volume form. There is an obvious quantization of
$f\in C^\infty(M)$ to
\begin{equation*}
    \psi \mapsto -i\hbar X_f\psi.
\end{equation*}
Unfortunately this will send constants to the zero operator. There is an immediate correction:
\begin{equation*}
    \psi \mapsto (-i\hbar X_f + f)\psi.
\end{equation*}
This quantization is no longer a Lie algebra homomorphism, so we add yet another term
\begin{equation*}
    \psi \mapsto (-i\hbar X_f + f -i\iota_{X_f}\lambda/\hbar)\psi,
\end{equation*}
where $\lambda$ is a one-form such that $d\lambda=\omega$. This prescription works, but now
depends on $\lambda$, which need not exist in general. The way out is to replace the
trivial bundle with a Hermitian bundle together with a connection whose curvature is $\omega$.
The key fact is as follows.

\begin{theorem}
    There exists a Hermitian line bundle $L\to M$ and a Hermitian connection $\nabla:\HH^0(M,L)\to\HH^0(M, L\otimes T^*M)$ on $L$ with
    curvature $\hbar^{-1}\omega$ if and only if $(2\pi\hbar)^{-1}\omega\in \HH^2(M,\Z)$.
    In this case, the choice of $(L,\nabla)$ is parameterized by $\HH^1(M,U(1))$.
\end{theorem}

Using this, we can define prequantization.
\begin{definition}
    Let $(M,\omega)$ be a integral symplectic manifold. Then a \textit{prequantum bundle} is
    a choice of line bundle $L\to M$ and connection $\nabla$
    on $L$ with curvature $\hbar^{-1}\omega$.
    The \textit{prequantum Hilbert space} is the space $\mathcal{H}^\text{pre}$ of square-integrable
    sections of $L$ together with the obvious inner product. The operator associated to $f\in C^\infty(M)$
    is given by
    \begin{equation*}
        \hat f\psi = (-i\hbar\nabla_{X_f} + f)\psi.
    \end{equation*}
\end{definition}

\begin{example}
    Consider $M=\R^{2n}$ with coordinates $(q,p)$ and its usual symplectic form $\omega=dq^i\wedge dp^i$.
    The integrality of $\omega$ is clear because $M$ is exact, $\omega=d(p_idq^i)=d\lambda$. In this case
    the trivial line bundle with connection $\nabla_{v}=v+\iota_{v}\lambda$ provides a prequantum
    line bundle. Notice that $X_{q^i}=\partial/\partial p^i$ and $X_{p_i}=-\partial/\partial q^i$.
    Hence we find
    \begin{align*}
        \hat q^i\psi &= \left(q^i -i\hbar \frac{\partial}{\partial p_i}\right)\psi\\
        \hat p_i\psi &= i\hbar\frac{\partial}{\partial q^i}\psi.
    \end{align*}
    A straightforward computation reveals that $[\hat q^i,\hat p_i]=-i\hbar=-i\hbar\widehat{\{q^i,p_i\}}$,
    as desired. On the other hand, there is something strange going on: our ``wavefunctions'' depend
    on both $p$ and $q$, which is why $\hat q^i$ has an unfamiliar $\partial/\partial p_i$ term. 
    Usually in quantum mechanics, we work with wavefunctions depending only on the $q^i$ or only on
    the $p_i$, with the two perspectives related via the Fourier transform.
\end{example}

The previous example shows that, even in the case of a particle in $\R^{2n}$, the prequantum
Hilbert space constructed by prequantization is morally twice as large as it should be.
The next step of geometric quantization, polarization, restricts the space of functions on $M$
that we quantize. On $T^*M$, for example, one has the vertical and horizontal polarizations,
which yield the usual position and momentum Hilbert spaces. Polarization is a rather delicate
and involved procedure, so we only sketch the beginning of the story.

\begin{definition}
    A \textit{polarization} on $(M,\omega)$ is an integrable Lagrangian subbundle $\mathscr{P}\hookrightarrow TM\otimes\C$.
    Then, if $L\to M$ is a prequantum line bundle and $\mathscr{P}$ is a polarization, we define the
    \textit{quantum Hilbert space} $\mathcal{H}$ to be the set of square integrable sections
    that are covariantly constant along $\mathscr{P}$, i.e.
    \begin{equation*}
        \mathcal{H} = \{s\in L^2(M,L) \mid \nabla_Xs=0, X\in\mathscr{P}\}.
    \end{equation*}
\end{definition}

\begin{example}
    Consider $M=T^*\R=\R^2$ with coordinates $(q,p)$ and the trivial prequantum line bundle
    $(L=M\times\C, \nabla_v=v+i\iota_v\lambda)$, where $\lambda=pdq$.
    There is an obvious polarization of $\R^2$ by contangent fibers so that the polarized sections
    satisfy $\nabla_{\partial/\partial p} \psi = 0$, i.e.
    \begin{equation*}
        \left( \frac{\partial}{\partial p}+i\iota_{\partial/\partial p}pdq \right)\psi=\frac{\partial\psi}{\partial p}=0.
    \end{equation*}
    Hence $\mathcal{H}$ consists of square-integrable sections depending on $q$. Unfortunately
    there are no such sections as
    \begin{equation*}
        \int_{\R^2}\overline{\psi(x)}\psi(x)\d x\d p=\infty
    \end{equation*}
    unless $\psi=0$.

    Alternatively one could choose another polarization, taking the zero section and its translates,
    upon which the polarized sections satisfy $\nabla_{\partial/\partial q}\psi =0$, which one
    can solve to find that $\psi(q,p)=\psi(p,0)e^{-iqp}$. Again, these will not be square-integrable.

    To actually obtain square-integrable sections we would, roughly speaking, have to cut down
    our base manifold from $T^*\R$ to $\R$. Hence even after polarization we do not necessarily
    obtain the Hilbert spaces familiar to physicists.
\end{example}

\begin{remark}
    Polarizations need not exist! There is a class of compact 4-dimensional symplectic manifolds
    admitting no polarizations at all, c.f. \cite{gotay}.
\end{remark}

Recall that a K\"ahler manifold $(M,J,\omega)$ is a symplectic manifold together
with an integrable almost complex structure $J:TM\to TM$ (with $J^2=-\id$) compatible with
the symplectic structure: $\omega(Jv,Jw)=\omega(v,w)$.
The complexified tangent bundle
now splits $TM\otimes\C=T^{1,0}M\oplus T^{0,1}M$ along the eigenvalues $\pm i$ of $J\otimes\C$.
The canonical \textit{K\"ahler polarization} on $(M,J,\omega)$ is simply the subbundle
$T^{1,0}M\subset TM\otimes\C$, i.e. the holomorphic tangent bundle.

Now, prequantization of $(M,J,\omega)$ yields a hermitian line bundle $L\to M$ equipped
with a hermitian connection $\nabla$ with curvature given by the K\"ahler form $\omega$.
The connection extends linearly to $\nabla^\C:\HH^0(M,L)\to\HH^0(M,L\otimes T^*M\otimes\C)$,
and we can define the \textit{K\"ahler polarized sections} of $L$ to be $s\in\HH^0(M,L)$ such that
$\nabla_Xs=0$ for all $X\in\HH^0(M,T^{0,1}M)$, that is, sections that are covariantly holomorphic.
The \textit{quantum Hilbert space} $\mathcal{H}$ is then the space of square-integrable K\"ahler
polarized sections.

\begin{exercise}
    Show that, in the K\"ahler case, $L$ has a natural holomorphic structure, and $\mathcal{H}$
    is simply the space of square-integrable holomorphic sections.
\end{exercise}

There are a number of undesirable features of this story.
While it can be shown that there always exist nonzero local polarized sections, there need not always exist
global square-integrable polarized sections.  Moreover, the question of what the operators on $\mathcal{H}$
are now becomes quite subtle.
For instance, it only makes sense to consider the subset of the prequantization operators
preserving $\mathcal{H}\subset\mathcal{H}^\text{pre}$.

\begin{example}
    Consider $M=\R^2$ with complex coordinates $z=q+ip, \bar z=q-ip$. Recall that the
    prequantum line bundle in this case is the trivial line bundle with connection $\nabla_v = v+\iota_v\lambda$,
    where $\lambda=pdq$. Some computation reveals that the polarized states are of the form
    \begin{equation*}
        \psi(q,p) = F(z)e^{-p^2/2\hbar} = F(z)e^{-\text{Im}(z)^2/\hbar},
    \end{equation*}
    where $F$ is an arbitrary holomorphic function. Those $\psi$ which are square-integrable
    form what is known to functional analysts as the Segal-Bargmann Hilbert space.
\end{example}

\nocite{*}
\printbibliography

\end{document}
