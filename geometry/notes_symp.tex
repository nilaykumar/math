\documentclass{amsart}

\usepackage[colorlinks=true]{hyperref}
\usepackage{enumerate}
\usepackage{tikz-cd}
\usepackage{amssymb}

\theoremstyle{plain}
\newtheorem{theorem}{Theorem}
\newtheorem{lemma}[theorem]{Lemma}
\newtheorem{proposition}[theorem]{Proposition}
\newtheorem{corollary}[theorem]{Corollary}

\theoremstyle{definition}
\newtheorem{definition}[theorem]{Definition}
\newtheorem{example}[theorem]{Example}
\newtheorem{exercise}[theorem]{Exercise}

\theoremstyle{remark}
\newtheorem{remark}[theorem]{Remark}

% Fonts
\newcommand{\A}{\mathbb{A}}
\newcommand{\C}{\mathbb{C}}
\newcommand{\R}{\mathbb{R}}
\newcommand{\Q}{\mathbb{Q}}
\newcommand{\Z}{\mathbb{Z}}
\newcommand{\N}{\mathbb{N}}
\newcommand{\G}{\mathbb{G}}
\newcommand{\fr}{\mathfrak}

% Topology/geometry

\DeclareMathOperator{\Gr}{Gr}
\DeclareMathOperator{\Fl}{Fl}
\DeclareMathOperator{\PP}{\mathbb{P}}
\DeclareMathOperator{\Der}{Der}
\DeclareMathOperator{\Lie}{Lie}
\DeclareMathOperator{\SL}{SL}
\DeclareMathOperator{\GL}{GL}
\DeclareMathOperator{\HH}{H}

% Representation theory

\DeclareMathOperator{\Ad}{Ad}

% Algebra

\DeclareMathOperator{\End}{End}
\DeclareMathOperator{\Aut}{Aut}
\DeclareMathOperator{\Hom}{Hom}
\DeclareMathOperator{\sHom}{\mathscr{H}\!om}
\DeclareMathOperator{\sEnd}{\mathscr{E}\!nd}
\DeclareMathOperator{\id}{id}
\DeclareMathOperator{\irr}{irr}
\DeclareMathOperator{\Diff}{Diff}
\DeclareMathOperator{\gr}{gr}
\DeclareMathOperator{\im}{im}
\DeclareMathOperator{\ad}{ad}
\DeclareMathOperator{\rk}{rk}
\DeclareMathOperator{\Spec}{Spec}
\DeclareMathOperator{\Specm}{Specm}
\DeclareMathOperator{\Stab}{Stab}
\DeclareMathOperator{\Sym}{Sym}

% Category theory

\DeclareMathOperator*{\colim}{colim}



\title{Notes on symplectic geometry}
\author{Nilay Kumar}
\date{Fall 2015}

\begin{document}

\maketitle
\tableofcontents

These notes were written for a reading course with Professor Eric Zaslow on
the basics of symplectic geometry. They follow Mcduff/Salamon quite closely.
These notes are rather rough, and in several places woefully incomplete: \textit{caveat lector}.\footnote{add references!}

\section{Week 1}

\subsection{The cotangent bundle}

\begin{definition}
Let $X$ be a smooth $n$-manifold and $\pi:M=T^*X\to X$ be its cotangent bundle. We define the
\textbf{canonical one-form} $\theta\in\Omega^1(M)$ as follows. For any $p=(x,\xi)\in M$,
set
\begin{equation*}
    \theta_p(v)=\xi( d_x\pi(v)).
\end{equation*}
\end{definition}

The one-form $\theta$ is canonical (or tautological) in the sense that its value at a point
is simply given by the covector determined by that point. More precisely, we have the following
characterization.

\begin{proposition}
    The canonical one-form $\theta$ is the (unique) one-form such that for every $\lambda\in\Omega^1(X)$,
    $\lambda^*\theta=\lambda$.
    \label{prop:can}
\end{proposition}
\begin{proof}
    We compute, for $v\in T_pX$,
    \begin{align*}
        (\lambda^*\theta)_p(v) &= \theta_{\lambda(p)}(d_p\lambda(v)) \\
        &= \lambda_p(d_p(\pi\circ\lambda)(v)) \\
        &= \lambda_p(v),
    \end{align*}
    where we have used the fact that $\lambda$ is a section of $\pi$, i.e. $\pi\circ\lambda=\id_X$.
    Uniqueness is easily checked.
\end{proof}

\begin{definition}
    The \textbf{canonical symplectic form} $\omega\in\Omega^2(M)$ is now defined to be the exterior
    derivative
    \begin{equation*}
        \omega = - d\theta,
    \end{equation*}
    of the canonical one-form. To be symplectic, $\omega$ must be closed and nondegenerate. That it
    is closed is obvious.
\end{definition}

\begin{proposition}
    The form $\omega\in\Omega^2(M)$ is nondegenerate and thus defines a symplectic structure on
    $M=T^*X$.\footnote{Is there a coordinate invariant proof?}
\end{proposition}
\begin{proof}
    For $\omega$ to be non-degenerate, it must be nondegenerate at each point $p\in M$. Given
    coordinates $p=(x,\xi)=(x^1,\ldots,x^n,\xi_1,\ldots,\xi_n)$ in a neighborhood of $p$, we can
    compute
    \begin{align*}
        \theta_{(x,\xi)}\left( v^i\frac{\partial}{\partial x^i} + \nu^i\frac{\partial}{\partial\xi^i} \right)
        &= \xi\left( v^i\frac{\partial}{\partial x^i} \right) \\
        &= \xi_iv^i
    \end{align*}
    and hence
    \begin{equation*}
        \theta = \xi_i dx^i.
    \end{equation*}
    Taking an exterior derivative, we find that
    \begin{align*}
        \omega &= - d\theta \\
        &= dx^i\wedge d\xi_i.
    \end{align*}
    Fix $v\in T_pM$ and suppose that $\iota_v\omega_p=0$, i.e. $\omega_p(v,w)=0$
    for all $w\in T_pM$. In coordinates, this implies that
    \begin{align*}
        \iota_{v^j\frac{\partial}{\partial x^j}+\nu^j\frac{\partial}{\partial\xi^j}}(dx^i\wedge d\xi_i)
        &= v^id\xi_i-\nu^idx^i\\
        &=0,
    \end{align*}
    and hence that $v^i=\nu^i=0$, i.e. $v=0$. We conclude that $\omega_p$ is nondegenerate at each $p\in M$.
\end{proof}

\begin{remark}
    Note that a 2-form $\omega$ on a manifold $M$ is nondegenerate if and only if $\omega^n$ is
    nowhere vanishing. Fix $p\in M$ and consider the vector space $(T_pM, \omega_p)$. If
    $\omega_p$ is nondegenerate, we can find a symplectic basis for $T_pM$, and so $\omega_p^n$
    evaluated on $(u_1,\ldots,u_n,v_1,\ldots ,v_n)$ is nonzero, whence $\omega_p^n$ is not zero
    on $V$. On the other hand, suppose $\omega_p$ is degenerate, i.e. there is a $v\neq0$ such
    that $\omega_p(v,w)=0$ for all $w\in V$. Choosing a basis $v_1,\ldots,v_{2n}$ for $V$ such
    that $v_1=v$, we find that $\omega_p(v_1,\ldots,v_{2n})=0$ and hence $\omega_p=0$ on $V$.

    We conclude that every symplectic manifold is orientable.
    \label{rem:nondeg}
\end{remark}

It is easy to see that $\omega$ provides an isomorphism $\iota:T_xX\xrightarrow{\sim} T_x^*X$
between tangent and cotangent spaces at each point $x\in X$: since $\omega_x$ is nondegenerate, the linear
map $\iota:v\mapsto \omega_x(v,-)$ is injective and hence bijective. In fact, we can say more.

\begin{proposition}
    The metric $\omega$ induces an isomorphism of vector bundles $\iota:TX\xrightarrow{\sim}T^*X=M$.
\end{proposition}
\begin{proof}
    Recall that an isomorphism in the category of smooth vector bundles is a smooth
    bijection\footnote{Existence of a smooth inverse is automatic (reference?).}
    $\iota$ such that the diagram
    \begin{equation*}
        \begin{tikzcd}
            TX\ar{rr}{\iota}\ar[swap]{rd}{\pi_1} & & T^*X\ar{ld}{\pi_2}\\
            & X
        \end{tikzcd}
    \end{equation*}
    commutes and for each $x\in X$, the restriction $\iota_x:T_xX\to T^*_xX$ is linear.
    The map $\iota:TX\to T^*X$ taking $(x,v)\mapsto (x,\omega(v,-))$ fits into the diagram above
    and is bijective and fiberwise linear. Moreover, $\iota$ is a smooth map, as is seen by
    its coordinate description computed above.
\end{proof}

\begin{definition}
    A \textbf{Hamiltonian} is a smooth function $H:M=T^*X\to\R$.
    we define the \textbf{Hamiltonian vector field} $v_H$ associated to $H$ to be the vector
    field on $M$ satisfying
    \begin{equation*}
        \iota_{v_H}\omega = dH.
    \end{equation*}
    The (local) flow $F:(-\varepsilon,\varepsilon)\times M\to M$ determined by $v_H$ is
    called the \textbf{Hamiltonian flow}.\footnote{Is this a global flow? Does it depend on $X$?}
\end{definition}

Note that an integral curve $\gamma_{v_H}:(-\varepsilon,\varepsilon)\to M$ of $v_H$ can be thought
of as the trajectory of a physical state in phase space. Indeed, Hamilton's equations are given
\begin{align*}
    \frac{\partial x^i}{\partial t} &= \frac{\partial H}{\partial \xi_i} \\
    \frac{\partial \xi_i}{\partial t} &= - \frac{\partial H}{\partial x^i},
\end{align*}
which is precisely the condition that $\gamma_{v_H}'(t)=(v_H)_{\gamma(t)}$. Moreover, $H$ is
constant along the Hamiltonian flow, as
\begin{equation*}
    dH(v_H) = (\iota_{v_H}\omega)(v_H)=\omega(v_H,v_H) = 0,
\end{equation*}
i.e. $v_H$ is tangent to the level sets of $H$. In a physical system, where $H$ is the
energy functional on phase space, this phenomenon is the law of conservation of energy.

\begin{proposition}
    The Hamiltonian flow is a symplectomorphism, i.e. $F_t^*\omega=\omega$.\footnote{Is there
    a better proof?}
\end{proposition}
\begin{proof}
    We use the following trick:
    \begin{equation*}
        \int_0^t \frac{d}{dt}F_t^*\omega\; dt = F_t^*\omega-\omega
    \end{equation*}
    since $F_0=\id_M$, and hence $F_t$ is a symplectomorphism if and only if the integrand
    is zero. But
    \begin{align*}
        \frac{d}{dt}F_t^*\omega &= \frac{d}{ds}\bigg|_{s=0}F_{t+s}^*\omega=F_t^*\frac{d}{ds}\bigg|_{s=0}F_s^*\omega \\
        &= F_t^*\mathcal{L}_{v_H}\omega,
    \end{align*}
    and Cartan's magic formula,
    \begin{equation*}
        \mathcal{L}_{v_H}\omega = d\iota_{v_H}\omega + \iota_{v_H}d\omega,
    \end{equation*}
    tells us that $\mathcal{L}_{v_H}\omega=0$ since $\iota_{v_H}\omega=dH$ is closed, as is $\omega$.
\end{proof}

\begin{corollary}[Liouville's Theorem]
    The volume form $\omega^n$ on $M=T^*X$ is preserved by the Hamiltonian flow.
\end{corollary}


\subsection{Geodesic flow as Hamiltonian flow}

We wish to discuss geodesics and geodesic flow. For this, we need the concept
of connections and covariant derivatives.\footnote{Reference do Carmo.}
\begin{definition}
    A \textbf{connection} on a vector bundle $E\to X$ is an $\R$-linear map
    $\nabla:\Gamma(X,E)\to\Gamma(X,E\otimes T^*X)$ such that the Leibniz rule
    \begin{equation*}
        \nabla(f\sigma) = (\nabla \sigma)f + \sigma\otimes df,
    \end{equation*}
    for all $f\in C^\infty(X)$ and $\sigma\in\Gamma(X,E)$.
\end{definition}

\begin{theorem}
    Given a Riemannian manifold $(X,g)$, there exists a unique connection on $\pi:TX\to X$,
    known as the \textbf{Levi-Civita connection}, satisfying
    \begin{enumerate}[(i)]
        \item symmetry:
            \begin{equation*}
                \nabla_XY-\nabla_YX-[X,Y]=0,
            \end{equation*}
            for $X,Y\in\Gamma(X,TX)$;
        \item compatibility with $g$:
            \begin{equation*}
                Xg(Y,Z)-g(\nabla_XY,Z)-g(Y,\nabla_XZ)=0,
            \end{equation*}
            for $X,Y,Z\in\Gamma(X,TX)$.
    \end{enumerate}
\end{theorem}

\begin{definition}
    Let $v$ be a vector field on $(X,g)$; we define the \textbf{covariant derivative} of $v$
    along a smooth curve $c:I\to X$ to be the vector field
    \begin{align*}
        \frac{Dv}{dt} = \nabla_{dc/dt}v,
    \end{align*}
    where $\nabla$ is the Levi-Civita connection. Explicitly, if we write $v=v^i\partial/\partial x^i$
    and $c(t)=(c_1(t),\ldots,c_n(t))$,
    \begin{equation*}
        \frac{Dv}{dt} = \sum_i \frac{dv^i}{dt}\frac{\partial}{\partial x^i} + \sum_{ijk}\frac{dc_i}{dt}v^i\Gamma_{ij}^k\frac{\partial}{\partial x^k}.
    \end{equation*}
    Here $\Gamma_{ij}^k$ are the Christoffel symbols of $\nabla$, determined by
    \begin{equation*}
        \nabla_{\partial/\partial x^i}\frac{\partial}{\partial x^j}=\sum_{ijk}\Gamma^k_{ij}\frac{\partial}{\partial x^k}.
    \end{equation*}
    We say that $c$ is \textbf{geodesic} at some $t\in I$ if $D/dt(dc/dt)=0$ at $t$,
    and that $c$ is geodesic if it is geodesic at all $t\in I$. In coordinates, the condition
    for $c$ to be geodesic is given by a system of second-order differential equations:
    \begin{equation*}
        \frac{d^2 c^i}{dt^2} + \sum_{jk}\Gamma_{jk}^i\frac{dc^j}{dt}\frac{dc^k}{dt}=0,
    \end{equation*}
    for $i=1,\ldots, n.$
\end{definition}

For the rest of the section, assume $(X,g)$ is Riemannian and we fix the Hamiltonian $H:M=T^*X\to\R$ as
\begin{equation*}
    H(x,\xi)=\frac{1}{2}\big|\xi_x\big|^2_g,
\end{equation*}
i.e. consisting of only a kinetic term. Here we are implicitly using the nondegeneracy of $g$
to associate $\xi_x$ with its corresponding vector (or, equivalently, using $g^{-1}$).

\begin{proposition}
    The Hamiltonian flow on $M=T^*X$ is dual to the geodesic flow on $TX$. In other words, the integral
    curves of the Hamiltonian vector field $v_H$ associated to the Hamiltonian above project to geodesics
    of $g$ on $X$.\footnote{Is there a coordinate-free proof? See Paternain's book.}
\end{proposition}
\begin{proof}
    It suffices to show, in coordinates, that Hamilton's equations (i.e. the condition for
    being on the integral curve) yield the geodesic equations above after the necessary
    dualization. Note first that in coordinates the Hamiltonian becomes
    \begin{equation*}
        H(x,\xi) = \frac{1}{2}g^{ij}\xi_i\xi_j.
    \end{equation*}
    For convenience we will denote the components of an integral curve as $x^i(t)$. Hamilton's equations yield
    \begin{align*}
        \frac{dx^i}{dt} &= \frac{\partial}{\partial\xi_i}\left( \frac{1}{2}g^{jk}\xi_j\xi_k \right)\\
        &= \frac{1}{2}g^{jk}\delta_{ij}\xi_k + \frac{1}{2}g^{jk}\xi_j\delta_{ik}\\
        &= g^{ij}\xi_j\\
        \frac{d\xi_i}{dt} &= -\frac{\partial}{\partial x^i}\left( \frac{1}{2}g^{jk}\xi_j\xi_k \right)\\
        &=-\frac{1}{2}\frac{\partial g^{jk}}{\partial x^i}\xi_j\xi_k.
    \end{align*}
    Differentiating the first equation with respect to $t$ and using both of Hamilton's equations yields
    \begin{align*}
        \frac{d^2x^i}{dt^2} &= \frac{\partial g^{ij}}{\partial x^k}\frac{dx^k}{dt}\xi_j+g^{im}\frac{d\xi_m}{dt}\\
        &= g^{kl}\left(\frac{\partial}{\partial x^k}g^{ij}\right)\xi_l\xi_j-\frac{1}{2}g^{im}\left(\frac{\partial}{\partial x^m}g^{nr}\right)\xi_n\xi_r.
    \end{align*}
    Next, differentiating the identity $g^{ij}g_{jk}=\delta^i_k$, it easy to see that
    \begin{align*}
        \frac{\partial}{\partial x^i}g^{kl}=-g^{la}g^{kb}\frac{\partial}{\partial x^i}g_{ab}.
    \end{align*}
    Using this, contracting indices, and using the first Hamilton's equation to dualize $\xi$'s into $dx/dt$'s,
    we find
    \begin{align*}
        \frac{d^2x^i}{dt^2} &= -g^{ib}\left( \frac{\partial}{\partial x^k}g_{lb} \right)\frac{dx^k}{dt}\frac{dx^l}{dt}
        +\frac{1}{2}g^{im}\left( \frac{\partial}{\partial x^m}g_{ts} \right)\frac{dx^s}{dt}\frac{dx^t}{dt}\\
        &= -\frac{1}{2}g^{ib}\left( \frac{\partial}{\partial x^k}g_{lb} \right)\frac{dx^k}{dt}\frac{dx^l}{dt}
        -\frac{1}{2}g^{ib}\left( \frac{\partial}{\partial x^l}g_{kb} \right)\frac{dx^k}{dt}\frac{dx^l}{dt}
        \\&+\frac{1}{2}g^{im}\left( \frac{\partial}{\partial x^m}g_{ts} \right)\frac{dx^s}{dt}\frac{dx^t}{dt}\\
        &= -\Gamma_{kl}^i\frac{dx^k}{dt}\frac{dx^l}{dt},
    \end{align*}
    as desired.
\end{proof}


\newpage
\section{Week 2}

\subsection{Darboux's theorem}

\begin{theorem}[Darboux]
    Let $(M,\omega)$ be a symplectic $2n$-manifold. Then $M$ is locally symplectomorphic to
    $(\R^{2n},\omega_{\R^{2n}})$.
\end{theorem}

We prove Darboux's theorem using the following stronger statement.

\begin{theorem}[Moser's trick]
    Let $M$ be a $2n$-dimensional manifold and $Q\subset M$ be a compact submanifold. Suppose that
    $\omega_1,\omega_2\in\Omega^2(M)$ are closed 2-forms such that at each point $q$ of $Q$ the forms
    $\omega_0$ and $\omega_1$ are equal and nondegenerate on $T_qM$. Then there exist neighborhoods
    $N_0$ and $N_1$ of $Q$ and a diffeomorphism $\psi:N_0\to N_1$ such that $\psi|_Q=\id_Q$ and
    $\psi^*\omega_1=\omega_0$.
    \label{thm:moser}
\end{theorem}
\begin{proof}
    Consider the family of closed two-forms
    \begin{equation*}
        \omega_t = \omega_0 + t(\omega_1-\omega_0)
    \end{equation*}
    on $M$ for $t\in[0,1]$. Note that $\omega_t|_Q=\omega_0|_Q$ is nondegenerate and hence there exists
    an open neighborhood $N_0$ of $Q$ such that $\omega_t|_{N_0}$ is nondegenerate.\footnote{Why?} Suppose,
    for now, that there is a one-form $\sigma\in\Omega^1(N_0)$ (possibly shrinking $N_0$), such that
    $\sigma|_{T_QM}=0$ and $d\sigma=\omega_1-\omega_0$ on $N_0$. Then
    \begin{equation*}
        \omega_t = \omega_0+td\sigma
    \end{equation*}
    and we obtain by nondegeneracy a smooth vector field $X_t$ on $N_0$ characterized by
    \begin{equation*}
        \iota_{X_t}\omega_t = -\sigma.
    \end{equation*}
    The condition $\sigma|_{T_QM}=0$ implies, again by nondegeneracy of $\omega_t$, that $X_t|_Q=0$.
    Now consider the initial value problem for the flow $\psi_t$ of $X_t$,
    \begin{align*}
        \frac{d}{dt}\psi_t &= X_t \circ \psi_t\\
        \psi_0 &= \id.
    \end{align*}
    This differential equation can be solved uniquely for $t\in[0,1]$ on some open neighborhood of
    $Q$ contained in $N_0$, call it again $N_0$.\footnote{Why?} Note that $\psi_t|_Q=\id_Q$ since $X_t|_Q=0$.
    We compute now that
    \begin{align*}
        \frac{d}{dt}\psi_t^*\omega_t &= \psi_t^*\left( \frac{d}{dt}\omega_t + \mathcal{L}_{X_t}\omega_t \right)\\
        &= \psi_t^*\left( d\sigma + d\iota_{X_t}\omega_t\right)\\
        &= 0.
    \end{align*}
    Hence $\psi_1^*\omega_1=\psi_0^*\omega_0=\omega_0$. Thus the desired diffeomorphism is
    $\psi_1$ and the desired neighborhoods are $N_0$ and $N_1$.
    The above argument is known as \textbf{Moser's trick}, and is extremely useful in symplectic
    geometry.
    
    It remains to construct a smooth one-form $\sigma$ satisfying $\sigma|_{T_QM}=0$ and $d\sigma=\omega_1-\omega_0$.
    If $Q$ were a point (or more generally, diffeomorphic to a star-shaped subset of Euclidean space),
    we could simply use the Poincar\'e lemma; in general, however the construction is as follows.
    Fix any Riemannian metric on $M$ and consider the restriction of the exponential map
    $\exp: TM\to M$ to a neighbhorhood $U_\varepsilon$ of the zero section of the normal bundle
    $TQ^\perp\to M$:
    \begin{equation*}
        U_\varepsilon = \{(q,v)\in TM \mid q\in Q, v\in T_qQ^\perp, |v|<\varepsilon\}.
    \end{equation*}
    Recall that $\exp$ becomes a diffeomorphism for $\varepsilon$ sufficiently small, so we choose
    $\varepsilon$ such that $N_0=\exp(U_\varepsilon)$ is contained in the neighborhood of $Q$ above
    on which $\omega_t$ is nondegenerate. Define now a family of maps $\phi_t:N_0\to N_0$
    for $t\in[0,1]$ by
    \begin{equation*}
        \phi_t(\exp(q,v)) = \exp(q,tv).
    \end{equation*}
    Note that $\phi_t$ is a diffeomorphism onto its image for $t\neq 0$. Moreover, $\phi_t|_Q=\id_Q$,
    $\phi_0(N_0)$, and $\phi_1 = \id_{N_0}$. If we now write $\tau = \omega_1-\omega_0$, we find that
    \begin{align*}
        \phi_0^*\tau &= 0\\
        \phi_1^*\tau &= \tau,
    \end{align*}
    since $\tau=0$ on $T_QM$. Now, for $t\in(0,1]$, we define a family of vector fields,
    \begin{equation*}
        Y_t = \left( \frac{d}{dt}\phi_t \right)\circ\phi_t^{-1}.
    \end{equation*}
    Then for any $\delta>0$,
    \begin{align*}
        \phi_1^*\tau - \phi_\delta^*\tau &= \int_\delta^1 \frac{d}{dt}\phi_t^*\tau dt=\int_\delta\phi_t^*\mathcal{L}_{Y_t}\tau dt\\
        &= \int_\delta^1 \phi_t^*(d\iota_{Y_t}\tau)dt\\
        &= d\int_\delta^1 \phi_t^*(\iota_{Y_t}\tau)dt
    \end{align*}
    Clearly $\phi_1^*\tau-\phi_\delta^*\tau=\tau-\phi_\delta^*\tau$ approaches $\tau$ as $\delta\to0^+$,
    so we find that
    \begin{equation*}
        \tau=d\int_0^1\phi_t^*(\iota_{Y_t}\tau)dt.
    \end{equation*}
    Defining
    \begin{equation*}
        \sigma = \int_0^1\phi_t^*(\iota_{Y_t}\tau) dt,
    \end{equation*}
    we find that $\tau=\omega_1-\omega_0=d\sigma$ and $\sigma|_{T_QM}=0$ because $\phi_t|_Q=\id_Q$ and
    $\tau=0$ on $Q$, forcing the integrand to vanish on $T_QM$. Hence $\sigma$ is the one-form required
    above for Moser's trick, and we are done.\footnote{Why is $\sigma$ smooth?}
    \label{thm:moser}
\end{proof}

The proof of Darboux's theorem is now straightforward: we choose a coordinate chart $\phi$ so that
$\phi^*\omega$ is equal to the standard form on a subset of $\R^{2n}$ at a single point, and then
apply Moser's theorem with $Q$ equal to the chosen point.

\begin{proof}[Proof of Darboux's theorem]
    Let $q\in M$ and fix a symplectic basis $\{u_i,v_i\}$ for the symplectic vector space $(T_qM,\omega_q)$.
    Fix any Riemannian metric on $M$ and pick an open $U\ni 0$ small enough such that $\exp$ restricted to
    $U\subset T_qM$ is a diffeomorphism and hence a chart $(x^i,y_i)=\exp:U\subset\R^{2n}\to M$
    ($i=1,\ldots,n$) such that $x^i(p)=y_i(p)=0$. Now we can compute, for example,
    \begin{align*}
        \exp^*\omega_p\left( \frac{\partial}{\partial x^j},\frac{\partial}{\partial y^k} \right) &=\omega_p\left(\exp_*\frac{\partial}{\partial x^j},\exp_*\frac{\partial}{\partial y^k}\right)\\
        &=\omega_p\left( u_j,v_k \right)=\delta_{jk},
    \end{align*}
    to check that $\exp^*\omega_p=(\omega_0)_0$ where $\omega_0$ is the standard form on $T_0U$.
    Here we have used the fact that $\exp_*=\id$ at $0\in U$.
    Applying Theorem \ref{thm:moser} to $U$ with $Q=0\in U$, we obtain a diffeomorphism $\psi$ of
    (some possibly smaller) $U$ such that $\psi^*\exp^*\omega=\omega_0$ on $U$. But now
    $\exp\circ\psi$ provides a symplectomorphism in a neighborhood of $q$ to a neighborhood of $\R^{2n}$
    pulling $\omega$ back to the standard form $\omega_0$.
\end{proof}

\newpage

\section{Week 3}


\subsection{Submanifolds of symplectic manifolds}

\begin{definition}
    Let $(V,\omega)$ be a symplectic vector space. We define the \textbf{symplectic complement}
    $U^{\omega}$ of a subspace $U\subset V$ as
    \begin{equation*}
        U^{\omega} = \{v\in V \mid \omega(v,u)=0\text{ for all }u\in U\}.
    \end{equation*}
\end{definition}

\begin{lemma}
    For any subspace $U\subset V$, $U^{\omega\omega}=U$ and
    \begin{equation*}
        \dim U + \dim U^{\omega} = \dim V.
    \end{equation*}
    \label{lem:dim}
\end{lemma}
\begin{proof}
    Nondegeneracy of $\omega$ yields an isomorphism $\iota_{\omega}:V\to V^*$ which identifies
    $U^\omega$ with $U^{\perp}\equiv\{\nu\in V^*\mid \nu(u)=0\text{ for all }u\in U\}$. The result
    now follows from the fact that $\dim U + \dim U^{\perp} = \dim V$.
\end{proof}

\begin{definition}
    Let $(M,\omega)$ be a symplectic manifold. A submanifold $Q\subset M$ is called
    \textbf{symplectic, isotropic, coisotropic, or Lagrangian} if for each $q\in Q$, the linear
    subspace $T_qQ\equiv V_q$ of $(T_qM,\omega_q)$ is
    \begin{enumerate}[(a)]
        \item symplectic: $V_q\cap V_q^{\omega_q}=0$,
        \item isotropic: $V_q\subset V_q^{\omega_q}$,
        \item coisotropic: $V_q^{\omega_q}\subset V_q$,
        \item Lagrangian: $V_q=V_q^{\omega_q}$,
    \end{enumerate}
    respectively.
\end{definition}

\begin{remark}
    Note that $Q\subset M$ is Lagrangian if and only if the restriction of $\omega$ to $Q$
    is zero and $\dim Q=\dim M/2$.
\end{remark}

\begin{example}
    Let $X$ be any manifold, and $(M=T^*X,\omega)$ be its cotangent bundle with the usual symplectic
    structure. Recall that $\omega=-d\theta$, where $\theta_\xi(v)=\xi(d_x\pi(v))$.\footnote{Can we do
    this coordinate-invariantly?}
    In coordinates, if $(x^i,\xi^i)$ are coordinates for $M$, we can write 
    $\omega=dx^i\wedge d\xi^i$.

    It is then easy to see that the fibre $T_x^*X\subset M$ is Lagrangian, as 
    \begin{align*}
        0 &= (dx^i\wedge d\xi^i)\left( a_j\frac{\partial}{\partial\xi^j}, b_k\frac{\partial}{\partial\xi^k}+c_l\frac{\partial}{\partial x^l} \right)\\
        &= (dx^i\wedge d\xi^i)\left( a_j\frac{\partial}{\partial\xi^j},c_l\frac{\partial}{\partial x^l} \right)\\
        &= a_ic_i,
    \end{align*}
    forces $c_i=0$.

    Similarly, the zero section $\Gamma_0\subset M$ is Lagrangian, as
    \begin{align*}
        0 &= (dx^i\wedge d\xi^i)\left( a_j\frac{\partial}{\partial x^j}, b_k\frac{\partial}{\partial\xi^k}+c_l\frac{\partial}{\partial x^l} \right)\\
        &= (dx^i\wedge d\xi^i)\left( a_j\frac{\partial}{\partial x^j},b_k\frac{\partial}{\partial \xi^k} \right)\\
        &= a_ib_i,
    \end{align*}
    forces $b_i=0$.

    More generally, given a submanifold $Q\subset L$, the annihilator
    \begin{equation*}
        TQ^{\perp} = \{(q,\nu)\in T^*L \mid q\in Q, \nu|_{T_qQ}=0\}
    \end{equation*}
    is Lagrangian.
\end{example}

\begin{example}
    Let $(M,\omega)$ be a symplectic manifold. The product $M\times M$ can be given
    a symplectic structure $\omega'=\alpha\pi_1^*\omega+\beta\pi_2^*\omega$ for $\alpha,\beta\in\R$.
    Consider in particular the case of $\alpha=1,\beta=-1$. Then it is clear that $M\times\{m\}$
    and $\{m\}\times M$ are symplectic submanifolds. Moreover, the diagonal $\Delta\subset M\times M$
    is Lagrangian, as
    \begin{align*}
        0&=\omega'\left( (u,u), (v,w) \right)\\
        &=\omega(u,v)-\omega(u,w)\\
        &=\omega(u,v-w)
    \end{align*}
    and hence $v=w$, as desired.
\end{example}

\begin{example}
    Let $S\subset (M,\omega)$ be a codimension 1 submanifold. Then $S$ is coisotropic.
    Indeed, fix $s\in S$, and note that $T_sS\subset T_sM$ is codimension one. By Lemma
    \ref{lem:dim}, $T_sS^{\omega_s}$ is a one-dimensional subspace. Pick any vector
    $v\in T_sS^{\omega_s}$; $v$ spans the entire symplectic complement, and hence if $v$
    is not in $T_sS^{\omega_s}$, $T_sS\cap T_sS^{\omega_s}=0$ and $T_sS$ is symplectic
    and thus even-dimensional. This is a contradiction, and hence $T_sS$ must be
    coisotropic.
\end{example}

\begin{proposition}
    The graph $\Gamma_\sigma\subset T^*X$ of a one-form is Lagrangian if and only if $\sigma$
    is closed.
\end{proposition}
\begin{proof}
    Note that $\Gamma_\sigma$ is defined to be the image of the embedding $\sigma:X\to T^*X$.
    Then $\dim\Gamma_\sigma=n$, so it remains to show that $\omega$
    restricts to zero on $\Gamma_\sigma$ if and only if $\sigma$ is closed. Using Proposition
    \ref{prop:can}, we compute
    \begin{align*}
        d\sigma = d\sigma^*\theta=\sigma^*d\theta=-\sigma^*\omega,
    \end{align*}
    which yields the desired statement, as $\sigma^*\omega=0$ on $X$ if and only if $\omega=0$
    on $\Gamma_\sigma$, by virtue of $\sigma$ being an embedding.
\end{proof}

With these definitions out of the way, we present a number of theorems characterizing
neighbhorhoods of special submanifolds of symplectic manifolds. 

\begin{theorem}[Symplectic neighborhood theorem]
    Let $(M_0,\omega_0),(M_1,\omega_1)$ be symplectic manifolds with compact symplectic
    submanifolds $Q_0,Q_1$ respectively. Suppose there is an isomorphism $\Phi:TQ_0^\omega\to TQ_1^\omega$
    of symplectic normal bundles covering a symplectomorphism $\phi:(Q_0,\omega_0)\to(Q_1,\omega_1)$.
    Then $\phi$ extends to a symplectomorphism $\psi:(N(Q_0),\omega_0)\to(N(Q_1),\omega_1)$
    such that $d\psi$ induces the map $\Phi$ on $TQ_0^{\omega}$.
\end{theorem}
\begin{proof}
    We use implicitly throughout that since $Q$ is symplectic, there is an isomorphism $TQ^\omega\to TQ^\perp$.
    Let $\exp_0,\exp_1$ be diffeomorphisms mapping neighborhoods of the zero section
    in the normal bundle to neighborhoods of $Q_0,Q_1$ in $X$, respectively. Then
    we obtain 
    \begin{equation*}
        \phi' = \exp_1\circ\Phi\circ\exp_0^{-1},
    \end{equation*}
    a diffeomorphism between these neighborhoods of $Q_0$ and $Q_1$. Now $\phi'^*\omega_1$
    and $\omega_0$ are two symplectic forms on $M_0$ whose restrictions to $Q_0$ agree.
    Now $\phi'$ extends to the desired $\psi$ by Theorem \ref{thm:moser}.
\end{proof}

\begin{theorem}[Lagrangian neighborhood theorem]
    Let $(M,\omega)$ be a symplectic manifold and let $L\subset M$ be a compact
    Lagrangian submanifold. Then there exists a neighborhood $N(\Gamma_0)\subset T^*L$
    of the zero section $\Gamma_0$, a neighborhood $U\subset M$ of $L$, and a
    diffeomorphism $\phi:N(\Gamma_0)\to U$ such that $\phi^*\omega=-d\theta$ and
    $\phi|_L=\id$, where $\theta$ is the canonical one-form on $T^*L$.
    \label{thm:lnt}
\end{theorem}
We postpone the proof of this theorem until after the discussion of complex structures.

\subsection{Contact manifolds}

Let $X$ be a differential manifold and $H\subset TX$ be a smooth hyperplane field,
i.e. a smooth subbundle of codimension one.
Then, locally on some open $U$, we can write $H=\ker\alpha$,
for $\alpha\in\Omega_1(U)$. In fact, if we assume that $H$ is \textbf{coorientable},
we can extend $U$ to all of $X$.\footnote{Why?} We will assume for what
follows that $H$ is coorientable.

\begin{definition}
    Let $X$ be a manifold of odd dimension $2n+1$. A \textbf{contact structure}
    on $X$ is a hyperplane field $H=\ker\alpha$ where the top-dimensional form
    $\alpha\wedge(d\alpha)^n$ is nowhere vanishing. We call $\alpha$ a
    \textbf{contact form}, and the pair $(X,H)$ a \textbf{contact manifold}.
\end{definition}

\begin{remark}
    Suppose we have $\alpha,\alpha'\in\Omega^1(X)$ such that $H=\ker\alpha=\ker\alpha'$.
    Then $\alpha$ is a contact form if and only if $\alpha'$ is. This is because
    the condition that $\alpha,\alpha'$ cut out $H$ requires $\alpha'=f\alpha$ for some
    nonzero $f:X\to\R$.
\end{remark}

\begin{remark}
    In the language of distributions, $H$ can be described as a codimension one
    distribution that is maximally non-integrable in the following sense. Recall
    that a distribution on $X$ is said to be integrable if every point $p$ of $X$ is
    contained in a integral manifold of $H$, i.e. in a nonempty immersed submanifold
    $N\subset X$ such that $T_pN=H_p$. The Frobenius theorem tells us that $H$ is
    integrable if and only if $H$ is involutive, i.e. $H$ is closed under the Lie
    bracket of local sections. Now, since
    \begin{equation*}
        d\alpha(X,Y) = X\alpha(Y)-Y\alpha(X)-\alpha[X,Y],
    \end{equation*}
    we find that $H$ is integrable if and only if $d\alpha=0$ on $H$. Thus asking for
    $d\alpha$ to be nondegenerate on $H$ forces the distribution to be ``as non-integrable
    as possible.''
    
    Indeed, we obtain the above definition of a contact structure by
    noting that $d\alpha$ is nondegenerate on $H$ if and only if $\alpha\wedge(d\alpha)^n$
    is nowhere vanishing, as follows. By remark \ref{rem:nondeg}, $d\alpha$ is nondegenerate
    on $H$ if and only if $(d\alpha)^n$ is nowhere vanishing, but this is simply
    equivalent to asking that $\alpha\wedge(d\alpha)^n$ be nowhere vanishing.
\end{remark}

Armed simply with the definition of a contact manifold, one might think that contact
geometry is somewhat obscure. We provide the following list of examples as evidence
that contact manifolds are actually quite common.

\begin{example}
    Let $X=\R^{2n+1}$ with coordinates $(x^1,\ldots,x^n,y^1,\ldots,y^n,z)$. The
    one-form
    \begin{equation*}
        \alpha = dz + x^i dy^i
    \end{equation*}
    is a contact form, as
    \begin{equation*}
        \alpha\wedge(d\alpha)^n = dz\wedge dx^1\wedge dy^1\wedge\cdots\wedge dx^n\wedge dy^n,
    \end{equation*}
    which is nowhere vanishing. We define the standard contact structure on $\R^{2n+1}$
    to be $H=\ker\alpha$.
\end{example}

For the next few examples the following lemma will be useful.
\begin{lemma}
    Let $(M,\omega)$ be a symplectic manifold of dimension $2n$.
    A vector field $Y$ on $M$ satisfying
    $\mathcal{L}_Y\omega=\omega$ is called a \textbf{Liouville vector field}. In
    this case, $\alpha=\iota_Y\omega$ is a contact form on any hypersurface $Q\subset M$
    transverse to $Y$ (i.e. at any point $p$, $T_pQ$ and $Y_p$ span $T_pM$).
    \label{lem:Liouville}
\end{lemma}
\begin{proof}
    Cartan's magic formula in this case tells us that $\omega=d\iota_Y\omega$, and hence
    \begin{align*}
        \alpha\wedge (d\alpha)^{n-1} &= \iota_Y\omega\wedge\omega^{n-1}\\
        &= \iota_Y(\omega^n)/n.
    \end{align*}
    Now, since $\omega^n$ is a volume form on $M$, we find that $\alpha\wedge(d\alpha)^{n-1}$
    is a volume form when restricted to the tangent bundle of any hypersurface transverse
    to $Y$.
\end{proof}

\begin{example}
    Consider $M=\R^4$ with its usual symplectic form $\omega=dx^1\wedge dy^1+dx^2\wedge dy^2$.
    The vector field
    \begin{equation*}
        Y=\frac{1}{2}\left( x_1\frac{\partial}{\partial x^1}+y^1\frac{\partial}{\partial y^1}+x^2\frac{\partial}{\partial x^2}+y^2\frac{\partial}{\partial y^2} \right)
    \end{equation*}
    is clearly transverse to the sphere $S^3$ given by $(x^1)^2+(y^1)^2+(x^2)^2+(y^2)^2=1$.
    It is a straightforward computation to check that $Y$ is Liouville, using
    the identity
    \begin{equation*}
        (\mathcal{L}_Y\omega)(v,w) = \mathcal{L}_Y(\omega(v,w)) - \omega([Y,v],w) - \omega(v,[Y,w]).
    \end{equation*}
    We conclude, using the previous lemma, that $S^3$ is a contact manifold, with a contact
    structure $\ker\iota_Y\omega$. This example is easily extended to show that $S^{2n+1}$
    has a contact structure.
\end{example}

\begin{example}
    Let $(M,g)$ be a Riemannian $n$-manifold. We define the \textbf{unit cotangent bundle}
    \begin{equation*}
        ST^*M = \{(p,\xi)\in T^*M \mid |\xi_p|_g^2=1\}\subset T^*M.
    \end{equation*}
    The unit cotangent bundle is a manifold of dimension $2n-1$ as it can be written as the level set of a
    Hamiltonian $H(p,\xi)=|\xi_p|^2_g/2$. Moreover, it is a sub-fiber bundle of the
    cotangent bundle, with fiber $S^{n-1}$. We claim that the canonical one-form on $T^*M$
    is a contact form for $ST^*M$. Indeed, let $Y$ be a vector field on $T^*M$ given by $\iota_Y\omega=\theta$.
    Then $Y$ is Liouville: $d(\iota_Y\omega)=d\theta=\omega$. In coordinates, $Y=p^i\partial/\partial p^i$,
    and hence is transverse to $ST^*M$. Note that if $M$ is compact, so is $SY^*M$ and in this
    case $ST^*M$ is an example of a compact contact manifold.
\end{example}

\begin{example}
    Let $(M,H=\ker\alpha)$ be a contact manifold. Then, if $\pi_M:M\times\R\to M$ is the projection
    onto the second factor, we claim that $(M\times\R, \omega=d(e^t\pi^*_M\alpha))$ is a symplectic manifold.
    Indeed, if $M$ has dimension $2n-1$, we compute
    \begin{align*}
        \omega^n &= (e^tdt\wedge\pi^*_M\alpha+\pi_M^*d\alpha)^n\\
        &=ne^{nt}dt\wedge\pi_M^*\alpha\wedge\pi_M^*(d\alpha)^{n-1}\\
        &=ne^{nt}dt\wedge\pi^*_M\left(\alpha\wedge (d\alpha)^{n-1}\right)\\
        &\neq0.
    \end{align*}
    We call $(M\times\R,d(e^t\pi_M^*\alpha))$ the \textbf{symplectization} of $(M,\alpha)$.
    Note that $\partial/\partial t$ is a Liouville vector field for $\omega$\footnote{compute!}
    and $M\subset M\times\R$ is a hypersurface transverse to $\partial/\partial t$.
\end{example}

\begin{definition}
    A \textbf{contactomorphism} from $(M_1,H_1)$ to $(M_2,H_2)$ is a diffeomorphism $f:M_1\to M_2$
    such that $df(H_1)=H_2$. Equivalently, if $H_1=\ker\alpha_1$ and $H_2=\ker\alpha_2$ then we
    require $f^*\alpha_2=g\alpha_1$ for some nowhere vanishing function $g:M_1\to\R\setminus\{0\}$.
\end{definition}

\newpage

\section{Week 4}
\subsection{Symplectic linear group and linear complex structures}
%Let $(V,\omega)$ be a real symplectic vector space. We define the symplectic linear group $\Sp(V,\omega)$
%to be the group of symplectomorphisms of $V$.
%Since all symplectic vector spaces are symplectomorphic
%to $(\R^{2n},\omega_0)$, it suffices to consider $\Sp(2n)=\Sp(2n,\R)=\Sp(\R^{2n},\omega_0)$.
%We claim that $\Sp(2n)$ is not compact; indeed, its maximal compact subgroup is $U(n)$.
%\begin{proposition}
%    The unitary group $U(n)$ is a maximal compact subgroup of $\Sp(2n)$ and the quotient
%    $\Sp(2n)/U(n)$ is contractible.
%\end{proposition}
%
%\begin{lemma}
%    Viewing $\GL(n,\C)$ as a closed subgroup of $\GL(2n,\R)$ under the inclusion
%    \begin{equation*}
%            A+iB \mapsto
%            \begin{pmatrix}
%                A & -B\\
%                B & A
%            \end{pmatrix}.
%    \end{equation*}
%\end{lemma}
%
\begin{definition}
    Let $(V,\omega)$ be a symplectic vector space. We denote the group of symplectomorphisms
    from $V$ to itself as $\Sp(V,\omega)$, the \textbf{symplectic linear group}. In the case
    of the standard symplectic structure on $\R^{2n}$ we write the group as $\Sp(2n)$.
\end{definition}

\begin{lemma}
    A real $2n\times 2n$ matrix $\Psi$ is in $\Sp(2n)$ if and only if
    \begin{equation*}
        \Psi^\top J_0\Psi = J_0,
    \end{equation*}
    where
    \begin{equation*}
        J_0=
        \begin{pmatrix}
            0 & -I_n\\
            I_n & 0
        \end{pmatrix}\in\Sp(2n).
    \end{equation*}
    \label{lem:matrixform}
\end{lemma}
\begin{proof}
    Let $u_i,v_i$ be a symplectic basis for $V$. For $x,y\in V$ write $x=(a,b),y=(c,d)$
    for $a,b,c,d\in\R^n$. Then
    \begin{align*}
        \omega(x,y) &= a^id^i - b^ic^i = -x^\top J_0y.
    \end{align*}
    Clearly $\Psi^*\omega=\omega$ if and only if $\Psi^\top J_0\Psi=J_0$.
\end{proof}


\begin{definition}
    Let $V$ be a vector space. A \textbf{complex structure} on $V$ is an automorphism
    $J:V\to V$ such that $J^2=-\id_V$. We denote the set of all complex
    structures on $V$ by $\mathcal{J}(V)$. Now suppose $(V,\omega)$ is a symplectic
    vector space. We say that a complex structure $J$ is \textbf{compatible} with
    $\omega$ if
    \begin{equation*}
        \omega(Jv, Jw)=\omega(v,w)
    \end{equation*}
    for all $v,w\in V$, and
    \begin{equation*}
        \omega(v,Jv)>0
    \end{equation*}
    for all nonzero $v\in V$. We denote the set of all compatible complex structures
    on $(V,\omega)$ by $\mathcal{J}(V,\omega)$.
\end{definition}

\begin{lemma}
    Let $J\in\mathcal{J}(V,\omega)$ be a compatible complex structure on $(V,\omega)$.
    Then 
    \begin{equation*}
        g_J(v,w) = \omega(v,Jw)
    \end{equation*}
    defines an inner product on $V$.
    \label{lem:inner}
\end{lemma}

\begin{lemma}
    Let $(V,\omega)$ be a symplectic vector space and $J$ be a complex structure on $V$.
    Then the following are equivalent:
    \begin{enumerate}[(a)]
        \item $J$ is compatible with $\omega$;
        \item the bilinear form $g_J:V\times V\to\R$ defined by
            \begin{equation*}
                g_J(v,w)=\omega(v,Jw)
            \end{equation*}
            is symmetric, positive-definite, and $J$-invariant.
        \item if we view $V$ as a complex vector space with $J$ as its complex structure,
            the form $H:V\times V\to\C$ defined by
            \begin{equation*}
                H(v,w) = \omega(v,Jw)+i\omega(v,w)
            \end{equation*}
            is complex linear in $w$, complex antilinear in $v$, satisfies $H(w,v)=\overline{H(v,w)}$,
            and has a positive-definite real part. Such a form is called a \textbf{Hermitian inner product}
            on $(V,J)$.
    \end{enumerate}
\end{lemma}
\begin{proof}
    That (a) implies (b) is clear from Lemma \ref{lem:inner}. For (b) implies (c), note first
    that the real part of $H$ is simply $g_J$ and hence is positive-definite. For linearity,
    we compute
    \begin{align*}
        H(Jv,w) &= \omega(Jv,Jw) + i\omega(Jv,w)\\
        &= g_J(Jv,w) - ig_J(w,v)\\
        &= g_J(w,Jv) - ig_J(v,w)\\
        &= -i H(v,w),
    \end{align*}
    and
    \begin{align*}
        H(v,Jw) &= -\omega(v,w) + i\omega(Jv,Jw)\\
        &= -\omega(v,w)+ig_J(Jv,w)\\
        &= -\omega(v,w)+i\omega(v,w)\\
        &= iH(v,w),
    \end{align*}
    as desired. Finally, note that
    \begin{align*}
        H(w,v) &= \omega(w,Ju)+i\omega(w,v)\\
        &= \omega(v,Jw)-i\omega(v,w)\\
        &= \overline{H(v,w)}.
    \end{align*}
    For (c) implies (a), $\omega(v,Jv)>0$ because the real part $\omega(v,Jw)$ is by hypothesis
    positive-definite. Moreover, $\omega(Jv,Jw)=\im H(Jv,Jw)=\im H(v,w)=\omega(v,w)$.
\end{proof}


The following result shows that all linear complex structures are isomorphic to the
standard complex structure.

\begin{proposition}
    Let $V$ be a $2n$-dimensional real vector space and let $J\in\mathcal{J}(V)$. Then
    there exists a vector space isomorphism $\Phi:\R^{2n}\to V$ such that
    \begin{equation*}
        J\Phi = \Phi J_0.
    \end{equation*}
    \label{lem:complex_iso}
\end{proposition}
\begin{proof}
    Consider the extension $J^\C$ of $J$ to the complexification $V^\C=V\otimes_\R\C\cong V$
    given by $J\otimes1$. Clearly $J^\C$ is a complex structure on $V^\C$ and thus has
    eigenvalues $\pm i$. We obtain a direct sum decomposition $V^\C\cong E^+\oplus E^-$
    of the $\pm i$ eigenspaces respectively, i.e. $J^\C|_{E^\pm}=\pm iI$. Clearly $\dim_\C E^{\pm}=n$.
    We claim that a basis $w_j=u_j+iv_j$ for $E^+$ yields a basis $u_j,v_j$ for $V$.
    It suffices to show that these vectors are linearly independent. Since $w_j$ is a basis
    for $E^+$,
    \begin{equation*}
        \sum_{j=1}^n (a_j+ib_j)(u_j\otimes 1+v_j\otimes i) = 0
    \end{equation*}
    for $a_j,b_j\in\R$ implies that $a_j=b_j=0$ for all $j$. Suppose there exist $\alpha_j,\beta_j\in\R$
    such that
    \begin{equation*}
        \sum_{j=1}^n \alpha_ju_j+\beta_jv_j=0.
    \end{equation*}
    Now since $w_j\in\ker(I-iJ)$, a straightforward computation reveals that $Ju_j=-v_j$ and $Jv_j=u_j$.
    Applying $J$ to the above equation, we obtain
    \begin{equation*}
        \sum_{j=1}^n \beta_j u_j - \alpha_j v_j = 0.
    \end{equation*}
    Then, taking $a_j=\beta_j,b_j=\alpha_j$, we find that
    \begin{align*}
        \sum_{j=1}^n(\beta_j+i\alpha_j)(u_j\otimes 1+v_j\otimes i) &= \left( \sum_{j=1}^n\beta_ju_j-\alpha_jv_j \right)\otimes 1
        + \left( \sum_{j=1}\beta_jv_j+\alpha_ju_j \right)\otimes i\\
        &=0.
    \end{align*}
    Linear independence of the $w_j$ now forces $\alpha_j=\beta_j=0$. Hence $u_j,v_j$ forms a basis
    for $V$.

    The required $\Phi:\R^{2n}\to V$ can now be written explicitly as
    \begin{equation*}
        \Phi(x_1,\ldots,x_n,y_1,\ldots,y_n) = \sum_{j=1}^n\left( x_ju_j-y_jv_j \right).
    \end{equation*}
    This map is clearly an isomorphism; moreover, if $x=(r_1,\ldots,r_n,s_1,\ldots,s_n)\in\R^{2n}$
    then
    \begin{equation*}
        J\Phi x = -s_1u_1-r_1v_1-\cdots-s_nu_n-r_nv_n = \Phi J_0x,
    \end{equation*}
    as desired.
\end{proof}

\begin{remark}
    Define an action of $\GL(2n,\R)$ on the set $\mathcal{J}(V)$ by $g\cdot J=g^{-1}Jg$.
    By Lemma \ref{lem:complex_iso}, $\GL(2n,\R)\cdot J_0=\mathcal{J}(V)$, i.e. the orbit
    of $J_0$ is the entire set. Moreover, since $\GL(n,\C)$ is naturally embedded (as a Lie
    subgroup) in $\GL(2n,\R)$ as $\{A\in\GL(2n,\R) \mid J_0A=AJ_0\}$, the stabilizer of $J_0$ is
    $\GL(n,\C)$.\footnote{The embedding is given by replacing each entry $a+bi$ with a block of the
    form $\begin{pmatrix}a&-b\\b&a\end{pmatrix}$.}
    We conclude that $\mathcal{J}(V)$ can be given the structure of a smooth manifold such
    that $\mathcal{J}(V)\cong\GL(2n,\R)/\GL(n,\C)$.
    \label{rem:gl}
\end{remark}

The following result shows that the choice of complex structure compatible with a fixed
symplectic form on $V$ is canonical up to homotopy.

\begin{proposition}
    The set $\mathcal{J}(V,\omega)$ of compatible complex structures is naturally identified
    with the space $\mathcal{P}$ of symmetric positive-definite symplectic matrices. In particular,
    $\mathcal{J}(V,\omega)$ is contractible.
    \label{prop:contr}
\end{proposition}
\begin{proof}
    By fixing a symplectic basis for $V$ we may assume that $(V,\omega)=(\R^{2n},\omega_0)$.
    By the proof of Lemma \ref{lem:matrixform}, we note that $J\in\Aut(\R^{2n})$ is a compatible
    complex structure if and only if the conditions
    \begin{align*}
        J^2 &= -\id_{\R^{2n}},\\
        J_0&=J^\top J_0J,\\
        0&<-v^\top J_0Jv,
    \end{align*}
    hold (for $v\neq0$). Set $P=J_0J$. $P$ is symmetric, since
    \begin{equation*}
        (J_0J)^\top = -J^\top J_0=J^\top J_0J^2=J_0J,
    \end{equation*}
    as well as positive-definite, and symplectic. Moreover, it is easy to check that 
    if any matrix $P$ has these three properties, then $J=-J_0P$ is a compatible complex structure.
    Hence $\mathcal{J}(V,\omega)$ is in bijective correspondence with the space $\mathcal{P}$ of
    symmetric positive-definite symplectic matrices.
    It remains to show that $\mathcal{P}$ is contractible. Suppose, for now, that if $P\in\mathcal{P}$
    then $P^{\alpha}\in\mathcal{P}$ for all $\alpha>0$, $\alpha\in\R$. Then the map
    $h:[0,1]\times\mathcal{P}\to\mathcal{P}$ given by $h(t,P)=P^{1-t}$ is a homotopy from
    $\id_\mathcal{P}$ to the constant map $P\mapsto \id_V$, and we are done.

    We now show that if $P\in\mathcal{P}$ then $P^\alpha\in\mathcal{P}$ for all $\alpha>0$.
    It is easy to see that $P^\alpha$ is symmetric and positive-definite. It remains to
    show that $\omega_0(P^\alpha v,P^\alpha w)=\omega_0(v,w)$ for all $\alpha>0$. Decompose
    $\R^{2n}$ into eigenspaces $V_\lambda$ for eigenvalues $\lambda$ of $P$. Note that for
    a symplectic matrix $P$, if $\lambda,\lambda'$ are eigenvalues such that $\lambda\lambda'\neq1$
    then $\omega_0(z,z')=0$, where $z,z'$ are the eigenvectors of $\lambda,\lambda'$, respectively:
    \begin{equation*}
        \lambda\lambda'\omega_0(z,z')=\omega_0(Pz,Pz')=\omega_0(z,z').
    \end{equation*}
    Now, since $V_\lambda$ is also the eigenspace for the eigenvalue $\lambda^\alpha$ for $P^\alpha$,
    if $z\in V_\lambda,z'\in V_{\lambda'}$,
    \begin{equation*}
        \omega_0(P^\alpha z,P^\alpha z') = (\lambda\lambda')^\alpha\omega_0(z,z').
    \end{equation*}
    Writing any $v,w\in\R^{2n}$ in the basis of eigenvectors for $P^\alpha$, we find by linearity,
    and the remarks about $\lambda,\lambda'$ above, that $\omega_0(P^\alpha v,P^\alpha w)=\omega_0(v,w)$
    for all $\alpha>0$.
\end{proof}

Often it is enough to consider a slightly weaker notion of compatibility.
\begin{definition}
    A complex structure $J\in\mathcal{J}(V)$ is called \textbf{$\omega$-tame} if $\omega(v,Jv)>0$
    for all nonzero $v\in V$. The set of all $\omega$-tame complex structures on $V$ is
    written $\mathcal{J}_{\tau}(V,\omega)$. Note that $\mathcal{J}_\tau(V,\omega)$ is an
    open subset of $\mathcal{J}(V)\cong\GL(2n,\R)/\GL(n,\C)$ (as per Remark \ref{rem:gl}).
\end{definition}
In this case, we note that $g_J(v,w)=(\omega(v,Jw)+\omega(w,Jv))/2$ defines an inner product on $V$,
for all $J\in\mathcal{J}_\tau(V,\omega)$. We note that there is an analog of Proposition \ref{prop:contr}
for $\omega$-tame complex structures.

\begin{proposition}
    The space $\mathcal{J}_\tau(V,\omega)$ is contractible.
\end{proposition}
\begin{proof}
    See, for instance, McDuff/Salamon or Gromov.
\end{proof}

%\begin{proof}
%    Let $\Omega$ be the set of symplectic forms on $V$. By an argument similar to that in Remark
%    \ref{rem:gl}, we have a homeomorphism $\Omega\cong\GL(2n,\R)/\Sp(2n)$.
%    Define
%    \begin{align*}
%        \mathcal{C}_\tau &= \{(\omega,J)\in\Omega\times\mathcal{J}(V)\mid J\in\mathcal{J}_\tau(V,\omega)\},\\
%        \mathcal{C} &= \{(\omega,J)\in\Omega\times\mathcal{J}(V)\mid J\in\mathcal{J}(V,\omega)\}.
%    \end{align*}
%    Clearly $\mathcal{C}_\tau$ injects into $\mathcal{C}$.
%\end{proof}


\subsection{Symplectic vector bundles}

\begin{definition}
    A \textbf{symplectic vector bundle} $(E,\omega)$ over $X$ is a real vector bundle
    $\pi:E\to X$ together with a smooth symplectic bilinear form $\omega\in\Gamma(X,E^*\wedge E^*)$,
    i.e. a symplectic bilinear form on each $E_x$ that varies smoothly with $x$.
    A \textbf{complex structure} on $\pi:E\to M$ is a bundle automorphism $J:E\to E$
    such that $J^2=-\id_{E}$. We say $J$ is \textbf{compatible} with $\omega$ if the
    induced complex structure on $E_x$ is compatible with $\omega_x$ for all $x\in X$.
    We thus obtain a symmetric, positive-definite bilinear form
    $g_J\in\Gamma(X,\Sym^2 E^*)$, and we call the triple $(E, \omega, g_J)$ a
    \textbf{Hermitian structure} on $E$.
\end{definition}

\begin{theorem}
    Let $E\to X$ be a $2n$-dimensional vector bundle. For any symplectic structure
    $\omega$ on $E$, the space of compatible complex structures is nonempty and
    contractible. For any complex structure $J$ on $E$, the space of symplectic
    structures compatible with $J$ is nonempty and contractible.
    \label{thm:herm}
\end{theorem}
\begin{proof}
    See McDuff/Salamon.\footnote{Understand this!}
\end{proof}

We now prove the Theorem \ref{thm:lnt}, the Lagrangian neighborhood theorem, with the help
of the following lemma.

\begin{lemma}
    Let $J\in\mathcal{J}(V,\omega)$. Then a subspace $\Lambda\subset V$ is Lagrangian
    if and only if $J\Lambda^\perp=\Lambda$ with respect to $g_J$.
\end{lemma}
\begin{proof}
    %If $\Lambda$ is Lagrangian then for $v\in\Lambda,w\in V$,
    %\begin{equation*}
    %    g_J(Jv,w)=\omega(Jv,Jw)=\omega(v,w)=0
    %\end{equation*}
    %if and only if $w\in\Lambda$ and hence $J\Lambda^\perp=\Lambda$. Conversely,
    %if $J\Lambda^\perp=\Lambda$, for $v\in\Lambda,w\in V$,
    %\begin{equation*}
    %    \omega(v,w)=\omega(Jv,Jw)=g_J(Jv,w)=0
    %\end{equation*}
    %if and only if $w\in J\Lambda^\perp=\Lambda$ and hence $\Lambda$ is Lagrangian.
    For $v\in\Lambda,w\in V$, the assertion that
    \begin{equation*}
        g_J(Jv,w)=\omega(Jv,Jw)=\omega(v,w)=0
    \end{equation*}
    implies that $\Lambda$ is Lagrangian if and only if $J\Lambda^\perp=\Lambda$.
\end{proof}

\begin{theorem}[Lagrangian neighborhood theorem]
    Let $(M,\omega)$ be a symplectic manifold and let $L\subset M$ be a compact
    Lagrangian submanifold. Then there exists a neighborhood $N(\Gamma_0)\subset T^*L$
    of the zero section $\Gamma_0$, a neighborhood $U\subset M$ of $L$, and a
    diffeomorphism $\phi:N(\Gamma_0)\to U$ such that $\phi^*\omega=-d\theta$ and
    $\phi|_L=\id$, where $\theta$ is the canonical one-form on $T^*L$.
\end{theorem}
\begin{proof}
    By Theorem \ref{thm:herm}, we can fix an arbitrary complex structure $J$ on the
    tangent bundle $TM$ and denote the associated metric by $g_J$. Note that the metric yields
    a diffeomorphism of bundles $\Phi:T^*L\to TL$ given by
    \begin{equation*}
        g_J(\Phi_q(v^*),v)=v^*(v)
    \end{equation*}
    for $v\in T_qL,v^*\in T_q^*L$. Now the map $\phi:T^*L\to M$ defined by
    \begin{equation*}
        \phi(q,v^*)=\exp_q(J_q\Phi_qv^*)
    \end{equation*}
    is a diffeomorphism from some neighborhood $N(\Gamma_0)$ of $\Gamma_0$ onto its image $U$,
    where $\exp$ is the exponential map on $M$ corresponding to $g_J$.
    
    Now if $v=(v_0,v_1^*)\in T_{(q,0)}T^*L=T_qL\oplus T_q^*L$, we claim that
    \begin{equation*}
        d\phi_{(q,0)}(v) = v_0+J_q\Phi_qv_1^*.
    \end{equation*}
    By linearity, it suffices to compute $\d\phi_{(q,0)}$ on $T_qL$ and $T_q^*L$ separately.
    In particular, let $c:[0,1]\to TM$ be a curve given by $c(t)=(a(t),0)$, with $c'(0)=(v_0,0)$.
    Then
    \begin{align*}
        d\phi_{(q,0)}(v_0,0) &= \frac{d}{dt}\bigg|_{t=0}\exp_{a(t)}\left( J_{a(t)}\Phi_{a(t)}0 \right)\\
        &=\frac{d}{dt}\bigg|_{t=0}a(t)\\
        &=v_0.
    \end{align*}
    Next take $c(t)=(q,tv_1^*)$. Clearly $c'(0)=(0,v_1^*)$. Then
    \begin{align*}
        d\phi_{(q,0)}(0,v_1^*) &= \frac{d}{dt}\bigg|_{t=0}\exp_p(J_p\Phi_p tv_1^*)\\
        &= J_p\Phi_pv_1^*,
    \end{align*}
    as desired.
    
    We can now compute, for $v=(v_0,v_1^*),w=(w_0,w_1^*)\in T_{(q,0)}T^*L$,
    \begin{align*}
        \phi^*\omega_{(q,0)}(v,w) &= \omega_q\left( v_0+J_q\Phi_qv_1^*, w_0+J_q\Phi_qw_1^* \right)\\
        &= \omega_q(v_0,J_q\Phi_qw_1^*)-\omega_q(w_0,J_q\Phi_qv_1^*)\\
        &= g_J(v_0,\Phi_qw_1^*) - g_J(w_0,\Phi_qv_1^*)\\
        &= w_1^*(v_0) - v_1^*(w_0)\\
        &= -d\theta_{(q,0)}(v,w).
    \end{align*}
    This shows that $\phi^*\omega=-d\theta$ on the zero section. Now the result follows
    from Moser's trick, Theorem \ref{thm:moser}.
\end{proof}

\newpage
\section{Week 5}

\subsection{Almost complex manifolds}
\begin{definition}
    Let $M$ be a $2n$-dimensional real manifold. An \textbf{almost complex structure} on
    $M$ is a complex structure $J$ on the tangent bundle $TM$.
    In this situation  we say that $(M,J)$ is an almost complex manifold.
    The almost complex structure
    is \textbf{compatible} with a nondegenerate two-form $\omega$ on $M$ if $J$ is compatible 
    with $\omega$.
\end{definition}

\begin{theorem}
    For each nondegenerate two-form $\omega$ on $M$ the space of almost complex structures
    compatible with $\omega$ is nonempty and contractible. Conversely, for every almost
    complex structure on $M$ the space of compatible nondegenerate two-forms is nonempty
    and contractible.
\end{theorem}
\begin{proof}
    See Theorem \ref{thm:herm}.
\end{proof}

\begin{example}
    Let $X\subset\R^3$ be an oriented hypersurface. Let $\nu:X\to S^2$ be the Gauss map,
    which assigns to each point $x\in X$ the outward-pointing normal vector $\nu(x)\perp T_xX$.
    Define, for $u\in T_xX$,
    \begin{equation*}
        J_xu = \nu(x)\times u,
    \end{equation*}
    where the product is the vector (cross) product on $\R^3$. It follows from the vector triple
    product identity $a\times (b\times c)=b(g(a,c))-c(g(a,b))$, where $g$ is the standard
    metric on $\R^3$, that $J_x^2=-\id_{T_xX}$. Define a two-form $\omega$ on $X$ by 
    \begin{align*}
        \omega(v,w) &= \iota(\nu(x))\Omega\\
        &= g(\nu(x),v\times w),
    \end{align*}
    where $\Omega(u,v,w)$ is the determinant of the matrix whose columns are $u,v,w$.
    It is straightforward to check that $J$ is compatible with $\omega$: for $v,w\in T_xX$,
    \begin{align*}
        \omega(J_xv,J_xw) &= g\left(\nu(x),(\nu(x)\times v)\times(\nu(x)\times w)\right)\\
        &= g\left( \nu(x),\nu(x) g(\nu(x)\times v,w) \right)\\
        &= g(w,\nu(x)\times v)\\
        &= g(\nu(x),v\times w)\\
        &= \omega(v,w)\\
        \omega(v, J_xv) &= g(\nu(x), v\times (\nu(x)\times v))\\
        &=g(\nu(x),g(v,v)\nu(x))\\
        &=g(v,v)\\
        &>0,
    \end{align*}
    where we have used the vector triple product identity as well as the cyclic property of
    the scalar triple product.
    \label{ex:vecprod}
\end{example}

\begin{example}
    Consider $S^2\subset\R^3$ with the almost complex structure $J$ from the previous example.
    We compute the expression of $J$ in stereographic coordinates. Recall we have $\phi:S^2-(0,0,1)\to\R^2$
    given by
    \begin{equation*}
        \phi(x,y,z) = \left( \frac{x}{1-z}, \frac{y}{1-z} \right)
    \end{equation*}
    and inverse
    \begin{equation*}
        \psi(X,Y) = \left( \frac{2X}{1+X^2+Y^2},\frac{2Y}{1+X^2+Y^2},\frac{X^2+Y^2-1}{X^2+Y^2+1} \right).
    \end{equation*}
    For a point $p=(x,y,z)\in S^2$ and a vector $u=(v,w)\in T_pS^2$, some computation reveals that
    \begin{align*}
        J_p(v,w) &= d\phi\left((x,y,z)\times d\psi(v,w)\right)\\
        &= (w, -v).
    \end{align*}
\end{example}

\begin{definition}
    Let $(X,J)$ be an almost complex manifold. We define the \textbf{Nijenhuis tensor} $N_J$ by
    \begin{equation*}
        N_J(v,w) = [v,w] + J[Jv,w] + J[v,Jw] - [Jv,Jw]
    \end{equation*}
    for $v,w$ vector fields on $X$.
\end{definition}

\begin{lemma}
    The Nijenhuis tensor is a skew-symmetric covariant (2,0)-tensor on $X$ satisfying
    \begin{enumerate}[(a)]
        \item $N_J(v,Jv)=0$ for all vector fields $v$;
        \item $N_{J_0}=0$;
        \item If $\phi\in\Diff(M)$ and $v,w$ are vector fields then
            \begin{equation*}
                N_{\phi^*J}(\phi^*v,\phi^*w) = \phi^*N_J(v,w).
            \end{equation*}
    \end{enumerate}
\end{lemma}
\begin{proof}
    Writing $v=v^i\partial/\partial x^i, w=w^i\partial/\partial x^i$ in local coordinates,
    the Lie bracket $[v,w]$ becomes\footnote{We follow McDuff/Salamon in the convention that
        $[v,w]\equiv-\mathcal{L}_vw$.}
    \begin{equation*}
        [v,w] = \left( w^j\frac{\partial v^i}{\partial x^j} - v^j\frac{\partial w^i}{\partial x^j} \right)\frac{\partial}{\partial x^i}.
    \end{equation*}
    Finish this.
\end{proof}

Suppose now that $(X,J)$ is an almost complex manifold. Denote by $T_\C X$ the complexification
of the real vector bundle $TX$, i.e. $T_\C X=TX\otimes\C$. We note that the complexified
tangent bundle splits into $\pm i$ $J$-eigenbundles $T_\C X = T^{1,0}X\oplus T^{0,1}X$,
respectively.\footnote{Here, $J$ is really $J\otimes\C$.} These are often referred to as the
holomorphic and antiholomorphic tangent bundles of $X$.

\begin{definition}
    Let $X$ be an almost complex manifold. We define the vector bundles
    \begin{align*}
        \bigwedge\nolimits^k_\C X &\equiv \bigwedge\nolimits^k(T_\C X)^*\\
        \bigwedge\nolimits^{p,q}X &\equiv \bigwedge\nolimits^p(T^{1,0}X)^*\otimes_\C\bigwedge\nolimits^q(T^{0,1}X)^*.
    \end{align*}
    and write $\mathcal{A}^k_{X,\C}$ and $\mathcal{A}^{p,q}_X$ for their sheaves of sections,
    respectively. We denote the projections $\mathcal{A}^\bullet\to \mathcal{A}^k$ and
    $\mathcal{A}^\bullet\to\mathcal{A}^{p,q}$ by $\Pi^k$ and $\Pi^{p,q}$ respectively. It
    is not hard to show that
    \begin{align*}
        \bigwedge\nolimits^k_\C X &= \bigoplus_{p+q=k}\bigwedge\nolimits^{p,q}X\\
        \mathcal{A}^k_{\C} &= \bigoplus_{p+q=k}\mathcal{A}^{p,q}
    \end{align*}
    and additionally, that $\overline{\bigwedge^{p,q}X}=\bigwedge^{q,p}X$ and
    $\overline{\mathcal{A}^{p,q}}=\mathcal{A}^{q,p}$.
    Now if $d:\mathcal{A}^k_\C\to\mathcal{A}^{k+1}_\C$ is the exterior
    derivative\footnote{Here, $d$ is really $d\otimes\C$.}, we write
    \begin{align*}
        \partial &\equiv \Pi^{p+1,q}\circ d\\
        \bar\partial &\equiv \Pi^{p,q+1}\circ d,
    \end{align*}
    and $\partial,\bar\partial$ satisfy the appropriate graded Leibniz rule.
\end{definition}

With this notation now set, we come to the key definition.
\begin{proposition}
    Let $(X,J)$ be an almost complex manifold. Then the following conditions are equivalent:
    \begin{enumerate}[(a)]
        \item $d=\partial+\bar\partial$ on $\mathcal{A}^\bullet$;
        \item $\Pi^{0,2}\circ d=0$ on $\mathcal{A}^{1,0}$;
        \item $[T^{0,1}X,T^{0,1}X]\subset T^{0,1}X$;
        \item $N_J=0$.
    \end{enumerate}
    If $X$ satisfies one of these equivalent conditions then $J$ is said to be an
    \textbf{integrable} almost complex structure.
    \label{prop:integrable}
\end{proposition}
\begin{proof}
    We show that (a) is equivalent to (b), (b) is equivalent to (c), and that (c) is
    equivalent to (d).

    For (a)$\leftrightarrow$(b),
    suppose first that $d=\partial+\bar\partial$ and $\alpha\in\mathcal{A}^{1,0}$.
    Then
    \begin{align*}
        \Pi^{0,2}d\alpha &= \Pi^{0,2}(\partial+\bar\partial)\alpha\\
        &= \Pi^{0,2}(\Pi^{2,0}+\Pi^{1,1})d\alpha\\
        &= 0.
    \end{align*}
    Conversely, suppose $\Pi^{0,2}d=0$ on $\mathcal{A}^{1,0}$. Clearly $d=\partial+\bar\partial$
    if and only if $d\alpha\in\mathcal{A}^{p+1,q}\oplus\mathcal{A}^{p,q+1}$ for
    all $\alpha\in\mathcal{A}^{p,q}$. Now any $\alpha\in\mathcal{A}^{p,q}$ can locally
    be written as a linear combination of terms of the form
    $f_{IJ}w_{i_1}\wedge\cdots\wedge w_{i_p}\wedge w'_{j_1}\wedge\cdots\wedge w'_{j_q}$,
    with the $w\in \mathcal{A}^{1,0}$ and $w'\in\mathcal{A}^{0,1}$. Then $d\alpha$
    is expressed as a linear combination of terms involving $df_{IJ}$, $dw_i$, and $dw_j'$.
    We have that $df\in\mathcal{A}^2_\C=\mathcal{A}^{1,0}\oplus\mathcal{A}^{0,1}$,
    which takes care of the terms containing $df_{IJ}$. Similarly, since $\Pi^{0,2}d=0$
    on $\mathcal{A}^{1,0}$ by assumption, $dw_i\in\mathcal{A}^{2,0}\oplus\mathcal{A}^{1,1}$,
    which takes care of the terms containing the $dw_i$. Finally, we have that
    $dw_j'\in\mathcal{A}^{1,1}\oplus\mathcal{A}^{0,2}$ since $\Pi^{2,0}d=0$ on $\mathcal{A}^{0,1}$
    (seen by conjugating (b)), which takes care of the terms containing the $dw'_j$. We
    conclude that $d\alpha\in\mathcal{A}^{p+1,q}\oplus\mathcal{A}^{p,q+1}$, as desired.

    We now prove (b)$\leftrightarrow$(c). Fix any $\alpha\in\mathcal{A}^{1,0}$ and
    $v,w$ sections of $T^{0,1}$. Then, by definition of $d\alpha$, and since $\alpha$
    vanishes on $T^{0,1}$, we find that
    \begin{align*}
        (d\alpha)(v,w) &= v\alpha(w)-w\alpha(v)-\alpha[v,w]\\
        &= -\alpha[v,w].
    \end{align*}
    We conclude that $\Pi^{0,2}d=0$ if and only if $[v,w]\in T^{0,1}$.

    We now prove $(c)\leftrightarrow(d)$. Suppose for now that any section of $T^{0,1}$
    can be written as $v+iJv$ for $v$ a section of $TX\otimes\C$. Then
    \begin{equation*}
        [v+iJv, w+iJw] = [v,w]-[Jv,Jw]-i\left( [Jv,w]+[v,Jw] \right).
    \end{equation*}
    This is of the form $u+iJu$ if and only if
    \begin{equation*}
        J\left( [v,w]-[Jv,Jw] \right)=[Jv,w]+[v,Jw],
    \end{equation*}
    which is equivalent to $N_J(v,w)=0$. It remains to show that any section of
    $T^{0,1}$ can be written as $v+iJv$. Finish this.
\end{proof}

\begin{example}
    Let $X$ be a complex manifold. Then we have local coordinates $z_i,\bar z_i$
    for $i=1,\ldots,n$ and the standard almost complex structure $J_0$ 
    acting as $i$ on $\partial/\partial z_i$ and $-i$ on $\partial/\partial\bar z_i$.
    Now we note that for $\alpha\in\mathcal{A}^{p,q}$ written $\alpha=\alpha_{IJ}dz^I\wedge d\bar z^J$,
    we have
    \begin{equation*}
        d\alpha = \left(\frac{\partial\alpha_{IJ}}{\partial z^k}dz^k+\frac{\partial\alpha_{IJ}}{\partial\bar z^k}d\bar z^k\right)\wedge dz^I\wedge d\bar z^J.
    \end{equation*}
    Clearly then $d=\partial + \bar\partial$, as $\partial=\Pi^{p+1,q}d$ and $\bar\partial=\Pi^{p,q+1}$.
    Hence, by Proposition \ref{prop:integrable}(a), $J_0$ is integrable.
\end{example}

The above example shows that complex manifolds induce integrable almost complex
structures on their underyling real manifolds in a natural way. It is a highly
nontrivial fact that the converse is also true.

\begin{theorem}[Newlander-Nirenberg, 1957]
    Let $(X,J)$ be an almost complex manifold. Then $J$ is integrable if and only if
    $X$ has a holomorphic atlas (making it a complex manifold) such that the induced almost
    complex structure is $J$.
\end{theorem}

\begin{example}
    Let $(X,J)$ be a two-dimensional almost complex manifold. In this case
    $\mathcal{A}^2_\C=\mathcal{A}^{1,1}$ and hence by Proposition \ref{prop:integrable}(b),
    we find that $J$ is integrable. We conclude using the Newlander-Nirenberg theorem
    that every two-dimensional almost complex manifold is in fact a complex manifold.
    \label{ex:riemannintegrable}
\end{example}

\begin{example}
    It turns out that there exists a vector product on $\R^7$ that is bilinear and skew-symmetric,
    and hence it follows along the lines of Example \ref{ex:vecprod} that every oriented
    hypersurface $X\subset\R^7$ carries an almost complex structure. This argument shows, in
    particular, that $S^6$ is an almost complex manifold. It was shown by Calabi, however,
    that this almost complex structure is not integrable. Indeed, the existence of an integrable
    almost complex structure on $S^6$ is still an open problem.
\end{example}

\subsection{K\"ahler manifolds}

\begin{definition}
    A \textbf{K\"ahler} manifold is a symplectic manifold $(M,\omega)$ equipped
    with an integrable almost complex structure $J\in\mathcal{J}(M,\omega)$.
\end{definition}

\begin{example}
    The most basic example of a K\"ahler manifold is $(\R^{2n},\omega_0,J_0)$.
    Indeed, viewing $\R^{2n}$ as $\C^n$ we can introduce coordinates $z^i=x^i+iy^i,\bar z^i=x^i-iy^i$ with
    respect to which $T^{1,0}\C^{n}$ and $T^{0,1}\C^{n}$ are trivialized by
    the frames $\partial/\partial z^i$ and $\partial/\partial \bar z^i$, respectively.
    Then it is straightforward to check that $d=\partial+\bar\partial$ on $\mathcal{A}^\bullet_\C$.
    In these coordinates,
    \begin{align*}
        dz^i &= dx^i+idy^i\\
        d\bar z^i &= dx^i - idy^i.
    \end{align*}
    and a easy computation reveals that the symplectic form $\omega_0$ can be written
    \begin{equation*}
        \omega_0 = \frac{i}{2}\sum_{i=1}^ndz^i\wedge d\bar z^i.
    \end{equation*}
    In fact, if we let $f=\sum_i^n\bar z^iz^i$, we can write $\omega_0=i\partial\bar\partial f/2$.
\end{example}

\begin{example}
    Every two-dimensional symplectic manifold is K\"ahler with respect to any compatible
    almost complex structure.
\end{example}

\begin{example}[Complex projective space]
    Let $\PP^n$ denote the complex projective space, which is a complex manifold of dimension $n$. 
    Let $J$ be the induced integrable almost complex structure.
\end{example}


\newpage
\section{Week 6}
\subsection{Poisson brackets}
\begin{definition}
    Let $(M,\omega)$ be a symplectic manifold. We say that a vector field $X\in\mathcal{X}(M)$
    is \textbf{symplectic} if
    \begin{equation*}
        d(\iota(X)\omega) = 0,
    \end{equation*}
    or equivalently,
    \begin{equation*}
        \mathcal{L}_X\omega=0.
    \end{equation*}
    We denote the Lie algebra of symplectic vector fields by $\mathcal{X}(M,\omega)$.
\end{definition}

\begin{proposition}
    Let $M$ be closed and let $X\in\mathcal{X}(M)$ be a smooth vector field with flow
    $F:I\times M\to M$. Then $F_t$ is a symplectomorphism for all $t$ if and only if
    $X$ is symplectic.
\end{proposition}
\begin{proof}
    Note that $F_t^*\omega:I\to \Gamma(M,\bigwedge^2T^*M)$ gives us a smooth curve in
    the vector space $\Gamma(M,\bigwedge^2T^*M)$. Then
    \begin{align*}
        \frac{d}{dt}\left( F_t^*\omega \right) &= \frac{d}{ds}\bigg|_{s=0}\left( F_{s+t}^*\omega\right)\\
        &=F_t^*\mathcal{L}_X\omega\\
        &= F_t^*d(\iota_{X}\omega)
    \end{align*}
    and we see that the curve is constant at $\omega$ if and only if $X\in\mathcal{X}(M,\omega)$.\footnote{Understand
    this computation better.}
\end{proof}

For the most part, we will focus on a subset of symplectic vector fields known as
Hamiltonian vector fields (also introduced in section 1).
\begin{definition}
    Let $H:M\to \R$ be a smooth function and let $X_H$ be the vector field determined
    uniquely by
    \begin{equation*}
        \iota_{X_H}\omega = dH.
    \end{equation*}
    We say that $X_H$ is a \textbf{Hamiltonian vector field} for the \textbf{Hamiltonian} $H$.
    If $M$ is closed, $X_H$ generates a smooth one-parameter group of symplectomorphisms
    $F_H^t$ as its flow. We call this the \textbf{Hamiltonian flow}
    associated to $H$. Computing as in the proof of the proposition above, we find that
    \begin{align*}
        \frac{d}{dt}\left( (F_H^t)^*H\right) &= X_HH = dH(X_H)\\
        &=(\iota_{X_H}\omega)(X_H)\\
        &=\omega(X_H,X_H)\\
        &=0.
    \end{align*}
    We conclude that $H$ is constant along the Hamiltonian flow.
\end{definition}

\begin{example}
    Sphere with cylindrical polar coordinates and H the height function.
\end{example}

\begin{definition}
    Let $k$ be a field.
    A \textbf{Poisson algebra} $A$ over $k$ is an $k$-vector space equipped with bilinear products
    $\cdot$ and $\{\cdot,\cdot\}$ such that
    \begin{enumerate}[(a)]
        \item the product $\cdot$ gives $A$ the structure of an associative $k$-algebra;
        \item the bracket $\{\cdot,\cdot\}$ gives $A$ the structure of a Lie algebra;
        \item the bracket $\{\cdot,\cdot\}$ is a $k$-derivation over the product $\cdot$.
    \end{enumerate}
\end{definition}

\begin{proposition}
    Let $(M,\omega)$ be a symplectic manifold. Define a product on $C^\infty(M)$ as
    \begin{equation*}
        \{f,g\} \equiv \omega(X_f,X_g).
    \end{equation*}
    Then $C^\infty(M)$ forms a real Poisson algebra.
    \label{prop:symppoisson}
\end{proposition}
\begin{proof}
    That $C^\infty(M)$ is an associative $\R$-algebra under multiplication is clear
    (in fact, it is even commutative).
    Now, since
    \begin{equation*}
        \iota_{X_{f_1}+X_{f_2}}\omega=\iota_{X_{f_1}}\omega+\iota_{X_{f_2}}\omega=df_1+df_2=d(f_1+f_2)=\iota_{X_{f_1+f_2}}\omega.
    \end{equation*}
    uniqueness forces $X_{f_1}+X_{f_2}=X_{f_1+f_2}$. It follows immediately that the
    Poisson bracket is bilinear.
    That the bracket is alternating follows from the fact that $\omega$ is.
    Similarly, since
    \begin{equation*}
        \iota_{gX_h+hX_g}\omega=g\iota_{X_h}\omega+h\iota_{X_g}\omega=gdh+hdg=d(gh)=\iota_{X_{gh}\omega},
    \end{equation*}
    we conclude that $X_{gh}=gX_h+hX_g$, and hence
    \begin{equation*}
        \{f,gh\}=\omega(X_f,X_{gh})=g\omega(X_f,X_h)+h\omega(X_f,X_g)=g\{f,h\}+h\{f,g\},
    \end{equation*}
    which proves the derivation property (that the bracket is zero on a constant in $\R$ is easy to check).
    
    It remains to check the Jacobi identity
    \begin{equation*}
        \{f,\{g,h\}\} + \{g,\{h,f\}\} + \{h,\{f,g\}\} = 0.
    \end{equation*}
    Using anticommutativity and the fact that
    \begin{equation*}
        \{f,g\}=(\iota_{X_f}\omega)(X_g)=df(X_g)=X_g f,
    \end{equation*}
    we can rewrite the left-hand side as
    \begin{align*}
        X_fX_gh-X_gX_fh+X_{\{f,g\}}h &= -[X_f,X_g]h+X_{\{f,g\}}h.
    \end{align*}
    Hence it suffices to show that $X_{\{f,g\}}=[X_f,X_g]$\footnote{We follow Mcduff/Salamon in the
        convention that $[X,Y]=-\mathcal{L}_XY$.}
    To see this, note that
    \begin{equation*}
        \mathcal{L}_{X_f}\iota_{X_g}\omega = d\iota_{X_f}\iota_{X_g}\omega=d\{g,f\}=\iota_{X_{\{g,f\}}}\omega
    \end{equation*}
    and, using Cartan's (second magic) formula,\footnote{See Morita's Geometry of Differential Forms, Theorem 2.11(1).}
    \begin{equation*}
        \mathcal{L}_{X_f}\iota_{X_g}\omega=\iota_{\mathcal{L}_{X_f}X_g}\omega+\iota_{X_g}\mathcal{L}_{X_f}\omega=\iota_{[X_g,X_f]}\omega
    \end{equation*}
    (since $\mathcal{L}_{X_f}\omega=0$), so
    \begin{equation*}
        \iota_{X_{\{g,f\}}}\omega=\iota_{[X_f,X_g]}\omega.
    \end{equation*}
    Now uniqueness implies that $X_{\{f,g\}}=[X_f,X_g]$, as desired.
\end{proof}

A manifold equipped with a Poisson algebra structure on its smooth functions is called a
Poisson manifold. The previous proposition shows that every symplectic manifold is a Poisson
manifold. The following example shows that the converse is not true, as a Poisson manifold
can have arbitrary dimension.

\begin{example}[Lie-Poisson structure]
    Let $\fr g$ be a real Lie algebra. Denote by $\fr g^*$ the dual vector space. Treating
    $\fr g^*$ as a manifold, we note that the de Rham differential of $f\in C^\infty(\fr g^*)$
    is $df_\alpha:T_\alpha\fr g^*=\fr g^*\to\R$ for $\alpha\in\fr g^*$. Since $\fr g^{**}$ is
    naturally identified with $\fr g$, it is easy to check that
    \begin{equation*}
        \{f,g\}(\alpha) = \alpha[dg_\alpha,df_\alpha].
    \end{equation*}
    provides a Poisson structure on $\fr g^*$.
\end{example}

Note that the Poisson algebras in the two examples above are
commutative in the product $\cdot$, but these need not be the case in general.

Morphisms in the category of Poisson manifolds? (see Wikipedia)

What happens if $H:M\to\R$ is Morse? This implies that $dH:M\hookrightarrow T^*M$ intersects the zero section
of $T^*M$ transversely. What does this give us?

Can we extend the Poisson structure to the exterior algebra of forms?


\subsection{Group actions}

Before discussing group actions on symplectic manifolds, we review some basic notions
from Lie theory. Let $G$ be a Lie group and $\fr g=T_eG$ be its Lie algebra, and denote
left (right) multiplication by $g$ as $L_g$ ($R_g$).

\begin{lemma}
    There is a Lie algebra isomorphism between the Lie algebra $\fr g$ of $G$ and the space
    of left-invariant vector fields on $G$. In particular $X\in\fr g$ is sent to the vector field
    $\tilde X$ satisfying $(L_g^*\tilde X)_h=\tilde X_{gh}$ for all $g\in G$ such that $\tilde X_e=X$.
\end{lemma}

\begin{lemma}
    The left-invariant vector fields on $G$ are complete, i.e. their flows define
    diffeomorphisms of $G$.
\end{lemma}
\begin{proof}
    By the uniqueness of integral curves, it suffices to show that if $\gamma:I\to G$ is
    an integral curve then $L_g\circ\gamma$ is as well. This is a straightforward computation:
    \begin{align*}
        \frac{d}{dt}\left( L_g\circ \gamma \right) &= (dL_g\circ d\gamma)\left( \frac{d}{dt} \right)\\
        &= dL_g(X_{\gamma(t)})\\
        &= X_{L_g\circ\gamma(t)},
    \end{align*}
    as desired.
\end{proof}

\begin{definition}
    The \textbf{exponential map} is the smooth map $\exp: \fr g\to G$ given by
    \begin{equation*}
        \exp(\xi) = \phi_\xi^1(e),
    \end{equation*}
    where $\phi_\xi^1:G\to G$ is the time 1 flow associated to the left-invariant
    vector field $\tilde \xi$. It is easy to see that $\exp(t\xi)=\phi^t_\xi(e)$.
    Moreover, if $[\xi,\eta]=0$ then $\exp(\xi+\eta)=\exp(\xi)\exp(\eta)$.
    Finally, for a morphism $f:G\to H$ of Lie groups, we obtain a commutative diagram
    \begin{equation*}
        \begin{tikzcd}
            G\ar{r}{f} & H\\
            \fr g\ar{r}{df_e}\ar{u}{\exp} & \fr h\ar[swap]{u}{\exp}
        \end{tikzcd}
    \end{equation*}
    which we will refer to as the naturality of $\exp$.
    Note that the differential at $e$ of a Lie group homomorphism is a Lie algebra
    homomorphism, as is $df_e$ here.
\end{definition}

We now consider the symplectic case.

\begin{proposition}
    The Lie algebra of the Lie group of symplectomorphisms $\Symp(M,\omega)$ is
    the space of symplectic vector fields $\mathcal{X}(M,\omega)$.
    \label{prop:sympvf}
\end{proposition}
\begin{proof}
    This involves dealing with time-dependent vector fields, so I'll work through
    it later.\footnote{Finish}
\end{proof}

Now suppose $G$ acts on
$(M,\omega)$ symplectomorphically, i.e. there is a group homomorphism $\psi:G\to\Symp(M,\omega)$
taking $g\mapsto\psi_g$. Differentiating this map at the identity yields a Lie
algebra homomorphism $d\psi_e:\fr g\to \mathcal{X}(M,\omega)$. We denote the image of $\xi\in\fr g$
under $d\psi_e$ by $\xi_M$. Now let $c:G\to\Aut(G)$ be the conjugation homomorphism $c_gh=ghg^{-1}$ and
denote by $\Ad:G\to\Aut(\fr g)$ the homomorphism taking $g$ to $(dc_g)_e$.

\begin{lemma}
    In the notation above, we have an equality of vector fields
    \begin{equation*}
        (\Ad_g\xi)_M = \psi_{g^{-1}}^*\xi_M,
    \end{equation*}
    for $\xi\in\fr g$.
\end{lemma}
\begin{proof}
    For $p\in M$ we can write, using the naturality of $\exp$ and the chain rule,\footnote{Review this.}
    \begin{align*}
        (\Ad_g\xi)_M(p) &= \frac{d}{dt}\bigg|_{t=0}\psi_{\exp(t\Ad_g\xi)}(p)\\
        &= \frac{d}{dt}\bigg|_{t=0}\psi_{g\exp(t\xi)g^{-1}}(p)\\
        &= \frac{d}{dt}\bigg|_{t=0}\psi_g\psi_{\exp(t\xi)}(\psi_{g^{-1}}(p))\\
        &= d\psi_g|_{\psi_{g^{-1}}(p)}\xi_M(\psi_{g^{-1}}(p))\\
        &= (\psi^*_{g^{-1}}d\psi|_e\xi)(p).
    \end{align*}
\end{proof}

\begin{definition}
    An action $\psi$ of $G$ on $(M,\omega)$ is \textbf{weakly Hamiltonian} if the vector
    field $\xi_M$ is Hamiltonian for each $\xi\in\fr g$, i.e.
    \begin{equation*}
        \iota_{\xi_M}\omega = dH_\xi
    \end{equation*}
    for some $H_\xi\in C^\infty(M)$. For a weakly Hamiltonian action, then, we obtain a map
    $\fr g\to C^\infty(M)$ taking $\xi\mapsto H_\xi$. This map is \textit{a priori} not even
    linear. However, since each $H_\xi$ is defined only up to a constant, we can choose the
    $H_\xi$ to make $\fr g\to C^\infty(M)$ linear.

    We say the action $\psi$ of $G$ on $(M,\omega)$ is \textbf{Hamiltonian} if the map
    $\fr g\to C^\infty(M)$ can be chosen to be a Lie algebra homomorphism (with respect
    to the Poisson structure on $C^\infty(M)$).
\end{definition}

\begin{definition}
    Suppose $\psi$ is a Hamiltonian action of $G$ on $(M,\omega)$. We say that a map
    $\mu:M\to\fr g^*$ is a \textbf{moment map} for the action if
    \begin{equation*}
        H_\xi(p) = \langle \mu(p), \xi\rangle,
    \end{equation*}
    where $\langle\cdot,\cdot\rangle$ is the pairing between $\fr g^*$ and $\fr g$.
\end{definition}

\begin{example}
    Consider the action of $S^1$ on the sphere (with its usual symplectic structure)
    that rotates the sphere about its vertical axis. More precisely, using cylindrical
    coordinates $\theta,z$ away from the poles, the action is given by $\psi:S^1\times S^2\to S^2$
    as $(\rho, (\theta,z))\mapsto (\theta+\rho,z)$. The associated Lie algebra
    action is then $d\psi_e:\fr u(1)\cong\R\to \mathcal{X}(S^2,\omega)$ given by
    $\xi\mapsto \xi\partial/\partial\theta$. In the notation above,
    \begin{equation*}
        \xi_M = \xi\frac{\partial}{\partial\theta}.
    \end{equation*}
    Now, since $\omega=d\theta\wedge dz$ away from the poles, we find that
    $\iota_{\xi_M} \omega = \xi dz$. Hence the action is Hamiltonian (the Poisson condition
    is trivial as $\fr u(1)$ is one-dimensional) since 
    \begin{equation*}
        H_\xi=\xi z.
    \end{equation*}
    We obtain a moment map $\mu:S^2\to\fr u(1)^*\cong\R$ given simply by
    \begin{equation*}
        \mu(\theta,z) = z.
    \end{equation*}
    This is simply the height function on the sphere, whose fibers are precisely
    the orbits of the $S^1$ action.
\end{example}

\begin{definition}
    Suppose $(M,\omega=-d\lambda)$ be an exact symplectic manifold. We say that the
    action $\psi$ of $G$ on $M$ is \textbf{exact} if $\psi^*_g\lambda=\lambda$ for each
    $g\in G$.
\end{definition}

\begin{remark}
    Recall that a closed symplectic $2n$-manifold $(M,\omega)$ cannot be exact. Indeed,
    if it were, the volume form $\omega^n$ would be exact and Stokes' theorem would imply
    that $\int_M\omega^n=0$, which is not possible. Hence for $M$ closed, $\omega$ must
    represent a nontrivial class in $\HH^{2}(M;\R)$.
\end{remark}

\begin{proposition}
    Let $(M,\omega=-d\lambda)$ be an exact symplectic manifold. Then every exact action of $G$
    on $M$ is Hamiltonian with
    \begin{equation*}
        H_\xi = \iota_{X_\xi}\lambda
    \end{equation*}
    for $\xi\in\fr g$.
\end{proposition}

\subsection{Cohomological obstructions}

In general, weakly Hamiltonian actions need not be Hamiltonian. 
In this section, we digress briefly to derive sufficient conditions for an action to be
weakly Hamiltonian, and a necessary condition for a weakly Hamiltonian action to be
Hamiltonian.
For this, we quickly present Lie algebra cohomology, following Ortega/Ratiu.\footnote{add citation!}

Let $G$ be a real Lie group of dimension $n$. Similarly to the case of vector fields, we say that a differential
$k$-form $\omega\in\Omega^k(G)$ is left invariant if $L_g^*\omega=\omega$ for each $g\in G$.
Note that left invariant $k$-forms can be identified with the $k$-forms $\Lambda^k\fr g^*$,
since they are determined by their action at the identity. We now obtain a chain complex
of left-invariant forms
\begin{equation*}
    \begin{tikzcd}
        0\ar{r}& \Lambda^0\fr g^*\cong\R\ar{r}& \Lambda^1\fr g\cong\fr g^*\ar{r} & \cdots\ar{r} & \Lambda^n\fr g^*\cong\R\ar{r} & 0,
    \end{tikzcd}
\end{equation*}
where the differentials are given by the expected formula: for $\omega\in\Lambda^k\fr g^*$,
\begin{equation*}
    d\omega(\xi_0,\ldots,\xi_k) = \sum_{0\leqslant i<j\leqslant k}(-1)^{i+j}\omega([\xi_i,\xi_j],\xi_0,\ldots,\hat\xi_i,\ldots,\hat\xi_j,\ldots,\xi_k).
\end{equation*}
We note the following low-dimensional cases, which will be the ones of interest to us.
For $\omega\in\Lambda^0\fr g^*$, clearly $d\omega=0$. For $\omega\in\Lambda^1\fr g^*\cong\fr g^*$,
\begin{equation*}
    d\omega(\xi_1,\xi_2) = -\omega [\xi_1,\xi_2]
\end{equation*}
and for $\omega\in\Lambda^2\fr g^*$,
\begin{equation*}
    d\omega(\xi_1,\xi_2,\xi_3)=-\omega([\xi_1,\xi_2],\xi_3) + \omega([\xi_3,\xi_1],\xi_2) + \omega([\xi_2,\xi_3],\xi_1).
\end{equation*}
We now define the \textbf{Lie algebra cohomology} $\HH^\bullet(\fr g,\R)$ to be the cohomology of the above complex.

\begin{remark}
    More generally, let $\fr g$ be a Lie algebra over $k$ and let $M$ be a $\fr g$-module.
    Denote by $-^\fr g:\fr g\textsc{-Mod}\to\fr g\textsc{-Mod}$ the invariants functor.
    Then one defines $\HH^\bullet(\fr g,M)$, the cohomology groups of $\fr g$ with coefficients
    in $M$, as the right derived functors $R^\bullet(-^\fr g)(M)$. Of course, this is much more
    generality than we will need; the formulation above is computing
    the cohomology of the Chevalley-Eilenberg resolution of $\R$.\footnote{Cite Weibel}
\end{remark}

The first cohomology group is quite easily computed. Indeed,
\begin{equation*}
    \HH^1(\fr g,\R)=\{\omega\in\fr g^*\mid \omega[\xi_1,\xi_2]=0\}.
\end{equation*}
Noting that $\omega\in\fr g^*$ is a map $\omega:\fr g\to\R$ annihilating precisely $[\fr g,\fr g]$,
and that such maps are in correspondence with maps $\fr g/[\fr g, \fr g]\to\R$, we conclude that
\begin{equation*}
    \HH^1(\fr g,\R) \cong (\fr g/[\fr g, \fr g])^*.
\end{equation*}
In particular, we note that if $\fr g$ is abelian or even semisimple, $\HH^1(\fr g,\R)=0$.
We now relate these cohomology groups back to Hamiltonian actions.

\begin{proposition}
    The commutator of two symplectic vector fields on $(M,\omega)$ is Hamiltonian.
\end{proposition}
\begin{proof}
    Let $X,Y\in\mathcal{X}(M,\omega)$, i.e. $d\iota_X\omega=d\iota_Y\omega=0$ or equivalently
    $\mathcal{L}_X\omega=\mathcal{L}_Y\omega=0$. Now, using both of Cartan's magic formulas,
    we find that
    \begin{align*}
        \iota_{[X,Y]}\omega &= [\mathcal{L}_X,\iota_Y]\omega\\
        &= \mathcal{L}_X\iota_Y\omega\\
        &= d\iota_X\iota_Y\omega.
    \end{align*}
    We conclude that $[X,Y]$ is Hamiltonian with $H_{[X,Y]}=\omega(X,Y)$.
\end{proof}

\begin{corollary}
    Suppose $G$ acts on $(M,\omega)$ through symplectomorphisms and that 
    $\HH^1(\fr g,\R)=0$ or $\HH^1_\mathrm{dR}(M,\R)=0$. Then the action is
    weakly Hamiltonian.
\end{corollary}
\begin{proof}
    If the first Lie algebra cohomology vanishes, we must have that $\fr g=[\fr g,\fr g]$.
    In particular, the induced symplectic vector fields on $M$ must be Hamiltonian by
    the previous proposition. If the first de Rham cohomology vanishes, every closed one-form
    on $M$ is exact and thus, by definition, every symplectic vector field is Hamiltonian.
\end{proof}

Less trivial are the obstructions for a weakly Hamiltonian action to be Hamiltonian.
\begin{proposition}
    Suppose the action of $G$ on $(M,\omega)$ is weakly Hamiltonian, where $M$ is connected.
    Then the action
    determines a cocycle $[\tau]\in\HH^2(\fr g,\R)$ which vanishes if and only if the
    action is Hamiltonian.
\end{proposition}
\begin{proof}
    Since the action is weakly Hamiltonian we may choose a linear map $\fr g\to C^\infty(M)$
    sending $\xi\mapsto H_\xi$ such that $\iota_{\xi_M}\omega=dH_\xi$.
    For each pair $\xi,\eta\in\fr g$, define a function on $M$
    \begin{equation*}
        \tau(\xi,\eta) = \{H_\xi,H_\eta\}-H_{[\xi,\eta]}.
    \end{equation*}
    Since
    \begin{equation*}
        X_{H_{[\xi,\eta]}}=[\xi,\eta]_M=[\xi_M,\eta_M]=[X_{H_\xi},X_{H_\eta}]=X_{\{H_\xi,H_\eta\}},
    \end{equation*}
    we find that
    \begin{equation*}
        d\left(H_{[\xi,\eta]}-\{H_\xi,H_\eta\}\right)=0
    \end{equation*}
    so $\tau$ is locally constant, hence constant. Clearly then $\tau\in\Lambda^2\fr g^*$.

    We now claim that $d\tau=0$, i.e.
    \begin{equation*}
        \tau([\xi,\eta],\zeta) + \tau([\eta,\zeta],\xi)+\tau([\zeta,\xi],\eta) = 0.
    \end{equation*}
    Reasoning as in the previous paragraph, we find that
    \begin{equation*}
        \{H_{[\xi,\eta]},H_\zeta\}=\{\{H_\xi,H_\eta\},H_\zeta\},
    \end{equation*}
    so by the Jacobi identity for the Poisson bracket,
    \begin{equation*}
        d\tau(\xi,\eta,\zeta)=-\left(H_{[ [\xi,\eta],\zeta]}+H_{[ [\eta,\zeta],\xi]}+H_{[ [\zeta,\xi],\eta]}\right)=0,
    \end{equation*}
    by linearity of the map $\xi\mapsto H_\xi$ and the Jacobi identity for $\fr g$.

    Hence $\tau$ represents a cocycle $[\tau]\in\HH^2(\fr g,\R)$. If the action is Hamiltonian
    to begin with, obviously $\tau=0$, since $\xi\mapsto H_\xi$ is a Lie algebra homomorphism.
    Conversely, suppose $[\tau]=0$. This is equivalent to asking that $\tau$ be a coboundary
    \begin{equation*}
        \tau(\xi,\eta)=\sigma[\xi,\eta]
    \end{equation*}
    for some $\sigma\in\fr g^*$. Modifying the given map $\xi\mapsto H_\xi$ to $\xi\mapsto H_\xi+\sigma(\xi)$,
    we find that
    \begin{equation*}
        [\xi,\eta]\mapsto H_{[\xi,\eta]} + \sigma[\xi,\eta]=\{H_\xi,H_\eta\},
    \end{equation*}
    and we conclude that the action is Hamiltonian.
\end{proof}

\begin{example}
    The second Whitehead lemma states that for $\fr g$ semisimple, $\HH^2(\fr g,\R)=0$.\footnote{Reference Weibel}
    Thus, if the Lie algebra of $G$ is semisimple, every weakly Hamiltonian $G$-action on
    $(M,\omega)$ is Hamiltonian.
\end{example}

\newpage
\section{Week 7}

\end{document}
