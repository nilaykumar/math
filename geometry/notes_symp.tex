\documentclass{amsart}

\usepackage[colorlinks=true]{hyperref}
\usepackage{enumerate}
\usepackage{tikz-cd}
\usepackage{amssymb}

\theoremstyle{plain}
\newtheorem{theorem}{Theorem}
\newtheorem{lemma}[theorem]{Lemma}
\newtheorem{proposition}[theorem]{Proposition}
\newtheorem{corollary}[theorem]{Corollary}

\theoremstyle{definition}
\newtheorem{definition}[theorem]{Definition}
\newtheorem{example}[theorem]{Example}
\newtheorem{exercise}[theorem]{Exercise}

\theoremstyle{remark}
\newtheorem{remark}[theorem]{Remark}

% Fonts
\newcommand{\A}{\mathbb{A}}
\newcommand{\C}{\mathbb{C}}
\newcommand{\R}{\mathbb{R}}
\newcommand{\Q}{\mathbb{Q}}
\newcommand{\Z}{\mathbb{Z}}
\newcommand{\N}{\mathbb{N}}
\newcommand{\G}{\mathbb{G}}
\newcommand{\fr}{\mathfrak}

% Topology/geometry

\DeclareMathOperator{\Gr}{Gr}
\DeclareMathOperator{\Fl}{Fl}
\DeclareMathOperator{\PP}{\mathbb{P}}
\DeclareMathOperator{\Der}{Der}
\DeclareMathOperator{\Lie}{Lie}
\DeclareMathOperator{\SL}{SL}
\DeclareMathOperator{\GL}{GL}
\DeclareMathOperator{\SO}{SO}
\DeclareMathOperator{\Sp}{Sp}
\DeclareMathOperator{\HH}{H}
\DeclareMathOperator{\Symp}{Symp}

% Representation theory

\DeclareMathOperator{\Ad}{Ad}

% Algebra

\DeclareMathOperator{\End}{End}
\DeclareMathOperator{\Aut}{Aut}
\DeclareMathOperator{\Hom}{Hom}
\DeclareMathOperator{\sHom}{\mathscr{H}\!om}
\DeclareMathOperator{\sEnd}{\mathscr{E}\!nd}
\DeclareMathOperator{\id}{id}
\DeclareMathOperator{\irr}{irr}
\DeclareMathOperator{\Diff}{Diff}
\DeclareMathOperator{\gr}{gr}
\DeclareMathOperator{\im}{im}
\DeclareMathOperator{\ad}{ad}
\DeclareMathOperator{\rk}{rk}
\DeclareMathOperator{\Spec}{Spec}
\DeclareMathOperator{\Specm}{Specm}
\DeclareMathOperator{\Stab}{Stab}
\DeclareMathOperator{\Sym}{Sym}

% Category theory

\DeclareMathOperator*{\colim}{colim}

\makeatletter
\renewcommand\d[1]{\mspace{6mu}\mathrm{d}#1\@ifnextchar\d{\mspace{-3mu}}{}}
\makeatother



\title{Notes on symplectic geometry}
\author{Nilay Kumar}
\date{Fall 2015}

\begin{document}

\maketitle
\tableofcontents

\section{Week 1}

\subsection{The cotangent bundle}

\begin{definition}
Let $X$ be a smooth $n$-manifold and $\pi:M=T^*X\to X$ be its cotangent bundle. We define the
\textbf{canonical one-form} $\theta\in\Omega^1(M)$ as follows. For any $p=(x,\xi)\in M$,
set
\begin{equation*}
    \theta_p(v)=\xi( d_x\pi(v)).
\end{equation*}
\end{definition}

The one-form $\theta$ is canonical (or tautological) in the sense that its value at a point
is simply given by the covector determined by that point. More precisely, we have the following
characterization.

\begin{proposition}
    The canonical one-form $\theta$ is the (unique) one-form such that for every $\lambda\in\Omega^1(X)$,
    $\lambda^*\theta=\lambda$.
    \label{prop:can}
\end{proposition}
\begin{proof}
    We compute, for $v\in T_pX$,
    \begin{align*}
        (\lambda^*\theta)_p(v) &= \theta_{\lambda(p)}(d_p\lambda(v)) \\
        &= \lambda_p(d_p(\pi\circ\lambda)(v)) \\
        &= \lambda_p(v),
    \end{align*}
    where we have used the fact that $\lambda$ is a section of $\pi$, i.e. $\pi\circ\lambda=\id_X$.
    Uniqueness is easily checked.
\end{proof}

\begin{definition}
    The \textbf{canonical symplectic form} $\omega\in\Omega^2(M)$ is now defined to be the exterior
    derivative
    \begin{equation*}
        \omega = - d\theta,
    \end{equation*}
    of the canonical one-form. To be symplectic, $\omega$ must be closed and nondegenerate. That it
    is closed is obvious.
\end{definition}

\begin{proposition}
    The form $\omega\in\Omega^2(M)$ is nondegenerate and thus defines a symplectic structure on
    $M=T^*X$.\footnote{Is there a coordinate invariant proof?}
\end{proposition}
\begin{proof}
    For $\omega$ to be non-degenerate, it must be nondegenerate at each point $p\in M$. Given
    coordinates $p=(x,\xi)=(x^1,\ldots,x^n,\xi_1,\ldots,\xi_n)$ in a neighborhood of $p$, we can
    compute
    \begin{align*}
        \theta_{(x,\xi)}\left( v^i\frac{\partial}{\partial x^i} + \nu^i\frac{\partial}{\partial\xi^i} \right)
        &= \xi\left( v^i\frac{\partial}{\partial x^i} \right) \\
        &= \xi_iv^i
    \end{align*}
    and hence
    \begin{equation*}
        \theta = \xi_i dx^i.
    \end{equation*}
    Taking an exterior derivative, we find that
    \begin{align*}
        \omega &= - d\theta \\
        &= dx^i\wedge d\xi_i.
    \end{align*}
    Fix $v\in T_pM$ and suppose that $\iota_v\omega_p=0$, i.e. $\omega_p(v,w)=0$
    for all $w\in T_pM$. In coordinates, this implies that
    \begin{align*}
        \iota_{v^j\frac{\partial}{\partial x^j}+\nu^j\frac{\partial}{\partial\xi^j}}(dx^i\wedge d\xi_i)
        &= v^id\xi_i-\nu^idx^i\\
        &=0,
    \end{align*}
    and hence that $v^i=\nu^i=0$, i.e. $v=0$. We conclude that $\omega_p$ is nondegenerate at each $p\in M$.
\end{proof}

\begin{remark}
    Note that a 2-form $\omega$ on a manifold $M$ is nondegenerate if and only if $\omega^n$ is
    nowhere vanishing. Fix $p\in M$ and consider the vector space $(T_pM, \omega_p)$. If
    $\omega_p$ is nondegenerate, we can find a symplectic basis for $T_pM$, and so $\omega_p^n$
    evaluated on $(u_1,\ldots,u_n,v_1,\ldots ,v_n)$ is nonzero, whence $\omega_p^n$ is not zero
    on $V$. On the other hand, suppose $\omega_p$ is degenerate, i.e. there is a $v\neq0$ such
    that $\omega_p(v,w)=0$ for all $w\in V$. Choosing a basis $v_1,\ldots,v_{2n}$ for $V$ such
    that $v_1=v$, we find that $\omega_p(v_1,\ldots,v_{2n})=0$ and hence $\omega_p=0$ on $V$.

    We conclude that every symplectic manifold is orientable.
    \label{rem:nondeg}
\end{remark}

It is easy to see that $\omega$ provides an isomorphism $\iota:T_xX\xrightarrow{\sim} T_x^*X$
between tangent and cotangent spaces at each point $x\in X$: since $\omega_x$ is nondegenerate, the linear
map $\iota:v\mapsto \omega_x(v,-)$ is injective and hence bijective. In fact, we can say more.

\begin{proposition}
    The metric $\omega$ induces an isomorphism of vector bundles $\iota:TX\xrightarrow{\sim}T^*X=M$.
\end{proposition}
\begin{proof}
    Recall that an isomorphism in the category of smooth vector bundles is a smooth
    bijection\footnote{Existence of a smooth inverse is automatic (reference?).}
    $\iota$ such that the diagram
    \begin{equation*}
        \begin{tikzcd}
            TX\ar{rr}{\iota}\ar[swap]{rd}{\pi_1} & & T^*X\ar{ld}{\pi_2}\\
            & X
        \end{tikzcd}
    \end{equation*}
    commutes and for each $x\in X$, the restriction $\iota_x:T_xX\to T^*_xX$ is linear.
    The map $\iota:TX\to T^*X$ taking $(x,v)\mapsto (x,\omega(v,-))$ fits into the diagram above
    and is bijective and fiberwise linear. Moreover, $\iota$ is a smooth map, as is seen by
    its coordinate description computed above.
\end{proof}

\begin{definition}
    A \textbf{Hamiltonian} is a smooth function $H:M=T^*X\to\R$.
    we define the \textbf{Hamiltonian vector field} $v_H$ associated to $H$ to be the vector
    field on $M$ satisfying
    \begin{equation*}
        \iota_{v_H}\omega = dH.
    \end{equation*}
    The (local) flow $F:(-\varepsilon,\varepsilon)\times M\to M$ determined by $v_H$ is
    called the \textbf{Hamiltonian flow}.\footnote{Is this a global flow? Does it depend on $X$?}
\end{definition}

Note that an integral curve $\gamma_{v_H}:(-\varepsilon,\varepsilon)\to M$ of $v_H$ can be thought
of as the trajectory of a physical state in phase space. Indeed, Hamilton's equations are given
\begin{align*}
    \frac{\partial x^i}{\partial t} &= \frac{\partial H}{\partial \xi_i} \\
    \frac{\partial \xi_i}{\partial t} &= - \frac{\partial H}{\partial x^i},
\end{align*}
which is precisely the condition that $\gamma_{v_H}'(t)=(v_H)_{\gamma(t)}$. Moreover, $H$ is
constant along the Hamiltonian flow, as
\begin{equation*}
    dH(v_H) = (\iota_{v_H}\omega)(v_H)=\omega(v_H,v_H) = 0,
\end{equation*}
i.e. $v_H$ is perpendicular to the level sets of $H$. In a physical system, where $H$ is the
energy functional on phase space, this phenomenon is the law of conservation of energy.

\begin{proposition}
    The Hamiltonian flow is a symplectomorphism, i.e. $F_t^*\omega=\omega$.\footnote{Is there
    a better proof?}
\end{proposition}
\begin{proof}
    We use the following trick:
    \begin{equation*}
        \int_0^t \frac{d}{dt}F_t^*\omega\; dt = F_t^*\omega-\omega
    \end{equation*}
    since $F_0=\id_M$, and hence $F_t$ is a symplectomorphism if and only if the integrand
    is zero. But
    \begin{align*}
        \frac{d}{dt}F_t^*\omega &= \frac{d}{ds}\bigg|_{s=0}F_{t+s}^*\omega=F_t^*\frac{d}{ds}\bigg|_{s=0}F_s^*\omega \\
        &= F_t^*\mathcal{L}_{v_H}\omega,
    \end{align*}
    and Cartan's magic formula,
    \begin{equation*}
        \mathcal{L}_{v_H}\omega = d\iota_{v_H}\omega + \iota_{v_H}d\omega,
    \end{equation*}
    tells us that $\mathcal{L}_{v_H}\omega=0$ since $\iota_{v_H}\omega=dH$ is closed, as is $\omega$.
\end{proof}

\begin{corollary}[Liouville's Theorem]
    The volume form $\omega^n$ on $M=T^*X$ is preserved by the Hamiltonian flow.
\end{corollary}


\subsection{Geodesic flow as Hamiltonian flow}

We wish to discuss geodesics and geodesic flow. For this, we need the concept
of connections and covariant derivatives.\footnote{Reference do Carmo.}
\begin{definition}
    A \textbf{connection} on a vector bundle $E\to X$ is an $\R$-linear map
    $\nabla:\Gamma(X,E)\to\Gamma(X,E\otimes T^*X)$ such that the Leibniz rule
    \begin{equation*}
        \nabla(f\sigma) = (\nabla \sigma)f + \sigma\otimes df,
    \end{equation*}
    for all $f\in C^\infty(X)$ and $\sigma\in\Gamma(X,E)$.
\end{definition}

\begin{theorem}
    Given a Riemannian manifold $(X,g)$, there exists a unique connection on $\pi:TX\to X$,
    known as the \textbf{Levi-Civita connection}, satisfying
    \begin{enumerate}[(i)]
        \item symmetry:
            \begin{equation*}
                \nabla_XY-\nabla_YX-[X,Y]=0,
            \end{equation*}
            for $X,Y\in\Gamma(X,TX)$;
        \item compatibility with $g$:
            \begin{equation*}
                Xg(Y,Z)-g(\nabla_XY,Z)-g(Y,\nabla_XZ)=0,
            \end{equation*}
            for $X,Y,Z\in\Gamma(X,TX)$.
    \end{enumerate}
\end{theorem}

\begin{definition}
    Let $v$ be a vector field on $(X,g)$; we define the \textbf{covariant derivative} of $v$
    along a smooth curve $c:I\to X$ to be the vector field
    \begin{align*}
        \frac{Dv}{dt} = \nabla_{dc/dt}v,
    \end{align*}
    where $\nabla$ is the Levi-Civita connection. Explicitly, if we write $v=v^i\partial/\partial x^i$
    and $c(t)=(c_1(t),\ldots,c_n(t))$,
    \begin{equation*}
        \frac{Dv}{dt} = \sum_i \frac{dv^i}{dt}\frac{\partial}{\partial x^i} + \sum_{ijk}\frac{dc_i}{dt}v^i\Gamma_{ij}^k\frac{\partial}{\partial x^k}.
    \end{equation*}
    Here $\Gamma_{ij}^k$ are the Christoffel symbols of $\nabla$, determined by
    \begin{equation*}
        \nabla_{\partial/\partial x^i}\frac{\partial}{\partial x^j}=\sum_{ijk}\Gamma^k_{ij}\frac{\partial}{\partial x^k}.
    \end{equation*}
    We say that $c$ is \textbf{geodesic} at some $t\in I$ if $D/dt(dc/dt)=0$ at $t$,
    and that $c$ is geodesic if it is geodesic at all $t\in I$. In coordinates, the condition
    for $c$ to be geodesic is given by a system of second-order differential equations:
    \begin{equation*}
        \frac{d^2 c^i}{dt^2} + \sum_{jk}\Gamma_{jk}^i\frac{dc^j}{dt}\frac{dc^k}{dt}=0,
    \end{equation*}
    for $i=1,\ldots, n.$
\end{definition}

For the rest of the section, assume $(X,g)$ is Riemannian and we fix the Hamiltonian $H:M=T^*X\to\R$ as
\begin{equation*}
    H(x,\xi)=\frac{1}{2}\big|\xi_x\big|^2_g,
\end{equation*}
i.e. consisting of only a kinetic term. Here we are implicitly using the nondegeneracy of $g$
to associate $\xi_x$ with its corresponding vector (or, equivalently, using $g^{-1}$).

\begin{proposition}
    The Hamiltonian flow on $M=T^*X$ is dual to the geodesic flow on $TX$. In other words, the integral
    curves of the Hamiltonian vector field $v_H$ associated to the Hamiltonian above project to geodesics
    of $g$ on $X$.\footnote{Is there a coordinate-free proof? See Paternain's book.}
\end{proposition}
\begin{proof}
    It suffices to show, in coordinates, that Hamilton's equations (i.e. the condition for
    being on the integral curve) yield the geodesic equations above after the necessary
    dualization. Note first that in coordinates the Hamiltonian becomes
    \begin{equation*}
        H(x,\xi) = \frac{1}{2}g^{ij}\xi_i\xi_j.
    \end{equation*}
    For convenience we will denote the components of an integral curve as $x^i(t)$. Hamilton's equations yield
    \begin{align*}
        \frac{dx^i}{dt} &= \frac{\partial}{\partial\xi_i}\left( \frac{1}{2}g^{jk}\xi_j\xi_k \right)\\
        &= \frac{1}{2}g^{jk}\delta_{ij}\xi_k + \frac{1}{2}g^{jk}\xi_j\delta_{ik}\\
        &= g^{ij}\xi_j\\
        \frac{d\xi_i}{dt} &= -\frac{\partial}{\partial x^i}\left( \frac{1}{2}g^{jk}\xi_j\xi_k \right)\\
        &=-\frac{1}{2}\frac{\partial g^{jk}}{\partial x^i}\xi_j\xi_k.
    \end{align*}
    Differentiating the first equation with respect to $t$ and using both of Hamilton's equations yields
    \begin{align*}
        \frac{d^2x^i}{dt^2} &= \frac{\partial g^{ij}}{\partial x^k}\frac{dx^k}{dt}\xi_j+g^{im}\frac{d\xi_m}{dt}\\
        &= g^{kl}\left(\frac{\partial}{\partial x^k}g^{ij}\right)\xi_l\xi_j-\frac{1}{2}g^{im}\left(\frac{\partial}{\partial x^m}g^{nr}\right)\xi_n\xi_r.
    \end{align*}
    Next, differentiating the identity $g^{ij}g_{jk}=\delta^i_k$, it easy to see that
    \begin{align*}
        \frac{\partial}{\partial x^i}g^{kl}=-g^{la}g^{kb}\frac{\partial}{\partial x^i}g_{ab}.
    \end{align*}
    Using this, contracting indices, and using the first Hamilton's equation to dualize $\xi$'s into $dx/dt$'s,
    we find
    \begin{align*}
        \frac{d^2x^i}{dt^2} &= -g^{ib}\left( \frac{\partial}{\partial x^k}g_{lb} \right)\frac{dx^k}{dt}\frac{dx^l}{dt}
        +\frac{1}{2}g^{im}\left( \frac{\partial}{\partial x^m}g_{ts} \right)\frac{dx^s}{dt}\frac{dx^t}{dt}\\
        &= -\frac{1}{2}g^{ib}\left( \frac{\partial}{\partial x^k}g_{lb} \right)\frac{dx^k}{dt}\frac{dx^l}{dt}
        -\frac{1}{2}g^{ib}\left( \frac{\partial}{\partial x^l}g_{kb} \right)\frac{dx^k}{dt}\frac{dx^l}{dt}
        \\&+\frac{1}{2}g^{im}\left( \frac{\partial}{\partial x^m}g_{ts} \right)\frac{dx^s}{dt}\frac{dx^t}{dt}\\
        &= -\Gamma_{kl}^i\frac{dx^k}{dt}\frac{dx^l}{dt},
    \end{align*}
    as desired.
\end{proof}


\newpage
\section{Week 2}

\subsection{Darboux's theorem}

\begin{theorem}[Darboux]
    Let $(M,\omega)$ be a symplectic $2n$-manifold. Then $M$ is locally symplectomorphic to
    $(\R^{2n},\omega_{\R^{2n}})$.
\end{theorem}

We prove Darboux's theorem using the following stronger statement.

\begin{theorem}
    Let $M$ be a $2n$-dimensional manifold and $Q\subset M$ be a compact submanifold. Suppose that
    $\omega_1,\omega_2\in\Omega^2(M)$ are closed 2-forms such that at each point $q$ of $Q$ the forms
    $\omega_0$ and $\omega_1$ are equal and nondegenerate on $T_qM$. Then there exist neighborhoods
    $N_0$ and $N_1$ of $Q$ and a diffeomorphism $\psi:N_0\to N_1$ such that $\psi|_Q=\id_Q$ and
    $\psi^*\omega_1=\omega_0$.
    \label{thm:moser}
\end{theorem}
\begin{proof}
    Consider the family of closed two-forms
    \begin{equation*}
        \omega_t = \omega_0 + t(\omega_1-\omega_0)
    \end{equation*}
    on $M$ for $t\in[0,1]$. Note that $\omega_t|_Q=\omega_0|_Q$ is nondegenerate and hence there exists
    an open neighborhood $N_0$ of $Q$ such that $\omega_t|_{N_0}$ is nondegenerate.\footnote{Why?} Suppose,
    for now, that there is a one-form $\sigma\in\Omega^1(N_0)$ (possibly shrinking $N_0$), such that
    $\sigma|_{T_QM}=0$ and $d\sigma=\omega_1-\omega_0$ on $N_0$. Then
    \begin{equation*}
        \omega_t = \omega_0+td\sigma
    \end{equation*}
    and we obtain by nondegeneracy a smooth vector field $X_t$ on $N_0$ characterized by
    \begin{equation*}
        \iota_{X_t}\omega_t = -\sigma.
    \end{equation*}
    The condition $\sigma|_{T_QM}=0$ implies, again by nondegeneracy of $\omega_t$, that $X_t|_Q=0$.
    Now consider the initial value problem for the flow $\psi_t$ of $X_t$,
    \begin{align*}
        \frac{d}{dt}\psi_t &= X_t \circ \psi_t\\
        \psi_0 &= \id.
    \end{align*}
    This differential equation can be solved uniquely for $t\in[0,1]$ on some open neighborhood of
    $Q$ contained in $N_0$, call it again $N_0$.\footnote{Why?} Note that $\psi_t|_Q=\id_Q$ since $X_t|_Q=0$.
    We compute now that
    \begin{align*}
        \frac{d}{dt}\psi_t^*\omega_t &= \psi_t^*\left( \frac{d}{dt}\omega_t + \mathcal{L}_{X_t}\omega_t \right)\\
        &= \psi_t^*\left( d\sigma + d\iota_{X_t}\omega_t\right)\\
        &= 0.
    \end{align*}
    Hence $\psi_1^*\omega_1=\psi_0^*\omega_0=\omega_0$. Thus the desired diffeomorphism is
    $\psi_1$ and the desired neighborhoods are $N_0$ and $N_1$.
    The above argument is known as \textbf{Moser's trick}, and is extremely useful in symplectic
    geometry.
    
    It remains to construct a smooth one-form $\sigma$ satisfying $\sigma|_{T_QM}=0$ and $d\sigma=\omega_1-\omega_0$.
    If $Q$ were a point (or more generally, diffeomorphic to a star-shaped subset of Euclidean space),
    we could simply use the Poincar\'e lemma; in general, however the construction is as follows.
    Fix any Riemannian metric on $M$ and consider the restriction of the exponential map
    $\exp: TM\to M$ to a neighbhorhood $U_\varepsilon$ of the zero section of the normal bundle
    $TQ^\perp\to M$:
    \begin{equation*}
        U_\varepsilon = \{(q,v)\in TM \mid q\in Q, v\in T_qQ^\perp, |v|<\varepsilon\}.
    \end{equation*}
    Recall that $\exp$ becomes a diffeomorphism for $\varepsilon$ sufficiently small, so we choose
    $\varepsilon$ such that $N_0=\exp(U_\varepsilon)$ is contained in the neighborhood of $Q$ above
    on which $\omega_t$ is nondegenerate. Define now a family of maps $\phi_t:N_0\to N_0$
    for $t\in[0,1]$ by
    \begin{equation*}
        \phi_t(\exp(q,v)) = \exp(q,tv).
    \end{equation*}
    Note that $\phi_t$ is a diffeomorphism onto its image for $t\neq 0$. Moreover, $\phi_t|_Q=\id_Q$,
    $\phi_0(N_0)$, and $\phi_1 = \id_{N_0}$. If we now write $\tau = \omega_1-\omega_0$, we find that
    \begin{align*}
        \phi_0^*\tau &= 0\\
        \phi_1^*\tau &= \tau,
    \end{align*}
    since $\tau=0$ on $T_QM$. Now, for $t\in(0,1]$, we define a family of vector fields,
    \begin{equation*}
        Y_t = \left( \frac{d}{dt}\phi_t \right)\circ\phi_t^{-1}.
    \end{equation*}
    Then for any $\delta>0$,
    \begin{align*}
        \phi_1^*\tau - \phi_\delta^*\tau &= \int_\delta^1 \frac{d}{dt}\phi_t^*\tau dt=\int_\delta\phi_t^*\mathcal{L}_{Y_t}\tau dt\\
        &= \int_\delta^1 \phi_t^*(d\iota_{Y_t}\tau)dt\\
        &= d\int_\delta^1 \phi_t^*(\iota_{Y_t}\tau)dt
    \end{align*}
    Clearly $\phi_1^*\tau-\phi_\delta^*\tau=\tau-\phi_\delta^*\tau$ approaches $\tau$ as $\delta\to0^+$,
    so we find that
    \begin{equation*}
        \tau=d\int_0^1\phi_t^*(\iota_{Y_t}\tau)dt.
    \end{equation*}
    Defining
    \begin{equation*}
        \sigma = \int_0^1\phi_t^*(\iota_{Y_t}\tau) dt,
    \end{equation*}
    we find that $\tau=\omega_1-\omega_0=d\sigma$ and $\sigma|_{T_QM}=0$ because $\phi_t|_Q=\id_Q$ and
    $\tau=0$ on $Q$, forcing the integrand to vanish on $T_QM$. Hence $\sigma$ is the one-form required
    above for Moser's trick, and we are done.\footnote{Why is $\sigma$ smooth?}
    \label{thm:moser}
\end{proof}

The proof of Darboux's theorem is now straightforward: we choose a coordinate chart $\phi$ so that
$\phi^*\omega$ is equal to the standard form on a subset of $\R^{2n}$ at a single point, and then
apply Moser's theorem with $Q$ equal to the chosen point.

\begin{proof}[Proof of Darboux's theorem]
    Let $q\in M$ and fix a symplectic basis $\{u_i,v_i\}$ for the symplectic vector space $(T_qM,\omega_q)$.
    Fix any Riemannian metric on $M$ and pick an open $U\ni 0$ small enough such that $\exp$ restricted to
    $U\subset T_qM$ is a diffeomorphism and hence a chart $(x^i,y_i)=\exp:U\subset\R^{2n}\to M$
    ($i=1,\ldots,n$) such that $x^i(p)=y_i(p)=0$. Now we can compute, for example,
    \begin{align*}
        \exp^*\omega_p\left( \frac{\partial}{\partial x^j},\frac{\partial}{\partial y^k} \right) &=\omega_p\left(\exp_*\frac{\partial}{\partial x^j},\exp_*\frac{\partial}{\partial y^k}\right)\\
        &=\omega_p\left( u_j,v_k \right)=\delta_{jk},
    \end{align*}
    to check that $\exp^*\omega_p=(\omega_0)_0$ where $\omega_0$ is the standard form on $T_0U$.
    Here we have used the fact that $\exp_*=\id$ at $0\in U$.
    Applying Theorem \ref{thm:moser} to $U$ with $Q=0\in U$, we obtain a diffeomorphism $\psi$ of
    (some possibly smaller) $U$ such that $\psi^*\exp^*\omega=\omega_0$ on $U$. But now
    $\exp\circ\psi$ provides a symplectomorphism in a neighborhood of $q$ to a neighborhood of $\R^{2n}$
    pulling $\omega$ back to the standard form $\omega_0$.
\end{proof}

\newpage

\section{Week 3}
\subsection{Submanifolds of symplectic manifolds}

\begin{definition}
    Let $(V,\omega)$ be a symplectic vector space. We define the \textbf{symplectic complement}
    $U^{\omega}$ of a subspace $U\subset V$ as
    \begin{equation*}
        U^{\omega} = \{v\in V \mid \omega(v,u)=0\text{ for all }u\in U\}.
    \end{equation*}
\end{definition}

\begin{lemma}
    For any subspace $U\subset V$, $U^{\omega\omega}=U$ and
    \begin{equation*}
        \dim U + \dim U^{\omega} = \dim V.
    \end{equation*}
    \label{lem:dim}
\end{lemma}
\begin{proof}
    Nondegeneracy of $\omega$ yields an isomorphism $\iota_{\omega}:V\to V^*$ which identifies
    $U^\omega$ with $U^{\perp}\equiv\{\nu\in V^*\mid \nu(u)=0\text{ for all }u\in U\}$. The result
    now follows from the fact that $\dim U + \dim U^{\perp} = \dim V$.
\end{proof}

\begin{definition}
    Let $(M,\omega)$ be a symplectic manifold. A submanifold $Q\subset M$ is called
    \textbf{symplectic, isotropic, coisotropic, or Lagrangian} if for each $q\in Q$, the linear
    subspace $T_qQ\equiv V_q$ of $(T_qM,\omega_q)$ is
    \begin{enumerate}[(a)]
        \item symplectic: $V_q\cap V_q^{\omega_q}=0$,
        \item isotropic: $V_q\subset V_q^{\omega_q}$,
        \item coisotropic: $V_q^{\omega_q}\subset V_q$,
        \item Lagrangian: $V_q=V_q^{\omega_q}$,
    \end{enumerate}
    respectively.
\end{definition}

\begin{remark}
    Note that $Q\subset M$ is Lagrangian if and only if the restriction of $\omega$ to $Q$
    is zero and $\dim Q=\dim M/2$.
\end{remark}

\begin{example}
    Let $X$ be any manifold, and $(M=T^*X,\omega)$ be its cotangent bundle with the usual symplectic
    structure. Recall that $\omega=-d\theta$, where $\theta_\xi(v)=\xi(d_x\pi(v))$.\footnote{Can we do
    this coordinate-invariantly?}
    In coordinates, if $(x^i,\xi^i)$ are coordinates for $M$, we can write 
    $\omega=dx^i\wedge d\xi^i$.

    It is then easy to see that the fibre $T_x^*X\subset M$ is Lagrangian, as 
    \begin{align*}
        0 &= (dx^i\wedge d\xi^i)\left( a_j\frac{\partial}{\partial\xi^j}, b_k\frac{\partial}{\partial\xi^k}+c_l\frac{\partial}{\partial x^l} \right)\\
        &= (dx^i\wedge d\xi^i)\left( a_j\frac{\partial}{\partial\xi^j},c_l\frac{\partial}{\partial x^l} \right)\\
        &= a_ic_i,
    \end{align*}
    forces $c_i=0$.

    Similarly, the zero section $\Gamma_0\subset M$ is Lagrangian, as
    \begin{align*}
        0 &= (dx^i\wedge d\xi^i)\left( a_j\frac{\partial}{\partial x^j}, b_k\frac{\partial}{\partial\xi^k}+c_l\frac{\partial}{\partial x^l} \right)\\
        &= (dx^i\wedge d\xi^i)\left( a_j\frac{\partial}{\partial x^j},b_k\frac{\partial}{\partial \xi^k} \right)\\
        &= a_ib_i,
    \end{align*}
    forces $b_i=0$.

    More generally, given a submanifold $Q\subset L$, the annihilator
    \begin{equation*}
        TQ^{\perp} = \{(q,\nu)\in T^*L \mid q\in Q, \nu|_{T_qQ}=0\}
    \end{equation*}
    is Lagrangian.
\end{example}

\begin{example}
    Let $(M,\omega)$ be a symplectic manifold. The product $M\times M$ can be given
    a symplectic structure $\omega'=\alpha\pi_1^*\omega+\beta\pi_2^*\omega$ for $\alpha,\beta\in\R$.
    Consider in particular the case of $\alpha=1,\beta=-1$. Then it is clear that $M\times\{m\}$
    and $\{m\}\times M$ are symplectic submanifolds. Moreover, the diagonal $\Delta\subset M\times M$
    is Lagrangian, as
    \begin{align*}
        0&=\omega'\left( (u,u), (v,w) \right)\\
        &=\omega(u,v)-\omega(u,w)\\
        &=\omega(u,v-w)
    \end{align*}
    and hence $v=w$, as desired.
\end{example}

\begin{example}
    Let $S\subset (M,\omega)$ be a codimension 1 submanifold. Then $S$ is coisotropic.
    Indeed, fix $s\in S$, and note that $T_sS\subset T_sM$ is codimension one. By Lemma
    \ref{lem:dim}, $T_sS^{\omega_s}$ is a one-dimensional subspace. Pick any vector
    $v\in T_sS^{\omega_s}$; $v$ spans the entire symplectic complement, and hence if $v$
    is not in $T_sS^{\omega_s}$, $T_sS\cap T_sS^{\omega_s}=0$ and $T_sS$ is symplectic
    and thus even-dimensional. This is a contradiction, and hence $T_sS$ must be
    coisotropic.
\end{example}

\begin{proposition}
    The graph $\Gamma_\sigma\subset T^*X$ of a one-form is Lagrangian if and only if $\sigma$
    is closed.
\end{proposition}
\begin{proof}
    Note that $\Gamma_\sigma$ is defined to be the image of the embedding $\sigma:X\to T^*X$.
    Then $\dim\Gamma_\sigma=n$, so it remains to show that $\omega$
    restricts to zero on $\Gamma_\sigma$ if and only if $\sigma$ is closed. Using Proposition
    \ref{prop:can}, we compute
    \begin{align*}
        d\sigma = d\sigma^*\theta=\sigma^*d\theta=-\sigma^*\omega,
    \end{align*}
    which yields the desired statement, as $\sigma^*\omega=0$ on $X$ if and only if $\omega=0$
    on $\Gamma_\sigma$, by virtue of $\sigma$ being an embedding.
\end{proof}

\subsection{Some neighborhood theorems}

\begin{definition}
    Let $V$ be a vector space. A \textbf{complex structure} on $V$ is an automorphism
    $J:V\to V$ such that $J^2=-\id_V$. Note that a complex structre gives $V$ the structure
    of a complex vector space ($i$ acts by $J$). We denote the set of all complex
    structures on $V$ by $\mathcal{J}(V)$. Now suppose $(V,\omega)$ is a symplectic
    vector space. We say that a complex structure $J$ is \textbf{compatible} with
    $\omega$ if
    \begin{equation*}
        \omega(Jv, Jw)=\omega(v,w)
    \end{equation*}
    for all $v,w\in V$, and
    \begin{equation*}
        \omega(v,Jv)>0
    \end{equation*}
    for all nonzero $v\in V$. We denote the set of all compatible complex structures
    on $(V,\omega)$ by $\mathcal{J}(V,\omega)$.
\end{definition}

\begin{lemma}
    Let $J\in\mathcal{J}(V,\omega)$ be a compatible complex structure on $(V,\omega)$.
    Then 
    \begin{equation*}
        g_J(v,w) = \omega(v,Jw)
    \end{equation*}
    defines an inner product on $V$.
\end{lemma}

\begin{definition}
    A \textbf{symplectic vector bundle} $(E,\omega)$ over $X$ is a real vector bundle
    $\pi:E\to X$ together with a smooth symplectic bilinear form $\omega\in\Gamma(X,E^*\wedge E^*)$,
    i.e. a symplectic bilinear form on each $E_x$ that varies smoothly with $x$.
    A \textbf{complex structure} on $\pi:E\to M$ is a bundle automorphism $J:E\to E$
    such that $J^2=-\id_{E}$. We say $J$ is \textbf{compatible} with $\omega$ if the
    induced complex structure on $E_x$ is compatible with $\omega_x$ for all $x\in X$.
    By the above lemma, we obtain a symmetric, positive-definite bilinear form
    $g_J\in\Gamma(X,\Sym^2 E^*)$, and we call the triple $(E, \omega, g_J)$ a
    \textbf{Hermitian structure} on $E$.
\end{definition}

\begin{theorem}
    Every symplectic vector bundle $(E,\omega)$ over a manifold $X$ admits a
    Hermitian structure.
\end{theorem}

With these definitions out of the way, we present a number of theorems characterizing
neighbhorhoods of special submanifolds of symplectic manifolds. 

\begin{theorem}[Symplectic neighborhood theorem]
    Let $(M_0,\omega_0),(M_1,\omega_1)$ be symplectic manifolds with compact symplectic
    submanifolds $Q_0,Q_1$ respectively. Suppose there is an isomorphism $\Phi:TQ_0^\omega\to TQ_1^\omega$
    of symplectic normal bundles covering a symplectomorphism $\phi:(Q_0,\omega_0)\to(Q_1,\omega_1)$.
    Then $\phi$ extends to a symplectomorphism $\psi:(N(Q_0),\omega_0)\to(N(Q_1),\omega_1)$
    such that $d\psi$ induces the map $\Phi$ on $TQ_0^{\omega}$.
\end{theorem}
\begin{proof}
    Let $\exp_0,\exp_1$ be diffeomorphisms mapping neighborhoods of the zero section
    in the normal bundle to neighborhoods of $Q_0,Q_1$ in $X$, respectively. Then
    we obtain 
    \begin{equation*}
        \phi' = \exp_1\circ\Phi\circ\exp_0^{-1},
    \end{equation*}
    a diffeomorphism between these neighborhoods of $Q_0$ and $Q_1$. Now $\phi'^*\omega_1$
    and $\omega_0$ are two symplectic forms on $M_0$ whose restrictions to $Q_0$ agree.
    Now $\phi'$ extends to the desired $\psi$ by Theorem \ref{thm:moser}.
\end{proof}

\begin{theorem}[Lagrangian neighborhood theorem]
    Let $(M,\omega)$ be a symplectic manifold and let $L\subset M$ be a compact
    Lagrangian submanifold. Then there exists a neighborhood $N(\Gamma_0)\subset T^*L$
    of the zero section $\Gamma_0$, a neighborhood $U\subset M$ of $L$, and a
    diffeomorphism $\phi:N(\Gamma_0)\to U$ such that $\phi^*\omega=-d\theta$ and
    $\phi|_L=\id$, where $\theta$ is the canonical one-form on $T^*L$.
\end{theorem}
\begin{proof}
    
\end{proof}

\subsection{Contact manifolds}

Let $X$ be a differential manifold and $H\subset TX$ be a smooth hyperplane field,
i.e. a smooth subbundle of codimension one.
Then, locally on some open $U$, we can write $H=\ker\alpha$,
for $\alpha\in\Omega_1(U)$. In fact, if we assume that $H$ is \textbf{coorientable},
we can extend $U$ to all of $X$.\footnote{Why?} We will assume for what
follows that $H$ is coorientable.

\begin{definition}
    Let $X$ be a manifold of odd dimension $2n+1$. A \textbf{contact structure}
    on $X$ is a hyperplane field $H=\ker\alpha$ where the top-dimensional form
    $\alpha\wedge(d\alpha)^n$ is nowhere vanishing. We call $\alpha$ a
    \textbf{contact form}, and the pair $(X,H)$ a \textbf{contact manifold}.
\end{definition}

\begin{remark}
    Suppose we have $\alpha,\alpha'\in\Omega^1(X)$ such that $H=\ker\alpha=\ker\alpha'$.
    Then $\alpha$ is a contact form if and only if $\alpha'$ is. This is because
    the condition that $\alpha,\alpha'$ cut out $H$ requires $\alpha'=f\alpha$ for some
    nonzero $f:X\to\R$.
\end{remark}

\begin{remark}
    In the language of distributions, $H$ can be described as a codimension one
    distribution that is maximally non-integrable in the following sense. Recall
    that a distribution on $X$ is said to be integrable if every point $p$ of $X$ is
    contained in a integral manifold of $H$, i.e. in a nonempty immersed submanifold
    $N\subset X$ such that $T_pN=H_p$. The Frobenius theorem tells us that $H$ is
    integrable if and only if $H$ is involutive, i.e. $H$ is closed under the Lie
    bracket of local sections. Now, since
    \begin{equation*}
        d\alpha(X,Y) = X\alpha(Y)-Y\alpha(X)-\alpha[X,Y],
    \end{equation*}
    we find that $H$ is integrable if and only if $d\alpha=0$ on $H$. Thus asking for
    $d\alpha$ to be nondegenerate on $H$ forces the distribution to be ``as non-integrable
    as possible.''
    
    Indeed, we obtain the above definition of a contact structure by
    noting that $d\alpha$ is nondegenerate on $H$ if and only if $\alpha\wedge(d\alpha)^n$
    is nowhere vanishing, as follows. By remark \ref{rem:nondeg}, $d\alpha$ is nondegenerate
    on $H$ if and only if $(d\alpha)^n$ is nowhere vanishing, but this is simply
    equivalent to asking that $\alpha\wedge(d\alpha)^n$ be nowhere vanishing.
\end{remark}

Armed simply with the definition of a contact manifold, one might think that contact
geometry is somewhat obscure. We provide the following list of examples as evidence
that contact manifolds are actually quite common.

\begin{example}
    Let $X=\R^{2n+1}$ with coordinates $(x^1,\ldots,x^n,y^1,\ldots,y^n,z)$. The
    one-form
    \begin{equation*}
        \alpha = dz + x^i dy^i
    \end{equation*}
    is a contact form, as
    \begin{equation*}
        \alpha\wedge(d\alpha)^n = dz\wedge dx^1\wedge dy^1\wedge\cdots\wedge dx^n\wedge dy^n,
    \end{equation*}
    which is nowhere vanishing. We define the standard contact structure on $\R^{2n+1}$
    to be $H=\ker\alpha$.
\end{example}

For the next few examples the following lemma will be useful.
\begin{lemma}
    Let $(M,\omega)$ be a symplectic manifold of dimension $2n$.
    A vector field $Y$ on $M$ satisfying
    $\mathcal{L}_Y\omega=\omega$ is called a \textbf{Liouville vector field}. In
    this case, $\alpha=\iota_Y\omega$ is a contact form on any hypersurface $Q\subset M$
    transverse to $Y$ (i.e. at any point $p$, $T_pQ$ and $Y_p$ span $T_pM$).
\end{lemma}
\begin{proof}
    Cartan's magic formula in this case tells us that $\omega=d\iota_Y\omega$, and hence
    \begin{align*}
        \alpha\wedge (d\alpha)^{n-1} &= \iota_Y\omega\wedge\omega^{n-1}\\
        &= \iota_Y(\omega^n)/n.
    \end{align*}
    Now, since $\omega^n$ is a volume form on $M$, we find that $\alpha\wedge(d\alpha)^{n-1}$
    is a volume form when restricted to the tangent bundle of any hypersurface transverse
    to $Y$.
\end{proof}

\begin{example}
    Consider $M=\R^4$ with its usual symplectic form $\omega=dx^1\wedge dy^1+dx^2\wedge dy^2$.
    The vector field
    \begin{equation*}
        Y=\frac{1}{2}\left( x_1\frac{\partial}{\partial x^1}+y^1\frac{\partial}{\partial y^1}+x^2\frac{\partial}{\partial x^2}+y^2\frac{\partial}{\partial y^2} \right)
    \end{equation*}
    is clearly transverse to the sphere $S^3$ given by $(x^1)^2+(y^1)^2+(x^2)^2+(y^2)^2=1$.
    It is a straightforward computation to check that $Y$ is Liouville, using
    the identity
    \begin{equation*}
        (\mathcal{L}_Y\omega)(v,w) = \mathcal{L}_Y(\omega(v,w)) - \omega([Y,v],w) - \omega(v,[Y,w]).
    \end{equation*}
    We conclude, using the previous lemma, that $S^3$ is a contact manifold, with a contact
    structure $\ker\iota_Y\omega$.
\end{example}

\end{document}
