\documentclass{amsart}

\usepackage[colorlinks=true]{hyperref}
\usepackage{enumerate}
\usepackage{tikz-cd}
\usepackage{amssymb}

\theoremstyle{plain}
\newtheorem{theorem}{Theorem}
\newtheorem{lemma}[theorem]{Lemma}
\newtheorem{proposition}[theorem]{Proposition}
\newtheorem{corollary}[theorem]{Corollary}

\theoremstyle{definition}
\newtheorem{definition}[theorem]{Definition}
\newtheorem{example}[theorem]{Example}
\newtheorem{exercise}[theorem]{Exercise}

\theoremstyle{remark}
\newtheorem{remark}[theorem]{Remark}

% Fonts
\newcommand{\A}{\mathbb{A}}
\newcommand{\C}{\mathbb{C}}
\newcommand{\R}{\mathbb{R}}
\newcommand{\Q}{\mathbb{Q}}
\newcommand{\Z}{\mathbb{Z}}
\newcommand{\N}{\mathbb{N}}
\newcommand{\G}{\mathbb{G}}
\newcommand{\fr}{\mathfrak}

% Topology/geometry

\DeclareMathOperator{\Gr}{Gr}
\DeclareMathOperator{\Fl}{Fl}
\DeclareMathOperator{\PP}{\mathbb{P}}
\DeclareMathOperator{\Der}{Der}
\DeclareMathOperator{\Lie}{Lie}
\DeclareMathOperator{\SL}{SL}
\DeclareMathOperator{\GL}{GL}
\DeclareMathOperator{\SO}{SO}
\DeclareMathOperator{\Sp}{Sp}
\DeclareMathOperator{\HH}{H}
\DeclareMathOperator{\Symp}{Symp}

% Representation theory

\DeclareMathOperator{\Ad}{Ad}

% Algebra

\DeclareMathOperator{\End}{End}
\DeclareMathOperator{\Aut}{Aut}
\DeclareMathOperator{\Hom}{Hom}
\DeclareMathOperator{\sHom}{\mathscr{H}\!om}
\DeclareMathOperator{\sEnd}{\mathscr{E}\!nd}
\DeclareMathOperator{\id}{id}
\DeclareMathOperator{\irr}{irr}
\DeclareMathOperator{\Diff}{Diff}
\DeclareMathOperator{\gr}{gr}
\DeclareMathOperator{\im}{im}
\DeclareMathOperator{\ad}{ad}
\DeclareMathOperator{\rk}{rk}
\DeclareMathOperator{\Spec}{Spec}
\DeclareMathOperator{\Specm}{Specm}
\DeclareMathOperator{\Stab}{Stab}
\DeclareMathOperator{\Sym}{Sym}

% Category theory

\DeclareMathOperator*{\colim}{colim}

\makeatletter
\renewcommand\d[1]{\mspace{6mu}\mathrm{d}#1\@ifnextchar\d{\mspace{-3mu}}{}}
\makeatother



\title{Notes on symplectic geometry}
\author{Nilay Kumar}
\date{Fall 2015}

\begin{document}

\maketitle
\tableofcontents

\section{Week 1}

\subsection{The cotangent bundle}

\begin{definition}
Let $X$ be a smooth $n$-manifold and $\pi:M=T^*X\to X$ be its cotangent bundle. We define the
\textbf{canonical one-form} $\theta\in\Omega^1(M)$ as follows. For any $p=(x,\xi)\in M$,
set
\begin{equation*}
    \theta_p(v)=\xi( d_x\pi(v)).
\end{equation*}
\end{definition}

The one-form $\theta$ is canonical (or tautological) in the sense that its value at a point
is simply given by the covector determined by that point. More precisely, we have the following
characterization.

\begin{proposition}
    The canonical one-form $\theta$ is the (unique) one-form such that for every $\lambda\in\Omega^1(X)$,
    $\lambda^*\theta=\lambda$.
\end{proposition}
\begin{proof}
    We compute, for $v\in T_pX$,
    \begin{align*}
        (\lambda^*\theta)_p(v) &= \theta_{\lambda(p)}(d_p\lambda(v)) \\
        &= \lambda_p(d_p(\pi\circ\lambda)(v)) \\
        &= \lambda_p(v),
    \end{align*}
    where we have used the fact that $\lambda$ is a section of $\pi$, i.e. $\pi\circ\lambda=\id_X$.
    Uniqueness is easily checked.
\end{proof}

\begin{definition}
    The \textbf{canonical symplectic form} $\omega\in\Omega^2(M)$ is now defined to be the exterior
    derivative
    \begin{equation*}
        \omega = - d\theta,
    \end{equation*}
    of the canonical one-form. To be symplectic, $\omega$ must be closed and nondegenerate. That it
    is closed is obvious.
\end{definition}

\begin{proposition}
    The form $\omega\in\Omega^2(M)$ is nondegenerate and thus defines a symplectic structure on
    $M=T^*X$.\footnote{Is there a coordinate invariant proof?}
\end{proposition}
\begin{proof}
    For $\omega$ to be non-degenerate, it must be nondegenerate at each point $p\in M$. Given
    coordinates $p=(x,\xi)=(x^1,\ldots,x^n,\xi_1,\ldots,\xi_n)$ in a neighborhood of $p$, we can
    compute
    \begin{align*}
        \theta_{(x,\xi)}\left( v^i\frac{\partial}{\partial x^i} + \nu^i\frac{\partial}{\partial\xi^i} \right)
        &= \xi\left( v^i\frac{\partial}{\partial x^i} \right) \\
        &= \xi_iv^i
    \end{align*}
    and hence
    \begin{equation*}
        \theta = \xi_i dx^i.
    \end{equation*}
    Taking an exterior derivative, we find that
    \begin{align*}
        \omega &= - d\theta \\
        &= dx^i\wedge d\xi_i.
    \end{align*}
    Fix $v\in T_pM$ and suppose that $\iota_v\omega_p=0$, i.e. $\omega_p(v,w)=0$
    for all $w\in T_pM$. In coordinates, this implies that
    \begin{align*}
        \iota_{v^j\frac{\partial}{\partial x^j}+\nu^j\frac{\partial}{\partial\xi^j}}(dx^i\wedge d\xi_i)
        &= v^id\xi_i-\nu^idx^i\\
        &=0,
    \end{align*}
    and hence that $v^i=\nu^i=0$, i.e. $v=0$. We conclude that $\omega_p$ is nondegenerate at each $p\in M$.
\end{proof}



%Now suppose $(X,g)$ is a Riemannian manifold, i.e. $g$ is a symmetric positive-definite covariant
%two-tensor on $X$.

It is easy to see that $\omega$ provides an isomorphism $\iota:T_xX\xrightarrow{\sim} T_x^*X$
between tangent and cotangent spaces at each point $x\in X$: since $\omega_x$ is nondegenerate, the linear
map $\iota:v\mapsto \omega_x(v,-)$ is injective and hence bijective. In fact, we can say more.

\begin{proposition}
    The metric $\omega$ induces an isomorphism of vector bundles $\iota:TX\xrightarrow{\sim}T^*X=M$.
\end{proposition}
\begin{proof}
    Recall that an isomorphism in the category of smooth vector bundles is a smooth
    bijection\footnote{Existence of a smooth inverse is automatic (reference?).}
    $\iota$ such that the diagram
    \begin{equation*}
        \begin{tikzcd}
            TX\ar{rr}{\iota}\ar[swap]{rd}{\pi_1} & & T^*X\ar{ld}{\pi_2}\\
            & X
        \end{tikzcd}
    \end{equation*}
    commutes and for each $x\in X$, the restriction $\iota_x:T_xX\to T^*_xX$ is linear.
    The map $\iota:TX\to T^*X$ taking $(x,v)\mapsto (x,\omega(v,-))$ fits into the diagram above
    and is bijective and fiberwise linear. Moreover, $\iota$ is a smooth map, as is seen by
    its coordinate description computed above.
\end{proof}

\begin{definition}
    A \textbf{Hamiltonian} is a smooth function $H:M=T^*X\to\R$.
    we define the \textbf{Hamiltonian vector field} $v_H$ associated to $H$ to be the vector
    field on $M$ satisfying
    \begin{equation*}
        \iota_{v_H}\omega = dH.
    \end{equation*}
    The (local) flow $F:(-\varepsilon,\varepsilon)\times M\to M$ determined by $v_H$ is
    called the \textbf{Hamiltonian flow}.\footnote{Is this a global flow? Does it depend on $X$?}
\end{definition}

Note that an integral curve $\gamma_{v_H}:(-\varepsilon,\varepsilon)\to M$ of $v_H$ can be thought
of as the trajectory of a physical state in phase space. Indeed, Hamilton's equations are given
\begin{align*}
    \frac{\partial x^i}{\partial t} &= \frac{\partial H}{\partial \xi_i} \\
    \frac{\partial \xi_i}{\partial t} &= - \frac{\partial H}{\partial x^i},
\end{align*}
which is precisely the condition that $\gamma_{v_H}'(t)=(v_H)_{\gamma(t)}$. Moreover, $H$ is
constant along the Hamiltonian flow, as
\begin{equation*}
    dH(v_H) = (\iota_{v_H}\omega)(v_H)=\omega(v_H,v_H) = 0,
\end{equation*}
i.e. $v_H$ is perpendicular to the level sets of $H$. In a physical system, where $H$ is the
energy functional on phase space, this phenomenon is the law of conservation of energy.

\begin{proposition}
    The Hamiltonian flow is a symplectomorphism, i.e. $F_t^*\omega=\omega$.\footnote{Is there
    a better proof?}
\end{proposition}
\begin{proof}
    We use the following trick:
    \begin{equation*}
        \int_0^t \frac{d}{dt}F_t^*\omega\; dt = F_t^*\omega-\omega
    \end{equation*}
    since $F_0=\id_M$, and hence $F_t$ is a symplectomorphism if and only if the integrand
    is zero. But
    \begin{align*}
        \frac{d}{dt}F_t^*\omega &= \frac{d}{ds}\bigg|_{s=0}F_{t+s}^*\omega=F_t^*\frac{d}{ds}\bigg|_{s=0}F_s^*\omega \\
        &= F_t^*\mathcal{L}_{v_H}\omega,
    \end{align*}
    and Cartan's magic formula,
    \begin{equation*}
        \mathcal{L}_{v_H}\omega = d\iota_{v_H}\omega + \iota_{v_H}d\omega,
    \end{equation*}
    tells us that $\mathcal{L}_{v_H}\omega=0$ since $\iota_{v_H}\omega=dH$ is closed, as is $\omega$.
\end{proof}

\begin{corollary}[Liouville's Theorem]
    The volume form $\omega^n$ on $M=T^*X$ is preserved by the Hamiltonian flow.
\end{corollary}


\subsection{Geodesic flow as Hamiltonian flow}

We wish to discuss geodesics and geodesic flow. For this, we need the concept
of connections and covariant derivatives.\footnote{Reference do Carmo.}
\begin{definition}
    A \textbf{connection} on a vector bundle $E\to X$ is an $\R$-linear map
    $\nabla:\Gamma(X,E)\to\Gamma(X,E\otimes T^*X)$ such that the Leibniz rule
    \begin{equation*}
        \nabla(f\sigma) = (\nabla \sigma)f + \sigma\otimes df,
    \end{equation*}
    for all $f\in C^\infty(X)$ and $\sigma\in\Gamma(X,E)$.
\end{definition}

\begin{theorem}
    Given a Riemannian manifold $(X,g)$, there exists a unique connection on $\pi:TX\to X$,
    known as the \textbf{Levi-Civita connection}, satisfying
    \begin{enumerate}[(i)]
        \item symmetry:
            \begin{equation*}
                \nabla_XY-\nabla_YX-[X,Y]=0,
            \end{equation*}
            for $X,Y\in\Gamma(X,TX)$;
        \item compatibility with $g$:
            \begin{equation*}
                Xg(Y,Z)-g(\nabla_XY,Z)-g(Y,\nabla_XZ)=0,
            \end{equation*}
            for $X,Y,Z\in\Gamma(X,TX)$.
    \end{enumerate}
\end{theorem}

\begin{definition}
    Let $v$ be a vector field on $(X,g)$; we define the \textbf{covariant derivative} of $v$
    along a smooth curve $c:I\to X$ to be the vector field
    \begin{align*}
        \frac{Dv}{dt} = \nabla_{dc/dt}v,
    \end{align*}
    where $\nabla$ is the Levi-Civita connection. Explicitly, if we write $v=v^i\partial/\partial x^i$
    and $c(t)=(c_1(t),\ldots,c_n(t))$,
    \begin{equation*}
        \frac{Dv}{dt} = \sum_i \frac{dv^i}{dt}\frac{\partial}{\partial x^i} + \sum_{ijk}\frac{dc_i}{dt}v^i\Gamma_{ij}^k\frac{\partial}{\partial x^k}.
    \end{equation*}
    Here $\Gamma_{ij}^k$ are the Christoffel symbols of $\nabla$, determined by
    \begin{equation*}
        \nabla_{\partial/\partial x^i}\frac{\partial}{\partial x^j}=\sum_{ijk}\Gamma^k_{ij}\frac{\partial}{\partial x^k}.
    \end{equation*}
    We say that $c$ is \textbf{geodesic} at some $t\in I$ if $D/dt(dc/dt)=0$ at $t$,
    and that $c$ is geodesic if it is geodesic at all $t\in I$. In coordinates, the condition
    for $c$ to be geodesic is given by a system of second-order differential equations:
    \begin{equation*}
        \frac{d^2 c^i}{dt^2} + \sum_{jk}\Gamma_{jk}^i\frac{dc^j}{dt}\frac{dc^k}{dt}=0,
    \end{equation*}
    for $i=1,\ldots, n.$
\end{definition}

For the rest of the section, assume $(X,g)$ is Riemannian and we fix the Hamiltonian $H:M=T^*X\to\R$ as
\begin{equation*}
    H(x,\xi)=\frac{1}{2}\big|\xi_x\big|^2_g,
\end{equation*}
i.e. consisting of only a kinetic term. Here we are implicitly using the nondegeneracy of $g$
to associate $\xi_x$ with its corresponding vector (or, equivalently, using $g^{-1}$).

\begin{proposition}
    The Hamiltonian flow on $M=T^*X$ is dual to the geodesic flow on $TX$. In other words, the integral
    curves of the Hamiltonian vector field $v_H$ associated to the Hamiltonian above project to geodesics
    of $g$ on $X$.\footnote{Is there a coordinate-free proof? See Paternain's book.}
\end{proposition}
\begin{proof}
    It suffices to show, in coordinates, that Hamilton's equations (i.e. the condition for
    being on the integral curve) yield the geodesic equations above after the necessary
    dualization. Note first that in coordinates the Hamiltonian becomes
    \begin{equation*}
        H(x,\xi) = \frac{1}{2}g^{ij}\xi_i\xi_j.
    \end{equation*}
    For convenience we will denote the components of an integral curve as $x^i(t)$. Hamilton's equations yield
    \begin{align*}
        \frac{dx^i}{dt} &= \frac{\partial}{\partial\xi_i}\left( \frac{1}{2}g^{jk}\xi_j\xi_k \right)\\
        &= \frac{1}{2}g^{jk}\delta_{ij}\xi_k + \frac{1}{2}g^{jk}\xi_j\delta_{ik}\\
        &= g^{ij}\xi_j\\
        \frac{d\xi_i}{dt} &= -\frac{\partial}{\partial x^i}\left( \frac{1}{2}g^{jk}\xi_j\xi_k \right)\\
        &=-\frac{1}{2}\frac{\partial g^{jk}}{\partial x^i}\xi_j\xi_k.
    \end{align*}
    Differentiating the first equation with respect to $t$ and using both of Hamilton's equations yields
    \begin{align*}
        \frac{d^2x^i}{dt^2} &= \frac{\partial g^{ij}}{\partial x^k}\frac{dx^k}{dt}\xi_j+g^{im}\frac{d\xi_m}{dt}\\
        &= g^{kl}\left(\frac{\partial}{\partial x^k}g^{ij}\right)\xi_l\xi_j-\frac{1}{2}g^{im}\left(\frac{\partial}{\partial x^m}g^{nr}\right)\xi_n\xi_r.
    \end{align*}
    Next, differentiating the identity $g^{ij}g_{jk}=\delta^i_k$, it easy to see that
    \begin{align*}
        \frac{\partial}{\partial x^i}g^{kl}=-g^{la}g^{kb}\frac{\partial}{\partial x^i}g_{ab}.
    \end{align*}
    Using this, contracting indices, and using the first Hamilton's equation to dualize $\xi$'s into $dx/dt$'s,
    we find
    \begin{align*}
        \frac{d^2x^i}{dt^2} &= -g^{ib}\left( \frac{\partial}{\partial x^k}g_{lb} \right)\frac{dx^k}{dt}\frac{dx^l}{dt}
        +\frac{1}{2}g^{im}\left( \frac{\partial}{\partial x^m}g_{ts} \right)\frac{dx^s}{dt}\frac{dx^t}{dt}\\
        &= -\frac{1}{2}g^{ib}\left( \frac{\partial}{\partial x^k}g_{lb} \right)\frac{dx^k}{dt}\frac{dx^l}{dt}
        -\frac{1}{2}g^{ib}\left( \frac{\partial}{\partial x^l}g_{kb} \right)\frac{dx^k}{dt}\frac{dx^l}{dt}
        \\&+\frac{1}{2}g^{im}\left( \frac{\partial}{\partial x^m}g_{ts} \right)\frac{dx^s}{dt}\frac{dx^t}{dt}\\
        &= -\Gamma_{kl}^i\frac{dx^k}{dt}\frac{dx^l}{dt},
    \end{align*}
    as desired.
\end{proof}


\newpage
\section{Week 2}

\subsection{Darboux's theorem}

\begin{theorem}[Darboux]
    Let $(M,\omega)$ be a symplectic $2n$-manifold. Then $M$ is locally symplectomorphic to
    $(\R^{2n},\omega_{\R^{2n}})$.
\end{theorem}

We prove Darboux's theorem using the following stronger statement.

\begin{theorem}
    Let $M$ be a $2n$-dimensional manifold and $Q\subset M$ be a compact submanifold. Suppose that
    $\omega_1,\omega_2\in\Omega^2(M)$ are closed 2-forms such that at each point $q$ of $Q$ the forms
    $\omega_0$ and $\omega_1$ are equal and nondegenerate on $T_qM$. Then there exist neighborhoods
    $N_0$ and $N_1$ of $Q$ and a diffeomorphism $\psi:N_0\to N_1$ such that $\psi|_Q=\id_Q$ and
    $\psi^*\omega_1=\omega_0$.
\end{theorem}
\begin{proof}
    Consider the family of closed two-forms
    \begin{equation*}
        \omega_t = \omega_0 + t(\omega_1-\omega_0)
    \end{equation*}
    on $M$ for $t\in[0,1]$. Note that $\omega_t|_Q=\omega_0|_Q$ is nondegenerate and hence there exists
    an open neighborhood $N_0$ of $Q$ such that $\omega_t|_{N_0}$ is nondegenerate.\footnote{Why?} Suppose,
    for now, that there is a one-form $\sigma\in\Omega^1(N_0)$ (possibly shrinking $N_0$), such that
    $\sigma|_{T_QM}=0$ and $d\sigma=\omega_1-\omega_0$ on $N_0$. Then
    \begin{equation*}
        \omega_t = \omega_0+td\sigma
    \end{equation*}
    and we obtain by nondegeneracy a smooth vector field $X_t$ on $N_0$ characterized by
    \begin{equation*}
        \iota_{X_t}\omega_t = -\sigma.
    \end{equation*}
    The condition $\sigma|_{T_QM}=0$ implies, again by nondegeneracy of $\omega_t$, that $X_t|_Q=0$.
    Now consider the initial value problem for the flow $\psi_t$ of $X_t$,
    \begin{align*}
        \frac{d}{dt}\psi_t &= X_t \circ \psi_t\\
        \psi_0 &= \id.
    \end{align*}
    This differential equation can be solved uniquely for $t\in[0,1]$ on some open neighborhood of
    $Q$ contained in $N_0$, call it again $N_0$.\footnote{Why?} Note that $\psi_t|_Q=\id_Q$ since $X_t|_Q=0$.
    We compute now that
    \begin{align*}
        \frac{d}{dt}\psi_t^*\omega_t &= \psi_t^*\left( \frac{d}{dt}\omega_t + \mathcal{L}_{X_t}\omega_t \right)\\
        &= \psi_t^*\left( d\sigma + d\iota_{X_t}\omega_t\right)\\
        &= 0.
    \end{align*}
    Hence $\psi_1^*\omega_1=\psi_0^*\omega_0=\omega_0$. Thus the desired diffeomorphism is
    $\psi_1$ and the desired neighborhoods are $N_0$ and $N_1$.
    The above argument is known as \textbf{Moser's trick}, and is extremely useful in symplectic
    geometry.
    
    It remains to construct a smooth one-form $\sigma$ satisfying $\sigma|_{T_QM}=0$ and $d\sigma=\omega_1-\omega_0$.
    If $Q$ were a point (or more generally, diffeomorphic to a star-shaped subset of Euclidean space),
    we could simply use the Poincar\'e lemma; in general, however the construction is as follows.
    Fix any Riemannian metric on $M$ and consider the restriction of the exponential map
    $\exp: TM\to M$ to a neighbhorhood $U_\varepsilon$ of the zero section of the normal bundle
    $TQ^\perp\to M$:
    \begin{equation*}
        U_\varepsilon = \{(q,v)\in TM \mid q\in Q, v\in T_qQ^\perp, |v|<\varepsilon\}.
    \end{equation*}
    Recall that $\exp$ becomes a diffeomorphism for $\varepsilon$ sufficiently small, so we choose
    $\varepsilon$ such that $N_0=\exp(U_\varepsilon)$ is contained in the neighborhood of $Q$ above
    on which $\omega_t$ is nondegenerate. Define now a family of maps $\phi_t:N_0\to N_0$
    for $t\in[0,1]$ by
    \begin{equation*}
        \phi_t(\exp(q,v)) = \exp(q,tv).
    \end{equation*}
    Note that $\phi_t$ is a diffeomorphism onto its image for $t\neq 0$. Moreover, $\phi_t|_Q=\id_Q$,
    $\phi_0(N_0)$, and $\phi_1 = \id_{N_0}$. If we now write $\tau = \omega_1-\omega_0$, we find that
    \begin{align*}
        \phi_0^*\tau &= 0\\
        \phi_1^*\tau &= \tau,
    \end{align*}
    since $\tau=0$ on $T_QM$. Now, for $t\in(0,1]$, we define a family of vector fields,
    \begin{equation*}
        Y_t = \left( \frac{d}{dt}\phi_t \right)\circ\phi_t^{-1}.
    \end{equation*}
    Then for any $\delta>0$,
    \begin{align*}
        \phi_1^*\tau - \phi_\delta^*\tau &= \int_\delta^1 \frac{d}{dt}\phi_t^*\tau dt=\int_\delta\phi_t^*\mathcal{L}_{Y_t}\tau dt\\
        &= \int_\delta^1 \phi_t^*(d\iota_{Y_t}\tau)dt\\
        &= d\int_\delta^1 \phi_t^*(\iota_{Y_t}\tau)dt
    \end{align*}
    Clearly $\phi_1^*\tau-\phi_\delta^*\tau=\tau-\phi_\delta^*\tau$ approaches $\tau$ as $\delta\to0^+$,
    so we find that
    \begin{equation*}
        \tau=d\int_0^1\phi_t^*(\iota_{Y_t}\tau)dt.
    \end{equation*}
    Defining
    \begin{equation*}
        \sigma = \int_0^1\phi_t^*(\iota_{Y_t}\tau) dt,
    \end{equation*}
    we find that $\tau=\omega_1-\omega_0=d\sigma$ and $\sigma|_{T_QM}=0$ because $\phi_t|_Q=\id_Q$ and
    $\tau=0$ on $Q$, forcing the integrand to vanish on $T_QM$. Hence $\sigma$ is the one-form required
    above for Moser's trick, and we are done.\todo{Why is $\sigma$ smooth?}
    \label{thm:moser}
\end{proof}

The proof of Darboux's theorem is now straightforward: we choose a coordinate chart $\phi$ so that
$\phi^*\omega$ is equal to the standard form on a subset of $\R^{2n}$ at a single point, and then
apply Moser's theorem with $Q$ equal to the chosen point.

\begin{proof}[Proof of Darboux's theorem]
    Let $q\in M$ and fix a symplectic basis $\{u_i,v_i\}$ for the symplectic vector space $(T_qM,\omega_q)$.
    Fix any Riemannian metric on $M$ and pick an open $U\ni 0$ small enough such that $\exp$ restricted to
    $U\subset T_qM$ is a diffeomorphism and hence a chart $(x^i,y_i)=\exp:U\subset\R^{2n}\to M$
    ($i=1,\ldots,n$) such that $x^i(p)=y_i(p)=0$. Now we can compute, for example,
    \begin{align*}
        \exp^*\omega_p\left( \frac{\partial}{\partial x^j},\frac{\partial}{\partial y^k} \right) &=\omega_p\left(\exp_*\frac{\partial}{\partial x^j},\exp_*\frac{\partial}{\partial y^k}\right)\\
        &=\omega_p\left( u_j,v_k \right)=\delta_{jk},
    \end{align*}
    to check that $\exp^*\omega_p=(\omega_0)_0$ where $\omega_0$ is the standard form on $T_0U$.
    Here we have used the fact that $\exp_*=\id$ at $0\in U$.
    Applying Theorem \ref{thm:moser} to $U$ with $Q=0\in U$, we obtain a diffeomorphism $\psi$ of
    (some possibly smaller) $U$ such that $\psi^*\exp^*\omega=\omega_0$ on $U$. But now
    $\exp\circ\psi$ provides a symplectomorphism in a neighborhood of $q$ to a neighborhood of $\R^{2n}$
    pulling $\omega$ back to the standard form $\omega_0$.
\end{proof}

\end{document}
