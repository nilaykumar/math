\documentclass{amsart}

\usepackage[colorlinks=true]{hyperref}
\usepackage{enumerate}
\usepackage{color}
\usepackage{mathrsfs}
\usepackage{tikz-cd}
\usepackage{amssymb}

\usepackage[backend=bibtex, style=alphabetic]{biblatex}
%\addbibresource{bott_periodicity.bib}

\theoremstyle{plain}
\newtheorem{theorem}{Theorem}
\newtheorem{lemma}[theorem]{Lemma}
\newtheorem{proposition}[theorem]{Proposition}
\newtheorem{corollary}[theorem]{Corollary}

\theoremstyle{definition}
\newtheorem{definition}[theorem]{Definition}
\newtheorem{example}[theorem]{Example}
\newtheorem{exercise}[theorem]{Exercise}

\theoremstyle{remark}
\newtheorem{remark}[theorem]{Remark}

% Fonts
\newcommand{\A}{\mathbb{A}}
\newcommand{\C}{\mathbb{C}}
\newcommand{\R}{\mathbb{R}}
\newcommand{\Q}{\mathbb{Q}}
\newcommand{\Z}{\mathbb{Z}}
\newcommand{\N}{\mathbb{N}}
\newcommand{\G}{\mathbb{G}}
\newcommand{\fr}{\mathfrak}

% Topology/geometry

\DeclareMathOperator{\Gr}{Gr}
\DeclareMathOperator{\Fl}{Fl}
\DeclareMathOperator{\PP}{\mathbb{P}}
\DeclareMathOperator{\Der}{Der}
\DeclareMathOperator{\Lie}{Lie}
\DeclareMathOperator{\SL}{SL}
\DeclareMathOperator{\GL}{GL}
\DeclareMathOperator{\SO}{SO}
\DeclareMathOperator{\Sp}{Sp}
\DeclareMathOperator{\HH}{H}
\DeclareMathOperator{\Symp}{Symp}

% Representation theory

\DeclareMathOperator{\Ad}{Ad}

% Algebra

\DeclareMathOperator{\End}{End}
\DeclareMathOperator{\Aut}{Aut}
\DeclareMathOperator{\Hom}{Hom}
\DeclareMathOperator{\sHom}{\mathscr{H}\!om}
\DeclareMathOperator{\sEnd}{\mathscr{E}\!nd}
\DeclareMathOperator{\id}{id}
\DeclareMathOperator{\irr}{irr}
\DeclareMathOperator{\Diff}{Diff}
\DeclareMathOperator{\gr}{gr}
\DeclareMathOperator{\im}{im}
\DeclareMathOperator{\ad}{ad}
\DeclareMathOperator{\rk}{rk}
\DeclareMathOperator{\Spec}{Spec}
\DeclareMathOperator{\Specm}{Specm}
\DeclareMathOperator{\Stab}{Stab}
\DeclareMathOperator{\Sym}{Sym}

% Category theory

\DeclareMathOperator*{\colim}{colim}

\makeatletter
\renewcommand\d[1]{\mspace{6mu}\mathrm{d}#1\@ifnextchar\d{\mspace{-3mu}}{}}
\makeatother



\title{Brown representability and CW-spectra}
\author{Nilay Kumar}
\date{March 4, 2016}

\begin{document}

\maketitle

Outline:
\begin{enumerate}
    \item Generalized (reduced) cohomology theory, from $\Omega$-spectra
    \item Brown representability
    \item every spectrum is weak equivalent to an omega-spectrum (?)
    \item are these categories equivalent?
    \item properties of CW spectra
    \item smash products, axiomatically, ring spectra?
\end{enumerate}

Note that I will write $h$ instead of $\tilde h$, etc. -- all cohomology theories will be reduced.

Last time Sean defined the CW-spectra category and its associated homotopy category, the stable homotopy category.
The main goal for today is to understand the relation between CW-spectra and generalized cohomology theories. Let
us first recall the definitions.

\begin{definition}
    A \textbf{(reduced) generalized cohomology theory} is a functor $h^\bullet:\textsc{CW}_*\to\textsc{Ab}^\Z$ from
    pointed CW-complexes to graded abelian groups together with natural isomorphisms
    \begin{equation*}
        \begin{tikzcd}
            h^\bullet\rar{\sim}& h^{\bullet+1}\circ\Sigma,
        \end{tikzcd}
    \end{equation*}
    called suspension isomorphisms, such that
    \begin{enumerate}
        \item if $f,g:X\to Y$ are homotopic then $f^*=g^*:h^\bullet Y\to h^\bullet X$;
        \item for $\iota:A\to X$ an inclusion and $j:X\to C\iota$ the induced mapping cone, we have an exact sequence
            \begin{equation*}
                \begin{tikzcd}
                    h^\bullet(C\iota)\rar & h^\bullet X\rar & h^\bullet A;
                \end{tikzcd}
            \end{equation*}
        \item for a wedge sum $\bigvee_\alpha X_\alpha$, the inclusions $\iota_\alpha:X_\alpha\hookrightarrow\bigvee_\alpha X_\alpha$
            induce an isomorphism
            \begin{equation*}
                \begin{tikzcd}
                    h^\bullet(\bigvee_\alpha X_\alpha)\rar & \prod_\alpha h^\bullet X_\alpha.
                \end{tikzcd}
            \end{equation*}
    \end{enumerate}
\end{definition}

The examples we've seen so far of such gadgets are ordinary (singular) cohomology and $K$-theory. A few more
examples are cohomotopy and cobordism. Note that in general a generalized cohomology theory need not be multiplicative.
Let us now recall the category of CW-spectra.

\begin{definition}
    A \textbf{CW-spectrum} $E$ is a sequence of pointed CW-complexes $\{E_n\}_{n\in\N}$ together with structure
    maps $\varepsilon_i:\Sigma E_n\to E_{n+1}$ such that $\Sigma E_n$ is mapped isomorphically (as a CW-complex)
    onto a subcomplex of $E_{n+1}$. A subspectrum $A\subset E$ is a CW-spectrum for which each $A_n$ is a subcomplex
    of $E_n$. A morphism $E\to F$ of CW-spectra is an equivalence class of maps $A_n\to F_n$ (commuting with structure
    maps) of cofinal subspectra $A\subset E$.\footnote{Sean explained what this means, as well as the notion of homotopy
        but we won't need these precise definition today.}
\end{definition}

The prototypical example is the suspension spectrum $\Sigma^\infty X$ of a CW-complex $X$, which is given
$E_n = \Sigma^n X$ with the obvious structure maps. The sphere spectrum, for instance, is $\mathbb{S}\equiv\Sigma^\infty S^0$.
Another important class of CW-spectra is that of $\Omega$-spectra, CW-spectra for which the adjoint structure
maps $\hat\varepsilon_n: X_n\to \Omega X_{n+1}$ are weak-equivalences.

We have in fact already seen two $\Omega$-spectra, as Sean pointed out. The first is the Eilenberg-Maclane spectrum,
given $H\Z_n=K(\Z,n)$. That this yields an $\Omega$-spectrum is clear from the path-fibration over $K(\Z,n)$.
The Eilenberg-Maclane spaces have the important property that they represent cohomology, i.e. there is a natural
isomorphism
\begin{equation*}
    H^\bullet X \cong [X, K(n,\Z)].
\end{equation*}

Another example we've seen is the $K$-theory spectrum $BU\times\Z$. Recall that $K^0(X)$ is the group completion of
the monoid of complex vector bundles over (a compact space) $X$ and $K^{-1}(X)$ is defined to be $K^0(\Sigma X)$.
The fact that rank $n$ vector bundles given by homotopy classes of maps to $BU(n)$ yields
\begin{equation*}
    K^0(X) \cong [X, BU\times\Z].
\end{equation*}
It follows from loops-suspension adjunction that
\begin{equation*}
    K^{-1}(X) \cong [X, \Omega(BU\times \Z)] \cong [X, \Omega BU]\cong [X, U],
\end{equation*}
where the last isomorphism comes from the homotopy equivalence $\Omega BU\simeq U$.
Moving further down, $K^{-2}(X)=K^0(\Sigma^2X)$, but Bott periodicity tells us that $\Omega U\simeq BU\times\Z$.
Hence we obtain a 2-periodic sequence of spaces $\{BU\times\Z, \Omega BU,BU\times\Z,\ldots\}$ that obviously form an
$\Omega$-spectrum.

It is natural to ask, now, whether every generalized cohomology theory is representable by an $\Omega$-spectrum.
This is Brown's representability theorem. Before we prove this, however, let us consider the converse question, which
is easier: does every $\Omega$-spectrum yield a generalized cohomology theory?

\begin{proposition}
    Let $E$ be an $\Omega$-spectrum. Then $E^\bullet \equiv [-, E_\bullet]$ is a generalized cohomology theory.
\end{proposition}
\begin{proof}
    For $X$ a CW-complex, the homotopy classes of maps $[X,E_\bullet]$ forms an abelian group. {\color{red} finish}

    That $E^\bullet$ is homotopy-invariant is clear: pullback by $f$ is given by precomposition by $f$ so if
    $f\sim g$ then $f^*\phi\sim g^*\phi$. The suspension isomorphism follows immediately from the fact that
    $E$ is an $\Omega$-spectrum:
    \begin{align*}
        E^{n+1}(\Sigma X) &= [\Sigma X, E^{n+1}]\\
        &\cong [X, \Omega E^{n+1}]\\
        &\cong [X, E^n]\\
        &= E^nX
    \end{align*}
    The wedge axiom is also straightforward: maps $[\bigvee_\alpha X_\alpha, E_\bullet]$ are in natural bijection
    with the product $\prod_\alpha[X_\alpha, E_\bullet]$.
\end{proof}

%\printbibliography

\end{document}
