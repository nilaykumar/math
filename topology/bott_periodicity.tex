\documentclass{amsart}

\usepackage[colorlinks=true]{hyperref}
\usepackage{enumerate}
\usepackage{color}
\usepackage{mathrsfs}
\usepackage{tikz-cd}
\usepackage{amssymb}

\usepackage[backend=bibtex, style=alphabetic]{biblatex}
\addbibresource{bott_periodicity.bib}

\theoremstyle{plain}
\newtheorem{theorem}{Theorem}
\newtheorem{lemma}[theorem]{Lemma}
\newtheorem{proposition}[theorem]{Proposition}
\newtheorem{corollary}[theorem]{Corollary}

\theoremstyle{definition}
\newtheorem{definition}[theorem]{Definition}
\newtheorem{example}[theorem]{Example}
\newtheorem{exercise}[theorem]{Exercise}

\theoremstyle{remark}
\newtheorem{remark}[theorem]{Remark}

% Fonts
\newcommand{\A}{\mathbb{A}}
\newcommand{\C}{\mathbb{C}}
\newcommand{\R}{\mathbb{R}}
\newcommand{\Q}{\mathbb{Q}}
\newcommand{\Z}{\mathbb{Z}}
\newcommand{\N}{\mathbb{N}}
\newcommand{\G}{\mathbb{G}}
\newcommand{\fr}{\mathfrak}

% Topology/geometry

\DeclareMathOperator{\Gr}{Gr}
\DeclareMathOperator{\Fl}{Fl}
\DeclareMathOperator{\PP}{\mathbb{P}}
\DeclareMathOperator{\Der}{Der}
\DeclareMathOperator{\Lie}{Lie}
\DeclareMathOperator{\SL}{SL}
\DeclareMathOperator{\GL}{GL}
\DeclareMathOperator{\HH}{H}

% Representation theory

\DeclareMathOperator{\Ad}{Ad}

% Algebra

\DeclareMathOperator{\End}{End}
\DeclareMathOperator{\Aut}{Aut}
\DeclareMathOperator{\Hom}{Hom}
\DeclareMathOperator{\sHom}{\mathscr{H}\!om}
\DeclareMathOperator{\sEnd}{\mathscr{E}\!nd}
\DeclareMathOperator{\id}{id}
\DeclareMathOperator{\irr}{irr}
\DeclareMathOperator{\Diff}{Diff}
\DeclareMathOperator{\gr}{gr}
\DeclareMathOperator{\im}{im}
\DeclareMathOperator{\ad}{ad}
\DeclareMathOperator{\rk}{rk}
\DeclareMathOperator{\Spec}{Spec}
\DeclareMathOperator{\Specm}{Specm}
\DeclareMathOperator{\Stab}{Stab}
\DeclareMathOperator{\Sym}{Sym}

% Category theory

\DeclareMathOperator*{\colim}{colim}


\DeclareMathOperator{\Vect}{Vect}

\title{Bott periodicity for complex $K$-theory}
\author{Nilay Kumar}
\date{February 15, 2016}

\begin{document}

\maketitle

{\color{red} What assumptions do we need to make on the spaces?}
All vector bundles throughout will be complex.

Recall from last time the definition of $K$-theory.
\begin{definition}
    Let $\Vect:\textsc{HTop}^\text{op}\to\textsc{CRig}$ denote the functor taking any space $X$ to the commutative semiring
    of vector bundles over $X$ with the operations of direct sum and tensor product. Notice that $f:X\to Y$ is taken to
    $f^*:\Vect Y\to\Vect X$. Then, by the usual Grothendieck construction,  there exists a universal functor
    \begin{equation*}
        K:\textsc{HTop}^\text{op}\to\textsc{CRing}
    \end{equation*}
    such that if $R$ is a commutative ring then the diagram
    \begin{equation*}
        \begin{tikzcd}
            \Vect(X)\rar\dar & R\\
            K(X)\urar[dashed]
        \end{tikzcd}
    \end{equation*}
    commutes. It follows that for a map $f:X\to Y$ the diagram
    \begin{equation*}
        \begin{tikzcd}
            \Vect Y\rar{f^*}\dar & \Vect X\dar\\
            K(Y) \rar{Kf} & K(X)
        \end{tikzcd}
    \end{equation*}
    commutes.
\end{definition}

\begin{remark}
    By construction, every element in $K(X)$ is of the form $[E]-[F]$ for $E,F\in\Vect(X)$.
    There exists $G\in\Vect X$ such that $F\oplus G$ is trivial of rank $n$: $F\oplus G\cong\underline{n}$.
    Then $[E]-[F]=[E]+[G]-([F]+[G])=[E\oplus G]-[\underline{n}]$. We conclude that every element
    of $K(X)$ is of the form $[H]-[\underline{n}]$.

    Moreover, suppose $E,F\in\Vect X$ are such that $[E]=[F]$. Then, by definition of the Grothendieck
    construction, there exists $G\in\Vect X$ such that $E\oplus G\cong F\oplus G$. If $G'\in\Vect X$
    is such that $G\oplus G'\cong[\underline{n}]$ then adding $G'$, we find that $E\oplus\underline{n}\cong F\oplus\underline{n}$.
    We conclude that $[E]=[F]$ if and only if $E$ and $F$ are \textit{stably equivalent}, i.e. there exists
    a suitable trivial bundle such that they become equivalent after adding the trivial bundle.
\end{remark}

\begin{definition}
    If $X\in\textsc{HTop}^\text{op}_+$, the inclusion of the basepoint into $X$ yields a map
    $K(X)\to K(+)$. We define the reduced $K$-theory $\tilde K(X)$ to be the kernel
    \begin{equation*}
        \begin{tikzcd}
            0\rar & \tilde K(X)\rar & K(X)\rar & K(+)\rar & 0.
        \end{tikzcd}
    \end{equation*}
    The constant map $X\to +$ yields a natural splitting of this exact sequence
    \begin{equation*}
        K(X) \cong \tilde K(X)\oplus K(+)\cong \tilde K(X)\oplus \Z,
    \end{equation*}
    hence inducing a functor $\tilde K:\textsc{HTop}^\text{op}_+\to\textsc{CRing}.$
    A straightforward computation shows that $[E]=[F]$ in $\tilde K(X)$ if and only if
    $E\oplus\underline{n}\cong F\oplus\underline{m}$ for some $m,n\in\N$.
\end{definition}

Let $A\subset X$. As in cohomology, we wish to obtain a long exact sequence on $K$-theory.
Consider the cofiber sequence associated to $i:A\to X$:
\begin{equation*}
    \begin{tikzcd}
        A\rar{i} & X\rar{j} & Ci\rar{\pi} & \Sigma A\rar{-\Sigma i} & \Sigma X\rar{-\Sigma\pi} & \Sigma Ci\rar{\Sigma^2 i} &\Sigma^2X\rar&\cdots
    \end{tikzcd}
\end{equation*}
where $(-\Sigma i)(x\wedge t)=f(x)\wedge(1-t)$. Applying $K$-theory, we claim:
\begin{proposition}
    For $i:A\hookrightarrow X$ there is a natural long exact sequence
    \begin{equation*}
        \begin{tikzcd}
            \cdots\rar & K^{-1}(X)\rar & K^{-1}(A)\rar & K^0(X,A)\rar & K^0(X)\rar & K^0(A),
        \end{tikzcd}
    \end{equation*}
    where $K^{-n}(X)\equiv K(\Sigma^n X)$ and $K^{-n}(X,A)\equiv K(\Sigma^n Ci)$ for all $n\geqslant0$,
    and the maps are induced from the cofiber sequence above.
\end{proposition}
\begin{proof}
    Work.
\end{proof}

Bott periodicity is a fundamental result stating that $K$-theory is 2-periodic.
\begin{theorem}[Bott periodicity]
    $K^{-n}(X)\cong K^{-n-2}(X)$ for all $n\geqslant0$.
    \label{theorem:bott}
\end{theorem}

In particular, defining 
\begin{align*}
    K^{2n}(X)&\equiv K^0(X)\\
    K^{2n+1}(X)&\equiv K^{-1}(X)
\end{align*}
for $n\geqslant0$,
the data of the long exact sequence for a pair $(X,A)$ reduces to the exact sequence
\begin{equation*}
    \begin{tikzcd}
        K^0(X,A)\rar & K^0(X)\rar & K^0(A)\dar\\
        K^1(A)\uar & K^1(X)\lar & K^1(X,A)\lar
    \end{tikzcd}
\end{equation*}

To prove Bott periodicity, we will use a series of results. We will work with reduced
$K$-theory for convenience.

\begin{lemma}
    Write $BU=\colim_nBU(n)$. Then there is a natural isomorphism of functors $\tilde K$ and $[-,BU]$.
    That is, there is a natural bijection
    \begin{equation*}
        [X, BU] \cong \tilde K(X).
    \end{equation*}
    \label{lemma:class}
\end{lemma}
\begin{proof}
    Recall that $[X, BU(n)]\cong\Vect^n X$, whence
    \begin{align*}
        [X,BU] &= [X,\colim_n BU(n)]\\
        &= \colim_n [X, BU(n)]\\
        &= \colim_n \Vect^n X.
    \end{align*}
    It is straightforward to see that the map $\Vect^n X\to\Vect^{n+1}X$ is given by adding a trivial
    bundle. Hence elements $E,F\in\colim_n\Vect^nX$ -- say, vector bundles of rank $r$ and $s$ respectively --
    are equal in the colimit if and only if there exist $l,k\in\N$ such that
    $E\oplus\underline{l}\cong F\oplus\underline{k}$. Thus there is a natural isomorphism
    $\colim_n\Vect^nX\cong \tilde K(X)$, completing the proof.
    {\color{red} why does colim commute with maps?}
\end{proof}

Assume, for the moment, the following proposition.
\begin{proposition}
    There is a homotopy equivalence $BU\simeq\Omega SU$.
    \label{proposition:may}
\end{proposition}

Then we can prove Bott periodicity as follows.
\begin{proof}[Proof of Theorem \ref{theorem:bott}]
    It suffices to show that there is a natural isomorphism $\tilde K(\Sigma^2X)\cong\tilde K(X)$.
    Using Lemma \ref{lemma:class} and the above proposition, we have natural isomorphisms 
    \begin{align*}
        \tilde K(\Sigma^2X) &\cong [\Sigma^2X,BU]\\
        &\cong [X, \Omega^2BU]\\
        &\cong [X,\Omega SU]\\
        &\cong [X, BU]\\
        &\cong \tilde K(X).
    \end{align*}
    {\color{red} Why is $\Omega BU\simeq SU$?}
\end{proof}

All the hard work lies lies in the proof of the homotopy equivalence $BU\simeq\Omega SU$.
We follow \cite{mcat} in the rest of this note.

\printbibliography

\end{document}
