\documentclass{amsart}

\usepackage[colorlinks=true]{hyperref}
\usepackage{enumerate}
\usepackage{color}
\usepackage{mathrsfs}
\usepackage{tikz-cd}
\usepackage{amssymb}

\theoremstyle{plain}
\newtheorem{theorem}{Theorem}
\newtheorem{lemma}[theorem]{Lemma}
\newtheorem{proposition}[theorem]{Proposition}
\newtheorem{corollary}[theorem]{Corollary}

\theoremstyle{definition}
\newtheorem{definition}[theorem]{Definition}
\newtheorem{example}[theorem]{Example}
\newtheorem{exercise}[theorem]{Exercise}

\theoremstyle{remark}
\newtheorem{remark}[theorem]{Remark}

% Fonts
\newcommand{\A}{\mathbb{A}}
\newcommand{\C}{\mathbb{C}}
\newcommand{\R}{\mathbb{R}}
\newcommand{\Q}{\mathbb{Q}}
\newcommand{\Z}{\mathbb{Z}}
\newcommand{\N}{\mathbb{N}}
\newcommand{\G}{\mathbb{G}}
\newcommand{\fr}{\mathfrak}

% Topology/geometry

\DeclareMathOperator{\Gr}{Gr}
\DeclareMathOperator{\Fl}{Fl}
\DeclareMathOperator{\PP}{\mathbb{P}}
\DeclareMathOperator{\Der}{Der}
\DeclareMathOperator{\Lie}{Lie}
\DeclareMathOperator{\SL}{SL}
\DeclareMathOperator{\GL}{GL}
\DeclareMathOperator{\HH}{H}

% Representation theory

\DeclareMathOperator{\Ad}{Ad}

% Algebra

\DeclareMathOperator{\End}{End}
\DeclareMathOperator{\Aut}{Aut}
\DeclareMathOperator{\Hom}{Hom}
\DeclareMathOperator{\sHom}{\mathscr{H}\!om}
\DeclareMathOperator{\sEnd}{\mathscr{E}\!nd}
\DeclareMathOperator{\id}{id}
\DeclareMathOperator{\irr}{irr}
\DeclareMathOperator{\Diff}{Diff}
\DeclareMathOperator{\gr}{gr}
\DeclareMathOperator{\im}{im}
\DeclareMathOperator{\ad}{ad}
\DeclareMathOperator{\rk}{rk}
\DeclareMathOperator{\Spec}{Spec}
\DeclareMathOperator{\Specm}{Specm}
\DeclareMathOperator{\Stab}{Stab}
\DeclareMathOperator{\Sym}{Sym}

% Category theory

\DeclareMathOperator*{\colim}{colim}


\DeclareMathOperator{\Vect}{Vect}

\title{Characteristic classes II}
\author{Nilay Kumar}
\date{January 20, 2016}

\begin{document}

\maketitle

\section{Stiefel-Whitney classes}

Recall from last time the axiomatic definition of the Stiefel-Whitney classes of a vector bundle
$\xi:E\to B$:\footnote{Throughout, all base spaces are assumed to be connected and paracompact.}
\begin{theorem}
    Let $\xi:E\to B$ be a real vector bundle. Then there exists a unique sequence of cohomology classes
    \begin{equation*}
        w_i(\xi) \in \HH^i(B, \F_2)
    \end{equation*}
    for $i=0,1,2,\ldots$ called the \textit{Stiefel-Whitney classes} of $\xi$ satisfying the following
    properties:
    \begin{enumerate}[(I)]
        \item the class $w_0(\xi)$ is equal to the unit element $1\in\HH^0(B,\F_2)$ and $w_i(\xi)=0$ for $i>\rk\xi$;
        \item if $f^*\xi$ is the pullback of $E$ along $f:A\to B$ then $w_i(f^*\xi)=f^*w_i(\xi)$;
        \item if $\eta:E'\to B$ is another real vector bundle then
            \begin{equation*}
                w_k(\xi\oplus \eta) = \sum_{i=0}^k w_i(\xi)\smile w_{k-i}(\eta);
            \end{equation*}
        \item if $\gamma_1^1$ is the tautological line bundle over $\R P^1$ then $w_1(\gamma_1^1)\in\HH^1(\R P^1,\F_2)=\F_2$
            is the unique nonzero element.
    \end{enumerate}
    The total (inhomogeneous) Stiefel-Whitney class of $\xi$ is the sum
    \begin{equation*}
        w(\xi) = \sum_{i=0}^{\rk\xi}w_i(\xi) = 1 + w_1(\xi) + \cdots + w_{\rk\xi}(\xi)\in\HH^\bullet(B,\F_2).
    \end{equation*}
    \label{sw}
\end{theorem}

%{\color{magenta} make a remark about bundle map diagrams just being pullback diagrams, as Dylan pointed out
%last time. so we don't lose anything on II}

Thus given, the Stiefel-Whitney classes allowed us to make some strong statements about parallelizability
and cobordisms. The goal of the first half of this talk is to sketch a proof of the above theorem, i.e.
show that Stiefel-Whitney classes do indeed exist. We will first prove the existence of certain analogous
classes:
\begin{theorem}
    Let $\iota_n:\OO(n-1)\to\OO(n)$ and $p_{ij}:\OO(i)\times \OO(j)\to \OO(i+j)$ be the obvious
    inclusions. Denote by the same symbols the induced maps on classifying spaces.
    Then there are unique classes $w_i\in \HH^i(\mathscr{B}\OO(n), \F_2)$ satisfying:
    \begin{enumerate}[(I)]
        \item $w_0=1$ and $w_i=0$ if $i>n$;
        \item $\iota_n^*w_i=w_i$ (and hence $\iota_n^*w_n=0$);
        \item $p_{ij}^*w_k=\sum_{a+b=k}w_a\otimes w_b$;
        \item $w_1\in\HH^1(\mathscr{B}\OO(1),\F_2)=\HH^1(\R P^\infty,\F_2)$ is the unique nonzero element.
    \end{enumerate}
    \label{sw2}
\end{theorem}
Here, as usual,
\begin{equation*}
    \mathscr{B}:\textsc{Grp}\to\textsc{Top}
\end{equation*}
denotes the classifying space functor.
%We can embed $O(1)^n\hookrightarrow O(n)$ as the subgroup of diagonal matrices. Then, since
%$\mathscr{B}$ is a product-preserving functor (up to homotopy), we obtain a map
%\begin{equation*}
%    \omega: (\R P^\infty)^n \simeq \mathscr{B}(O(1)^n) \to \mathscr{B}\OO(n)
%\end{equation*}
%of classifying spaces.
Almost all the work lies in computing the cohomology of $\mathscr{B}\OO(n)$. In an earlier talk,
Guchuan mentioned that -- via a number of computations using spectral sequences -- one
can show the following:
\begin{lemma}
    Let $\Sigma_n$ be the symmetric group on $n$ letters. Then there is a map $\Psi_n:(\R P^\infty)^n\to\mathscr{B}\OO(n)$
    inducing an isomorphism
    \begin{equation*}
        \Psi_n^*:\HH^\bullet(\mathscr{B}\OO(n),\F_2) \cong \HH^\bullet( (\R P^\infty)^n,\F_2)^{\Sigma_n}\cong\F_2[\sigma_1,\ldots,\sigma_n],
    \end{equation*}
    where $\Sigma_n$ acts on $(\R P^\infty)^n$ by permutation and hence $\sigma_i$ are the symmetric
    polynomials in $n$ variables with $\deg\sigma_i=i$.
    \label{computation}
\end{lemma}
\begin{proof}
    This is rather technical and requires more wizardry with spectral sequences than I am familiar with. 
    Reference {\color{magenta}unfinished May}.
\end{proof}
With this in hand, the proof of Theorem \ref{sw2} is now a straightforward diagram chase
with symmetric polynomials.
\begin{proof}[Proof of Theorem \ref{sw2}]
    We begin by proving existence of the classes $w_i$.
    Define the Stiefel-Whitney classes as
    \begin{equation*}
        w_i \equiv (\Psi_n^*)^{-1}\sigma_i
    \end{equation*}
    where $\Psi_n^*$ is the isomorphism from Lemma \ref{computation} above, and where
    \begin{align*}
        \sigma_1 &= x_1+\cdots+x_n\\
        \sigma_2 &= x_1x_2+x_1x_3+\cdots+x_{n-1}x_n\\
        &\vdots\\
        \sigma_n &= x_1\cdots x_n,
    \end{align*}
    are the symmetric polynomials on the generators of
    $\HH^\bullet( (\R P^\infty)^n,\F_2)\cong\otimes_i^n\F_2[x_i]$.
    Set $\sigma_0=1$ and $\sigma_i=0$ for $i>n$. Under these definitions, axiom I is immediate.

    Next, denoting $h_n:\mathscr{B}\OO(1)^{n-1}\to\mathscr{B}\OO(1)^n$ the map induced by
    the inclusion, we have a commutative diagram of inclusions
    \begin{equation*}
        \begin{tikzcd}
            \mathscr{B}\OO(1)^{n-1}\ar{r}{\Psi_{n-1}}\ar{d}{h_n} & \mathscr{B}\OO(n-1)\ar{d}{\iota_n}\\
            \mathscr{B}\OO(1)^n\ar{r}{\Psi_n} & \mathscr{B}\OO(n)
        \end{tikzcd}
    \end{equation*}
    and taking cohomology,
    \begin{equation*}
        \begin{tikzcd}
            \HH^\bullet(\mathscr{B}\OO(n))\ar{r}{\Psi_n^*}\ar{d}{\iota_n^*} & \HH^\bullet(\mathscr{B}\OO(1)^n)\ar{d}{h_n^*}\\
            \HH^\bullet(\mathscr{B}\OO(n-1))\ar{r}{\Psi_{n-1}^*} & \HH^\bullet(\mathscr{B}\OO(1)^{n-1})
        \end{tikzcd}
    \end{equation*}
    Clearly $h_n^*x_i=x_i$ for $i<n$ and $h_n^*x_n=0$. Therefore $h_n^*\sigma_i=\sigma_i$, which implies
    -- by the diagram above -- that $\iota_n^*w_i=w_i$. So much for axiom II.

    Consider now the commutative diagram of inclusions
    \begin{equation*}
        \begin{tikzcd}
            \mathscr{B}\OO(1)^i\times\mathscr{B}\OO(1)^j\ar{r}{\Psi_i\times\Psi_j}\ar[equal]{d} & \mathscr{B}\OO(i)\times\mathscr{B}\OO(j)\ar{d}{p_{ij}}\\
            \mathscr{B}\OO(1)^{i+j}\ar{r}{\Psi_{i+j}} & \mathscr{B}\OO(i+j)
        \end{tikzcd}
    \end{equation*}
    which after taking cohomology and applying K\"unneth becomes
    \begin{equation*}
        \begin{tikzcd}
            \HH^\bullet(\mathscr{B}\OO(i+j))\ar{r}{\Psi_{i+j}^*}\ar{d}{p_{ij}^*} & \HH^\bullet(\mathscr{B}\OO(1)^{i+j})\ar{d}\\
            \HH^\bullet(\mathscr{B}\OO(i))\otimes\HH^\bullet(\mathscr{B}\OO(j))\ar{r}{\Psi_i^*\otimes\Psi_j^*} & \HH^\bullet(\mathscr{B}\OO(1)^i)\otimes\HH^\bullet(\mathscr{B}\OO(1)^j)
        \end{tikzcd}
    \end{equation*}
    In this diagram, all the arrows are injective except for $p_{ij}^*$ and hence $p_{ij}^*$
    is injective. Moreover,
    \begin{equation*}
        (\Psi_i^*\otimes\Psi_j^*) p_{ij}^*w_k = \Psi_{i+j}^*w_k = \sigma_k(x_1,\ldots,x_{i+j}).
    \end{equation*}
    Some algebra with symmetric polynomials reveals that
    \begin{equation*}
        \sigma_k(x_1,\ldots,x_{i+j}) = \sum_{a+b=k}\sigma_a(x_1,\ldots,x_i)\sigma_b(x_{i+1},\ldots,x_{i+j}),
    \end{equation*}
    whence
    \begin{equation*}
        (\Psi_i\otimes\Psi_j^*)p_{ij}^*w_k = \sum_{a+b=k}\Psi_i^*w_a\otimes\Psi_j^*w_b = (\Psi_i^*\otimes\Psi_j^*)\sum_{a+b=k}w_a\otimes w_b,
    \end{equation*}
    proving axiom III by injectivity of $\Psi_i\otimes\Psi_j$.

    Axiom IV is clear: $w_1=(\Psi_1^*)^{-1}\sigma_1(x_1)=x_1$, the nonzero element in $\HH^1(\R P^\infty,\F_2)$.
    Finally, we prove uniqueness by induction on $n$. The base case $n=1$ is trivial.
    Assume uniqueness of the $w_i$ in $\HH^\bullet(\mathscr{B}\OO(m),\F_2)$ for $m<n$.
    Then for $i<n$ the $w_i\in\HH^\bullet(\mathscr{B}\OO(n),\F_2)$ are uniquely determined by
    axiom II and the fact that $\iota_n$ is an isomorphism in degrees smaller than $n$.
    For $i=n$ we note that $p_{1,n-1}^*w_n\in\HH^\bullet(\mathscr{B}\OO(1),\F_2)\otimes\HH^\bullet(\mathscr{B}\OO(n-1),\F_2)$
    and hence $w_n$ is determined by the induction hypothesis since $p_{1,n-1}^*$ is injective.
    This completes the proof.
\end{proof}

To relate these classes sitting in the cohomology of $\mathscr{B}\OO(n)$ to the previous
axiomatic definition of classes sitting in the cohomology of the base $B$, we need a classification
theorem for vector bundles on $B$.
Before we start, we note that
\begin{equation*}
    \Vect_\R^n:\textsc{Top}^\mathrm{op}\to\textsc{Set}
\end{equation*}
will denote the contravariant functor taking $B$ to the set (isomorphism classes) of real vector bundles
over $B$ and taking $f:B\to B'$ to the pullback $f^*:\Vect_\R^n B'\to \Vect_\R^n B$. Recall that two
vector bundles over $B$ are isomorphic if there is a map lifting $\id_B$ that is a fiberwise linear
isomorphism.

\begin{theorem}
    The space $\mathscr{B}\OO(n)\cong\Gr_n\R^\infty$ classifies rank $n$ real vector bundles, i.e. the natural
    transformation
    \begin{equation*}
        \Phi: [-,\mathscr{B}\OO(n)] \naturalto \Vect_\R^n-,
    \end{equation*}
    given by pullback $[f]\mapsto f^*\gamma^n_\infty$ of the tautological bundle, is a natural isomorphism.
    \label{class}
\end{theorem}

We first check that $\Phi$ is well-defined:
\begin{lemma}
    The pullbacks of a vector bundle along homotopic maps are isomorphic, i.e.  the functor $\Vect_\R$ factors
    through the homotopy category.
\end{lemma}
\begin{proof}[Proof sketch]
    Let $\xi:E\to B$ be a rank $n$ vector bundle and let $f,g:A\to B$ be two maps homotopic via
    $h:A\times I\to B$. Note first that $h^*E|_{A\times\{0\}}=f^*E$ and $h^*E|_{A\times\{1\}}=g^*E$.
    %\begin{align*}
    %    h^*E| &= \{(e,(a,t))\in E\times A\times I \mid \xi(e)=h(a,t)\}\\
    %    h^*E|_{A\times\{0\}} &= \{(e,a)\in E\times A \mid \xi(e)=h(a,0)=f(a)\}.
    %\end{align*}
    Thus it suffices to prove that for a vector bundle $\eta:F\to A\times I$, there is an isomorphism
    $F|_{A\times\{0\}}\cong F|_{A\times\{1\}}$. The idea, roughly, is to find countably many local
    trivializations over $U_i\subset B$ for $E$ and to then locally push
    $F|_{A\times\{0\}}$ to the right along $U_i\times I$. For details, see {\color{magenta}Hatcher, VBKT}.
\end{proof}

%Before we prove the theorem, let us outline the ideas in the case  of the tangent
%bundle $TM$ of a smooth compact $n$-manifold embedded in $\R^k$.

\begin{proof}[Proof of Theorem \ref{class}]
    Naturality of $\Phi$ follows immediately from the fact that if $\alpha:A\to B$ and
    $f\in[B,\mathscr{B}\OO(n)]$ then $(f\circ\alpha)^*\gamma_n^\infty=\alpha^*f^*\gamma_n^\infty$.
    We now prove that $\Phi_B:[B,\mathscr{O}(n)]\to\Vect_\R B$ is a bijection.

    The key observation is as follows.  Let $\xi:E\to B$ be a rank $n$ vector bundle. Then an isomorphism
    $E\cong f^*\gamma_n^\infty$ (for some map $f:B\to\Gr_n\R^\infty$) is equivalent to a map
    $g:E\to\R^\infty$ that is a linear injection on each fiber. To see this, suppose first that we have
    such an isomorphism. Then we have a commutative diagram
    \begin{equation*}
        \begin{tikzcd}
            E\ar{d}{\xi}\cong f^*\gamma_n^\infty\ar{d}\ar{r}{\tilde f}&\gamma_n^\infty\ar{r}{\pi}\ar{d} & \R^\infty\\
            B\ar{r}{f} & \Gr_n\R^\infty
        \end{tikzcd}
    \end{equation*}
    where $\pi$ is the obvious projection. Now $\pi\circ\tilde f:E\to\R^\infty$ is a fiberwise linear
    injection as both $f$ and $\pi$ are. Conversely, given $g:E\to\R^\infty$ a fiberwise linear
    injection, we can define a map $f:B\to\Gr_n\R^\infty$ given by $x\mapsto [g(\xi^{-1}(x))]$.
    Then $E\cong f^*\gamma_n^\infty$ because we have fiberwise linear isomorphisms
    \begin{equation*}
        f^*\gamma_n^\infty|_b \cong \gamma_n^\infty|_{f(b)}\cong E|_b.
    \end{equation*}

    Now, for surjectivity of $\Phi_B$, given $\xi:E\to B$ it suffices by the previous paragraph to
    construct a map $E\to\R^\infty$ a linear injection on each fiber. To do this, we fix countably
    many local trivializations over $U_i\subset B$ of $E$ together with partitions of unity $\phi_i$
    subordinate to the $U_i$. Then for each $i$ we obtain a map $g_i:E\to\R^n$ that is zero outside
    $\xi^{-1}U_i$ and the composition $E\to U_i\times\R^n\to\R^n$ otherwise. Summing $g=\sum g_i$,
    we obtain a map $g:E\to (\R^n)^\infty\cong\R^\infty$ that is obviously a linear injection on
    fibers.

    For injectivity, suppose we have isomorphisms $E\cong f_0^*\gamma_n^\infty$ and
    $E\cong f_1^*\gamma_n^\infty$ for two maps $f_0,f_1:B\to\Gr_n\R^\infty$. By arguments above,
    we obtain maps $g_0,g_1:E\to\R^\infty$ that are fiberwise linear injections. We claim that
    $g_0$ and $g_1$ are homotopic through maps $g_t$ that are fiberwise linear injections; this
    implies that $f_0$ and $f_1$ are homotopic via $f_t(x)=g_t(\xi^{-1}x)$. To do this, we first
    homotope $g_0$ so that it takes values only in odd coordinates via
    \begin{equation*}
        (x_1,x_2,\ldots) \mapsto (1-t)(x_1,x_2,\ldots)+t(x_1,0,x_2,0,\ldots)
    \end{equation*}
    and homotope $g_1$ so that it takes values only in even coordinates similarly. Now
    $g_t=(1-t)g_0+tg_1$ provides the necessary homotopy and it is clearly linear and injective
    on fibers.
\end{proof}

We can finally prove the existence of Stiefel-Whitney classes for vector bundles.

\begin{proof}[Proof of Theorem \ref{sw}]
    Let $\xi:E\to B$ be a real vector bundle of rank $n$. By Theorem \ref{class} there exists a
    unique map $\Phi_B:B\to\mathscr{B}O(n)$ such that $E\cong f^*\gamma_n^\infty$.
    Define
    \begin{equation*}
        w_i(\xi) \equiv \Phi_B^*w_i.
    \end{equation*}
    Axiom I now follows immediately from Theorem \ref{sw2}. Now suppose we have a pullback
    diagram
    \begin{equation*}
        \begin{tikzcd}
            f^*E\ar{r}\ar{d} & E\ar{d}{\xi}\\
            A\ar{r}{f} & B
        \end{tikzcd}
    \end{equation*}
    Then
    \begin{equation*}
        w_i(f^*\xi) = \Phi_A^*w_i = f^*\Phi_B^*w_i = f^*w_i(\xi),
    \end{equation*}
    which proves axiom II.

    Invoking K\"unneth and axiom III of Theorem \ref{sw2}, the diagram
    \begin{equation*}
        \begin{tikzcd}
            B\ar{r}{\xi\times\eta}\ar{d}{\Delta} & \mathscr{B}\OO(i)\times\mathscr{B}\OO(j)\ar{r}{p_{ij}} & \mathscr{B}\OO(i+j)\\
            B\times B\ar[swap]{ru}{f_\xi\times f_\eta},
        \end{tikzcd}
    \end{equation*}
    proves axiom III.

    The tautological line bundle on $\R P^1$ is given by the pullback of $\gamma_1^\infty$
    along the inclusion $j:\R P^1\hookrightarrow\R P^\infty$ so $w_1(\gamma_1^1)=j^*w_1$
    is the unique nonzero element in $\HH^1(\R P^1,\Z/2)$ by axiom IV of Theorem \ref{sw2}
    and because $j^*$ is an isomorphism in degrees $\leqslant 1$.

    The proof of uniqueness is essentially identical to the proof in Theorem \ref{sw2}.
\end{proof}


Recall that a vector bundle is called \textit{orientable} if there is an assignment
of orientation to each fiber as well as orientation-preserving local trivializations.
\begin{proposition}
    Let $B$ be a connected CW complex and let $\xi:E\to B$ be a real vector bundle.
    Then $E$ is orientable if and only if $w_1(\xi)=0$.
\end{proposition}
\begin{proof}
    Let $f:B\to\mathscr{B}\OO(n)$ be the map such that $\xi=f^*\gamma_n^\infty$.
    Then, by the universal coefficient theorem and the fact that $\HH_1(B)$ is the abelianization
    of $\pi_1(B)$, we have a commutative diagram
    \begin{equation*}
        \begin{tikzcd}
            \HH^1(\mathscr{B}\OO(n),\Z_2)\ar{r}{\sim}\ar{d}{f^*} & \Hom(\HH_1(\mathscr{B}\OO(n)),\Z_2)\ar{r}{\sim}\ar{d}{\circ f_*} & \Hom(\pi_1(\mathscr{B}\OO(n)),\Z_2)\ar{d}{\circ f_*}\\
            \HH^1(B,\Z_2)\ar{r}{\sim} & \Hom(\HH_1(B),\Z_2)\ar{r}{\sim} & \Hom(\pi_1(B),\Z_2)
        \end{tikzcd}
    \end{equation*}
    where the horizontal arrows are isomorphisms. Note, now that $\pi_1(\mathscr{B}\OO(n))\cong\Z_2$,
    and so $w_1\in\HH^1(\mathscr{B}\OO(n),\Z_2)$ corresponds to $\id_{\Z_2}\in\Hom(\pi(\mathscr{B}\OO(n)),\Z_2)$.
    Hence $w_1(\xi)=f^*w_1$ corresponds to a map $\pi_1(B)\to\Z_2$ that is trivial if and only if
    $f_*=0$. This is precisely the condition for $f$ to lift to the universal cover
    $\mathscr{B}\SO(n)=\widetilde{\mathscr{B}\OO(n)}$. We conclude that, since $\mathscr{B}\SO(n)$ is the
    classifying space for orientable rank $n$ bundles, $w_1(\xi)=0$ if and only if $E$ is orientable.
\end{proof}

\section{Chern classes}

Chern classes are the complex analog of Stiefel-Whitney classes, but with integral coefficients.
\begin{theorem}
    Let $\iota_n:\UU(n-1)\to\UU(n)$ and $p_{ij}:\UU(i)\times \UU(j)\to \UU(i+j)$ be the obvious
    inclusions. Denote by the same symbols the induced maps on classifying spaces.
    Then there are unique \textit{Chern classes} $c_i\in \HH^{2i}(\mathscr{B}\UU(n), \Z)$ satisfying:
    \begin{enumerate}[(I)]
        \item $c_0=1$ and $c_i=0$ if $i>n$;
        \item $\iota_n^*c_i=c_i$ (and hence $\iota_n^*c_n=0$);
        \item $p_{ij}^*c_k=\sum_{a+b=k}c_a\otimes c_b$;
        \item $c_1\in\HH^2(\mathscr{B}\UU(1),\Z)=\HH^2(\C P^\infty,\Z)$ is the canonical generator.
    \end{enumerate}
    \label{chern}
\end{theorem}
\begin{proof}
    The proof is exactly the same as in the case of Stiefel-Whitney classes.
    We note, in particular, that $\HH^\bullet(\mathscr{B}\UU(n),\Z)=\Z[c_1,\ldots,c_n]$.
\end{proof}

Chern classes can be related to complex vector bundles via the following analog of Theorem
\ref{class}. Notice that here $\Gr_n V$ denotes the space of complex $n$-planes in $V$.
\begin{theorem}
    The space $\mathscr{B}\UU(n)\cong\Gr_n\C^\infty$ classifies complex vector bundles, i.e. the natural
    transformation
    \begin{equation*}
        \Phi: [-,\mathscr{B}\UU(n)] \naturalto \Vect_\C-,
    \end{equation*}
    given by pullback $[f]\mapsto f^*\gamma^n_\infty$ of the tautological bundle, is a natural isomorphism.
\end{theorem}

We can now define the Chern class of a complex vector bundle.
\begin{theorem}
    Let $\xi:E\to B$ be a complex vector bundle. Then there exists a unique sequence of cohomology classes
    \begin{equation*}
        c_i(\xi) \in \HH^{2i}(B, \Z)
    \end{equation*}
    for $i=0,1,2,\ldots$ called the \textit{Chern classes} of $\xi$ satisfying the following
    properties:
    \begin{enumerate}[(I)]
        \item the class $c_0(\xi)$ is equal to the generator $1\in\HH^0(B,\Z)$ and $c_i(\xi)=0$ for $i>\rk\xi$;
        \item if $f^*\xi$ is the pullback of $E$ along $f:A\to B$ then $c_i(f^*\xi)=f^*c_i(\xi)$;
        \item if $\eta:E'\to B$ is another complex vector bundle then
            \begin{equation*}
                c_k(\xi\oplus \eta) = \sum_{i=0}^k c_i(\xi)\smile c_{k-i}(\eta);
            \end{equation*}
        \item if $\gamma_1^1$ is the tautological complex line bundle over $\C P^1$ then $c_1(\gamma_1^1)\in\HH^2(\C P^1,\Z)=\Z$
            is the canonical generator.
    \end{enumerate}
    The total (inhomogeneous) Chern class of $\xi$ is the sum
    \begin{equation*}
        c(\xi) = \sum_{i=0}^{\rk\xi}c_i(\xi) = 1 + c_1(\xi) + \cdots + c_{\rk\xi}(\xi)\in\HH^\bullet(B,\Z).
    \end{equation*}
    \label{sw}
\end{theorem}

Last time, Yajit computed the total Stiefel-Whitney class of the tangent bundle $T\R P^n$ to be
$w(\tau) = (1+a)^{n+1}$, where $a$ is the generator of $\HH^\bullet(\R P^n,\Z_2)$.
This computation extends almost identically to the complex case, but we first make a remark
about conjugate bundles.
\begin{remark}
    If $\xi$ is a complex vector bundle then there is a \textit{conjugate bundle}
    $\bar\xi$, which is the the underlying real vector bundle of $\xi$ equipped with the opposite
    complex structure. Swapping the complex structure is a nontrivial operation! In particular,
    if $\tau$ is the tangent bundle of $\C P^1$ then an isomorphism $\tau\to\bar\tau$ of complex
    vector bundles would consist of a reflection across a line at each tangent space. This implies
    the existence of a continuous nonvanishing vector field on $S^2$, so we conclude that
    $\tau\not\cong\bar\tau$.

    We note, in particular, that
    \begin{equation*}
        c_k(\bar\xi) = (-1)^k c_k(\xi)
    \end{equation*}
    and that in the presence of a Hermitian metric, $\bar\xi$ is canonically isomorphic to the
    dual bundle $\Hom(\xi,\underline{\C})$ (under the assignment $v\mapsto \langle -,v\rangle$).
\end{remark}

\begin{example}[The total Chern class of $T\C P^n$]
    Recall that the tangent bundle $\tau:T\C P^n\to\C P^n$ is identified with the bundle
    $\Hom(\gamma_n^1, (\gamma_n^1)^\perp),$ and so
    \begin{align*}
        \tau\oplus \underline{\C} &\cong \tau\oplus\Hom(\gamma_n^1,\gamma_n^1)\\
        &\cong \Hom(\gamma^1_n, (\gamma^1_n)^\perp\oplus \gamma_n^1)\\
        &\cong \Hom(\gamma^1_n, \underline{\C}^{n+1})\\
        &\cong \oplus^{n+1}\overline{\gamma_n^1}
    \end{align*}
    We conclude that
    \begin{equation*}
        c(\tau)=c(\tau\oplus\underline{\C})=c(\overline{\gamma_n^1})^{n+1}=(1-c_1(\gamma_n^1))^{n+1}=(1+a)^{n+1},
    \end{equation*}
    where we have taken $a=-c_1(\gamma_n^1)$.
\end{example}

Taking the underlying bundle of any complex rank $n$ vector bundle yields a real rank $2n$ vector bundle.
Hence it is natural to ask how Stiefel-Whitney classes are related to Chern classes.
\begin{proposition}
    Let $\mu_n:\mathscr{B}\UU(n) \to \mathscr{B}\OO(2n)$ be the map induced by the natural
    inclusion. Then $\mu_n^*w_{2i+1}=0$ and $\mu_n^*w_{2i}=c_i$.
\end{proposition}
\begin{proof}
    That $\mu^*_nw_{2i+1}=0$ is immediate from axiom II together with the fact that $\mathscr{B}\UU(n)$
    has cells only in even dimensions. For the other case, we recall that
    \begin{align*}
        \HH^\bullet(\mathscr{B}\OO(2n),\Z_2) &\cong \HH^\bullet(\mathscr{B}\OO(2)^n,\Z_2)^{\Sigma_n}\cong\Z_2[x_1,\ldots,x_{2n}]^{\Sigma_n}\\
        \HH^\bullet(\mathscr{B}\UU(n),\Z_2) &\cong \HH^\bullet(\mathscr{B}\UU(1)^n,\Z_2)^{\Sigma_n}\cong\Z_2[y_1,\ldots,y_n]^{\Sigma_n},
    \end{align*}
    where $\deg x_{2i+1}=1, \deg x_{2i}=2$ and $\deg y_i=2$. We note that $\mu_1^*x_1=0$ since $\mathscr{B}\UU(1)$
    has cells only in even degrees and $\mu_1^*x_2=y_1$ ({\color{magenta}why?}).
    Write $\sigma_i$ for the symmetric polynomials in $x_i$ and $\rho_i$ the symmetric polynomials in $y_i$.
    Using the commutativity of the diagram,
    \begin{equation*}
        \begin{tikzcd}
            \HH^\bullet(\mathscr{B}\OO(2n),\Z_2)\ar{r}{\mu_n^*}\ar{d} & \HH^\bullet(\mathscr{B}\UU(n),\Z_2)\ar{d}\\
            \HH^\bullet(\mathscr{B}\OO(2)^n,\Z_2)^{\Sigma_n}\ar{r}{(\mu_1^*)^n} & \HH^\bullet(\mathscr{B}\UU(1)^n,\Z_2)^{\Sigma_n}
        \end{tikzcd}
    \end{equation*}
    it now follows that $\mu_n^*\sigma_i=\rho_i$.
\end{proof}

\end{document}
